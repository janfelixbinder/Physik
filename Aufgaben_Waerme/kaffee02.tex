
\begin{aufgabe}
In guten Bars werden Kaffeetassen vor dem Gebrauch angewärmt. Dadurch kühlt der Kaffee nicht so schnell ab.
Der Kaffee kommt mit einer Temperatur von \SI{95}{\celsius} aus der Maschine.
Die Tassen haben eine Masse von \SI{100}{g} und eine spezifische Wärmekapazität von \SI{730}{J kg^{-1} K^{-1}}.

Bestimmen Sie die Temperatur des Kaffes (\SI{1}{dl}).
\begin{enumerate} [a)]
	\item Wenn die Tasse nicht vorwärmt (\SI{20}{\celsius}) wird?
	\item Wenn die Tasse auf \SI{45}{\celsius} vorgewärmt wird.
\end{enumerate}

\begin{loesung}
	Die Wärmemenge, die aus dem Kaffee in die Tasse fliesst, ist gleich gross wie die Wärmemenge, die die Tasse aufnimmt.
	Es fliesst so lange Wärme, bis Kaffee und Tasse die selbe Temperatur haben.
	Man kann also schreiben:
	\begin{eqnarray*}
		Q=\RI{C}{W}\cdot m\cdot(\vartheta_K - \vartheta_1) = -\RI{c}{T}\cdot m\cdot (\vartheta_T - \vartheta_1)
	\end{eqnarray*}<++>
\end{loesung}<++>

\end{aufgabe}

