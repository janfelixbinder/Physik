\begin{aufgabe}
\StartLueckentext

Steigt die Temperatur eines Körpers, so nimmt seine innere Energie zu. Dem Körper wird also \gl{Wärme} zugeführt.
Wie viel Wärme \gl{$Q$} muss einem Körper mit der Masse \gl{$m$} zugeführt werden, damit er eine Temperaturänderung \gl{$\Delta \vartheta$} erfährt?

Je grösser die Masse desto \gl{mehr} Wärme braucht es: Q \gl{$\sim$} m.

Je grösser die Temperaturänderung, umso \gl{mehr} Wärme braucht es: Q \gl{$\sim$} $\Delta T$.
Zusammenfassend ergibt sich folgender Zusammenhäng:
\begin{eqnarray*}
	Q &\sim\phantom{c\cdot m\cdot \Delta T \text{man kann hier noch}}\\ 
\end{eqnarray*}
Diese Proportionalität kann in eine Formel umgeschrieben werden:
\begin{eqnarray*}
	Q &=\phantom{c\cdot m\cdot \Delta T \text{man kann hier noch}} 
\end{eqnarray*}

Die \emph{spezifische Wärmekapazität} \gl{$c$} ist eine Materialkonstante. Sie beschreibt also eine Eigenschaft des Materials.
Sie gibt an, wie viel \gl{Wärme} nötig ist, um ein \gl{Kilogramm} des Materials um ein \gl{Kelvin} zu erwärmen.

\StoppLueckentext
\end{aufgabe}
