\documentclass{beamer}%    Standardklasse für Beamer
%\documentclass[draft]{beamer}%  Entwickler-Modus in dem das File schneller kompiliert wird
%\documentclass[handout]{beamer}%    Zum Ausdrucken besser geeignete Version der Folien
%\documentclass{article}%Artikel-Version; Funktioniert nur, wenn man zusätzlich das Paket beamerarticle mittels \usepackage{beamerarticle} einbindet.
%\usepackage{beamerarticle}
\usepackage[german]{babel}
\usepackage[T1]{fontenc}
\usepackage[utf8]{inputenc}

\usepackage{lmodern}
\usepackage{tikz}

%\usetheme{Boadilla}
\usetheme{Berlin}

%\newcommand{\IGRAPH}[1]{\includegraphics[width=1.0\textwidth] {#1}}
\newcommand{\IGRAPH}[5]{\includegraphics[trim = #2 #3 #4 #5, clip, width=\textwidth] {#1}}
% left, bottom, right and top

\setbeamercovered{transparent}
\beamertemplatenavigationsymbolsempty
\setbeamertemplate{footline}[frame number]

\title{Kugelbahn}

\author[Felix Binder]{Felix Binder}

\begin{document}

%\begin{frame}
%\titlepage	
%\end{frame}

\newcommand*\leadzeros[1]{\ifnum#1<10 00\else0\fi#1}

\section{}
\begin{frame}
	\frametitle{Kugelbahn}
	\begin{center}
	\begin{tikzpicture}
		
	\foreach \x in {1,...,45}{
	\draw (0,0) node {\includegraphics<\x>[width=0.99\textwidth]{./foo-\leadzeros{\x}.jpeg}};
%	\onslide<\x>{\draw (0,-2) node [minimum size=1cm,circle,fill=white]{\x};}}
	\only<\x>{\draw (0,-2) node [minimum size=1cm,circle,fill=white]{\x};}}
%	\includegraphics<\x>[width=0.95\textwidth]{./foo-\leadzeros{\x}.jpeg};
%	\includegraphics<2>[width=0.95\textwidth]{./foo-002.jpeg}
%	\includegraphics<3>[width=0.95\textwidth]{./foo-003.jpeg}
	\end{tikzpicture}
	\end{center}

	%	\draw (0,0) node {\IGRAPH{./foo-00\x.jpeg}{0cm}{0cm}{0cm}{0cm}};\pause
\end{frame}

\end{document}



