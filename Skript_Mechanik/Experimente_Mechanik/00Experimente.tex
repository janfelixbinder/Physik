%\documentclass[11pt,a4paper,titlepage,twoside]{article}
\documentclass[12pt,a4paper,twoside]{article}

%diese sind neu
\usepackage{style}
\usepackage{aufgaben}

\usepackage{gplot}
\usepackage{foto_v001}
\usepackage{mechanik_v001}

%\author{Felix Binder}
\title{Experimente Mechanik}
\date{}


\begin{document}
\maketitle

\tableofcontents

\section{Reibung}
\begin{aufgabe}
	
%Als Hausaufgabe:\\
Nehmen Sie einen schweren Gegenstand und schieben Sie ihn über eine ebene waagerechte Oberfläche z.B. einen Tisch oder den Boden.

\begin{enumerate}[a)]
	\item Schieben Sie den Gegenstand an, das heisst am Anfang ist der Gegenstand in Ruhe, dann bewegen Sie ihn.
	\item Schieben Sie den Gegenstand mit konstanter Geschwindigkeit. 
\end{enumerate}
Vergleichen Sie nun, die zwei Situationen. Brauchen Sie für beides gleich viel Kraft, 
oder ist in einer der beiden Situationen schwerer den Gegenstand zu schieben?

Um sicher zu gehen, wiederholen Sie das Experiment einige Male.

\end{aufgabe}

\newpage

Bestimmung der Haftreibungszahl und der Gleitreibungszahl und vielleicht der Rollreibungszahl.

Gruppenarbeit oder Paartnerarbeit

Nehmen Sie ein Buch oder einen Ordner mit festem Rücken. Ausserdem einen oder mehrere Gegenstände, die zu rutschen beginnen, wenn man sie auf den schrägen
Buchrücken legt.

Nun sollen die SuS die Reibungszahl für die Oberfläche mit dem Gegenstand bestimmen.


\end{document}
