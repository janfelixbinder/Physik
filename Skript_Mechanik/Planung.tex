%\documentclass[11pt,a4paper,titlepage,twoside]{article}
\documentclass[11pt,a4paper,twoside]{article}
%\documentclass[12pt,a4paper,twoside]{scrartcl}
\usepackage{mystyle}
\usepackage{gplot}
\usepackage{mechanik_v001}
\usepackage{foto_v001}


\usepackage[shell]{gnuplottex}
\usepackage{pgf}
\usepackage{pgfplots}
\usepackage{gnuplot-lua-tikz}
\pgfplotsset{compat=1.9}


\newcommand{\Oberthema}[1]{\newpage\section{#1}}
\newcommand{\Unterthema}[1]{\subsection{#1}}
\newcommand{\Ziele}[1]{\subsubsection*{Ziele: #1}}
\newcommand{\Teilziele}[1]{\subsubsubsection*{Teiliel: #1}}



%\author{Felix Binder}
\title{Jahresplanung Mechanik}
\date{}



\def\dir{./Aufgaben_Mechanik/}
\newcommand{\Einbinden}[1]{\input{#1}}


\begin{document}
\maketitle
\tableofcontents

\Oberthema{Kräfte}
	\Unterthema{Kräfte sind Vektoren}
		\Ziele{Sie können Kräfte graphisch und rechnerisch addieren.}
		\Ziele{Sie können Kräfte graphisch und rechnerisch zerlegen.}

	\Unterthema{Statik mit gemeinsamem Angriffspunkt}
		\Ziele{Sie können auftretende Kräfte in einen Kräfteplan einzeichnen.}
		\Ziele{Sie können erklären warum die resultierende Kraft bei Statikproblemen Null sein muss.}
		\Ziele{Sie können die resultierende Kraft graphisch und rechnerisch bestimmen.}

	\Unterthema{Schiefe Ebene}
		\Ziele{Die SuS können die Hangabtriebskraft erklären und in Beispielen berechnen.}

\Oberthema{Kinematik}
	\Unterthema{Geschwindigkeit}
		\Ziele{Sie können den Unterschied zwischen Durchschnittsgeschwindigkeit und Momentangeschwindigkeit.}

	\Unterthema{Beschleunigung}
		\Ziele{Sie kennen Situationen in denen beschleunigte Bewegungen vorkommen.}
		\Ziele{Sie können beschreiben, wie viel Weg ein Gegenstand während einer gleichmässig beschleunigten Bewegung zurücklegt.}
		\Ziele{Sie können beschreiben, wie sich die Geschwindigkeit während einer gleichmässig beschleunigten Bewegung ändert.}
	\Unterthema{Bewegungsdiagramme}
		\Ziele{Sie können Weg-Zeit-Diagramme, Geschwindigkeit-Zeit-Diagramme und Beschleunigung-Zeit-Diagramme lesen und erstellen.}

	\Unterthema{Kreisbewegungen}
		\Ziele{Sie können erklären, warum man für Kreisbewegungen neue Grössen einführt.}
		\Ziele{Sie können Frequenz, Umlaufzeit und Winkelgeschwindigkeit an Beispielen erklären.}
		\Ziele{Sie können Frequenz, Umlaufzeit und Winkelgeschwindigkeit in Aufgaben anwenden.}


	\Unterthema{Zusammengesetzte Bewegungen}
		\Ziele{Die SuS können erklären, was mit einem Ball passiert, der in die Luft geworfen wird.}


\Oberthema{Dynamik der Translationen}
	\Unterthema{Zweites Newtonsches Gesetz}
		\Ziele{Sie können das zweite Newtonsche Gesetz erklären und dieses in Aufgaben anwenden.}

	\Unterthema{Reibung}
		\Ziele{Sie kennen die Reibung und können zwischen Haft-, Gleit- und Rollreibung unterscheiden.}

	\Unterthema{Komplexe Aufgaben Dynamikaufgaben}
		\Ziele{Sie können Berechnungen an der schiefen Ebene durchführen.}
		\Ziele{Sie können Berechnungen an der schiefen Ebene durchführen.}
	
\Oberthema{Dynamik der Rotationen}
	\Unterthema{Hebelgesetz}
	\Ziele{Sie können das Hebelgesetz erklären und in Aufgaben anwenden.}
	
	\Unterthema{Das Drehmoment}
	\Ziele{Sie können erklären wie man ein Drehmoment zu einer bestimmten Drehachse bestimmt und diese dann berechnen.}
	
	\Unterthema{Schwerpunktbestimmung}
	\Ziele{Sie können den Flächenschwerpunkt von einfachen Flächen (Dreiecke, Rechtecke, Kreise) bestimmen.}
	\Ziele{Sie können den gemeinsamen Schwerpunkt von Teilschwerpunkten bestimmen.}
	
	\Unterthema{Statik mit Drehmomenten}
	\Ziele{Sie können das Drehmoment benutzen um Statikprobleme zu lösen.}

	\Unterthema{Kreisbewegungen}
	\Ziele{Sie können erklären warum sich ein Gegenstand auf einer Kreisbahn bewegt.}
	\Ziele{Sie kennen die Zentripetalbeschleunigung und die Zentripetalkraft und können Sie erklären.}
	\Ziele{Sie können Aufgaben lösen bei denen sich Gegenstände auf einer Kreisbahn bewegen.}

\Oberthema{Arbeit und Energie}
	\Unterthema{Einfache Maschinen}


\Oberthema{Kreisbewegungnen und Umlaufbahnen}
	\Unterthema{Zentripetalkraft}
	\Unterthema{Keplersche Gesetze}



\end{document}
