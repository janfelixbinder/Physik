\section{Grössen und Einheiten}
\subsection{Weg}

Eine sehr wichtige Grösse in der gesamten Physik ist der Weg. 
Um die Länge eines Weges zu bestimmen muss man ihn messen.
Messen bedeutet vergleichen mit einer Einheit.
Das Formelzeichen für den Weg ist $s$. 
Die Grundeinheit (SI-Einheit, von französisch Système international d’unités) des Weges ist der Meter.
Abgekürzt wird die Einheit mit \si{m}.
Die Einheit einer physikalischen Grösse schreibt man in eckigen Klammern, also $[s]=\si{m}$.

\Einbinden{\dir/einheiten01.tex}
\Einbinden{\dir/einheiten02.tex}
\Einbinden{\dir/einheiten03.tex}
\Einbinden{\dir/einheiten04.tex}
\Einbinden{\dir/einheiten_UErde.tex}



%\startkloesung
\subsection{Zeit}
Um die Zeit $t$ zu messen, orientiert sich die Menschheit schon seit Jahrtausenden an den Gestirnen.
Winter- und Sommersonnenwenden wurden schon in der Steinzeit gefeiert.
Das Messgerät zur Zeitmessung ist die Uhr.
Die SI-Einheit der Zeit ist die Sekunde (s).
Traditionell ist die Sekunde der \num{86400}-ste Teil $(24\cdot60\cdot60)$ eines Tages.
Seit 1967 wird die Sekunde über eine atomare Anregung definiert. Daher auch der Name Atomuhr.

\Einbinden{\dir/einheiten05.tex}

\subsection{Masse}
Eine weitere häufig gebrauchte Grösse ist die Masse $m$. Ihre SI-Einheit ist das Kilogramm (kg).
Anders als bei den anderen Einheiten, hat das Kilogramm noch keine moderne, ausschliesslich auf
Naturkonstanten basierende Definition. Das Urkilogramm besteht aus einer Platin-Iridium-Legierung und wird in Paris verwahrt.


\Einbinden{\dir/einheiten06.tex}
\Einbinden{\dir/einheiten07.tex}
\Einbinden{\dir/einheiten08.tex}

