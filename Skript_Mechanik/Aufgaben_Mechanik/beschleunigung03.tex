

\begin{aufgabe}
Ein Ball wird mit einer Geschwindigkeit von \SI{30}{m/s} nach oben geworfen, und fällt
durch die Fallbeschleunigung wieder zu Boden.
\begin{enumerate}[a)]
	\item Zeichnen Sie ein Beschleunigungs-Zeit- und in ein Geschwindigkeits-Zeit-Diagramm für diesen Wurf.
	\item Wie lange braucht der Ball bis zum höchsten Punkt?
	\item Wie hoch steigt der Ball insgesamt?
	\item Der Ball soll \SI{50}{m} hoch kommen. Mit welcher Geschwindigkeit $v_0$ muss er hochgeworfen werden?
	\item Zeichnen Sie ein Weg-Zeit-Diagramm und erklären Sie es.
\end{enumerate}

\kloesung{b) \SI{3.06}{s}, c) \SI{45.87}{m}, d) \SI{31.3}{m/s}}

\begin{loesung}
	\begin{enumerate}[a)]
		\item
		\item Am höchsten Punkt ist die Geschwindigkeit $v=0$. 
		Wir können also schreiben
		\begin{eqnarray*}
		0 = v_0 + a\cdot t \to t=-\frac{v_0}{a}=\frac{\SI{30}{m/s}}{{9.81}{m/s^2}}=\SI{3.06}{s}.
		\end{eqnarray*}

	\item Dies kann auf drei unterschiedlichen Wegen bestimmt werden. 
\begin{itemize}
	\item $s=v_0\cdot t +\frac{1}{2}\cdot a \cdot t^2 = \SI{30}{m/s}\cdot\SI{3.06}{s}+0.5\cdot(\SI{-9.81}{m/s^2})\cdot(\SI{3.06}{s})^2=\SI{91.743}{m}-\SI{45.872}{m}=\SI{45.872}{m}$
	\item $\bar{v}=\frac{1}{2}\cdot (v_0+v_1)$ $\to s=\bar{v}\cdot t=\SI{15}{m/s}\cdot\SI{3.06}{s}=\SI{45.872}{m}$
	\item Oder man bestimmt die Fläche im $v$-$t$-Diagramm.
\end{itemize}
	\item Der Ball braucht genauso viel Zeit um von unten nach oben aufzusteigen, wie um von oben nach unter herabzufallen.
		Wir berechnen nun die Zeit um 50 Meter zu fallen.
		\begin{eqnarray*}
			s=\frac{1}{2}\cdot a\cdot t^2 \to t=\sqrt{\frac{2\cdot s}{a}}=\sqrt{\frac{\SI{100}{m}}{\SI{9.81}{m/s^2}}}=\SI{3.2}{s}
		\end{eqnarray*}
		Mit
		\begin{eqnarray*}
			v=v_0 + a\cdot t\to v_0=-a\cdot t=\SI{9.81}{m/s^2}\cdot\SI{3.2}{s}=\SI{31.321}{m/s}\text{.}
		\end{eqnarray*}

		Alternativ kann man auch folgendes rechnen:

		\begin{align*}
		v^2=v_0^2 + 2\cdot a \cdot \Delta s \to v_0 &= \sqrt{v^2 - 2\cdot a \cdot \Delta s}=\sqrt{\SI{0}{m/s} - 2\cdot (\SI{-9.81}{m/s^2})\cdot\SI{50}{m}}=\\
			&=\sqrt{\SI{981}{m^2/s^2}}=\SI{31.321}{m/s}
		\end{align*}
\end{enumerate}


\end{loesung}
\end{aufgabe}
