\begin{aufgabe}
Beim Hammerwurf schwingen Sie eine Metallkugel an einem Drahtseil im Kreis.
Die Metallkugel wiegt \SI{7.26}{kg}.
Das Drahtseil ist \SI{2.10}{m} lang.

\begin{enumerate}[a)]
\item Wie hoch ist die Geschwindigkeit der Kugel wenn die Seilkraft \SI{2000}{N} beträgt?
\item Wie viele Umbrehungen machen Sie in der Sekunde bei dieser Geschwindigkeit?
\end{enumerate}



\kloesung{a) $v=\SI{24}{m/s}$, b)\SI{1.82}{Hz}}
\begin{loesung}
	\begin{enumerate} [a)]
		\item Die Geschwindigkeit der Kugel kann so bestimmt werden:
		\begin{eqnarray*}
			\RI{F}{Res}=\frac{m\cdot v^2}{r}\to v=\sqrt{\frac{\RI{F}{Res}\cdot r}{m}}=\SI{24}{m/s}
		\end{eqnarray*}
		\item Damit ergeben sich \num{1.82} Umdrehungen pro Sekunde.
		\begin{eqnarray*}
			v=\frac{\Delta s}{\Delta t}=\frac{U\cdot n}{\Delta t}\to n=\frac{v\cdot \Delta t}{U}=\SI{1.82}{1/s}=\SI{1.82}{Hz}
		\end{eqnarray*}
	\end{enumerate}
%
%
\end{loesung}



\end{aufgabe}
