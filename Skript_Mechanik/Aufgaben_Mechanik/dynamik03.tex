
%Kreisbewegungen
\begin{aufgabe}
	Die Internationale Raumstation (ISS) umkreist die Erde in einer Höhe von etwa \SI{380}{km}.
	Die Masse der Erde ist \SI{5.9722E24}{kg}, der Radius der Erde ist \SI{6.37E6}{m}.
	\begin{enumerate} [a)]
\item Warum kann die ISS ohne Antrieb um die Erde kreisen?
\item Machen Sie eine Skizze, in der Sie alle Kräfte eintragen.
\item Wie gross ist die Erdanziehung in dieser Höhe im Vergleich zum Erdboden?
\item Mit welcher Geschwindigkeit fliegt die ISS?
	\end{enumerate}
\end{aufgabe}

