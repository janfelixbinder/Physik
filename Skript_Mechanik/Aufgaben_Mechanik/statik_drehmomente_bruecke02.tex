%kurzversion der vorigen aufgabe
\begin{aufgabe}
	Eine Person (\SI{75}{kg}) geht über einen zehn Meter langen Stahlträger. Der Stahlträger liegt vorne und hinten auf.
Berechnen Sie die Kraft, die auf den ersten Aufliegepunkt wirkt, während die Person über die Brücke geht.
Stellen Sie ihr Ergebnis graphisch dar.
\begin{enumerate} [a)]
	\item Nehmen Sie zuerst an, der Stahlträger sei masselos.
	\item Berücksichtigen Sie die Masse des Stahlträgers von \SI{150}{kg}.
\end{enumerate}
\end{aufgabe}
\begin{loesung}

\end{loesung}
