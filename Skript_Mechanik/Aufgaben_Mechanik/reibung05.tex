
\begin{aufgabe}
	Auf einen ruhenden Holzblock der Masse \SI{500}{g}, der auf einer Tischplatte aus Holz liegt, greift eine horizontale Zugkraft von \SI{2}{N} an.
	Bewegt sich der Körper? Begründen Sie ihre Antwort. Die Haftreibungszahl ist \num{0.6}.

	\kloesung{Der Klotz bewegt sich nicht. $\RI{F}{R}=\SI{2.94}{N}$.}

	\begin{loesung}
		Der Körper bewegt sich dann, wenn die horizontal angreifende Zugkraft grösser als die maximale Reibungskraft ist.
		Die maximale Reibungskraft ergibt sich aus
		\begin{eqnarray*}
			\RI{F}{R}=\mu\cdot\RI{F}{N}=\num{0.6}\cdot\SI{0.5}{kg}\cdot\SI{9.81}{m/s^2}=\SI{2.94}{N}.
		\end{eqnarray*}
		Da die maximale Reibunskraft grösser als die Zugkraft ist, bewegt sich der Holzblock nicht.
	\end{loesung}


\end{aufgabe}
