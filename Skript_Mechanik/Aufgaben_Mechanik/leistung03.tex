
\begin{aufgabe}
	\begin{table}
		\centering
		%\begin{tabular}{<+dimensions+>}
	\begin{tabular}{c|c|c|c|c}
	Lithium-Ionen-Akku & starke Sprengstoffe & Schokolade     & Benzin   & Plutoniumbatterie \\
	\num{0.5}          &  \num{7}            &\num{23}      & \num{43} & \num{11200}\\ 
		\end{tabular}
		\caption{Energiedichte verschiedener Energieträger in MJ/kg. Plutoniumbatterien werden fast ausschliesslich in der Raumfahrt verwendet.}
		\label{tab:energiedichte}
	\end{table}
	Der Lithium-Ionen-Akku eines Smartphones hält bei normaler Nutzung etwa einen Tag. 
	Seine Kapazität beträgt \SI{7.98}{Wh} bei einem Gewicht (mit Schale) von \SI{38}{g}.
	\begin{enumerate} [a)]
		\item Wie gross ist die Energiedichte ($w=\frac{\Delta E}{\Delta m}$) des Akkus? Vergleichen Sie mit dem Tabellenwerte.
			\TAX{Berechnung mit numerischem Resultat K1}
		\item Wie lange würde das Telefon mit einem anderen Energiespeichermedium halten, wenn es sich für die Nutzung eignen würde?
			\TAX{Berechnung mit numerischem Resultat K2}
	\end{enumerate}
	%\begin{loesung}
	%Die Kapazität des Akkus muss zuerst in SI-Einheiten umgerechnet werden. Dabei gilt:
	%\begin{eqnarray*}
	%	\SI{1}{Wh}=\SI{3600}{Ws}=\SI{3600}{J}
	%\end{eqnarray*}
	%\begin{eqnarray*}
	%	w=\frac{\Delta E}{\Delta m}=\frac{\SI{28728}{J}}{\SI{0.038}{kg}}=\SI{756000}{J/kg}=\SI{756}{kJ/kg}
	%\end{eqnarray*}
	%\end{loesung}
\end{aufgabe}

