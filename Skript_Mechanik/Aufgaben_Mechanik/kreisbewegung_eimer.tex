
\begin{aufgabe}
	
Haben Sie schon einmal einen mit Wasser gefüllten Eimer vertikal über den Kopf schwingen lassen?
Schwingt man den Eimer schnell genug, dann wird man dabei nicht nass.
\begin{enumerate} [a)]
	\item Machen Sie eine Skizze der Situation in der der Eimer an der untersten bzw.~ obersten Position ist
und zeichnen Sie die wirkenden Kräfte ein.
\item Mit welcher Geschwindigkeit müssen Sie den Eimer mindestens schwingen, damit kein Wasser raus läuft? 
	Nehmen Sie einen Radius von einem Meter an.
	\kloesung{v=\SI{3.2}{m/s}}
\end{enumerate}
\end{aufgabe}
\begin{loesung}

\end{loesung}
