
\begin{aufgabe}
Ein Zug beschleunigt auf einer Hochgeschwindigkeitsstrecke aus dem Stand
mit einer Beschleunigung von $\SI{2}{m/s^2}$.\\
\begin{itemize}
	\item[a)] Welche Geschwindigkeit hat er nach \SI{1}{min} erreicht?
	\item[b)] Welche Strecke hat er bis dahin zurückgelegt?
\end{itemize}

\kloesung{a) \SI{120}{m/s}, b) \SI{3600}{m}}

\begin{loesung}
\begin{itemize}
	\item[a)] \[v = v_0 + a \cdot \Delta t = \SI{2}{m/s^2}\cdot \SI{60}{s} = \SI{120}{m/s}\]
	\item[b)] \[\Delta s = v_0\cdot\Delta t + \frac{1}{2} \cdot a \cdot (\Delta t)^2 = 0.5 \cdot \SI{2}{m/s^2} \cdot (\SI{60}{s})^2 = \SI{3600}{m}\]
\end{itemize}
\end{loesung}
\end{aufgabe}
