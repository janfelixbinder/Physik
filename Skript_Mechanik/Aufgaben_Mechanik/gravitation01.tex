
\begin{aufgabe}
	Wie gross ist die Anziehungskraft zwischen einer ein Kilogramm schweren Kugel aus Blei, und einer ein Kilogramm schweren Kugel aus Holz, deren
	Massenmittelpunkte einen Meter von einander entfernt sind.
	\begin{loesung}
		\begin{eqnarray*}
			\RI{F}{Gra} = G \cdot \frac{m_1 \cdot m_2}{r^2} = \SI{6.674E-11}{Nm^2/kg^2}\cdot\frac{\SI{1}{kg}\cdot\SI{1}{kg}}{(\SI{1}{m})^2}=\SI{6.674E-11}{N}
		\end{eqnarray*}
	\end{loesung}
\end{aufgabe}
