
\begin{aufgabe}
	Auf einer Achse stecken zwei Scheiben mit unterschiedlichen Durchmessern.
	An der ersten Scheibe mit 50 Zentimeter Durchmesser ist ein Schnur aufgewickelt.
	Am Ende der Schnur hängt eine Masse von einem Kilogramm frei zu Boden.

	\begin{enumerate} [a)]
		\item Wie gross ist das Drehmoment an der Achse, das von der herunter hängenden Masse erzeugt wird.
		\item An der zweiten Scheibe (\SI{10}{cm} Durchmesser) ist ebenfalls ein Faden befestigt an dessen Ende ein zweites Gewicht hängt.
			Sinkt das erste Gewicht, steigt das zweite an. Wie schwer darf das zweite Gewicht maximal sein, damit es noch angehoben wird?
	\end{enumerate}

	\kloesung{a) \SI{2.45}{Nm}, b) \SI{5}{kg}}
\end{aufgabe}
