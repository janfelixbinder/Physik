
\begin{aufgabe}
	Um vom Erdgeschoss des Lyceums zum Physikunterricht in den zweiten Stock zu kommen, muss man
	56 Treppenstufen von etwa \SI{16.5}{cm} Höhe nehmen.
	\begin{enumerate} [a)]
		\item Wie viel Energie benötigen Sie mindestens, um vom Erdgeschoss in den Physikraum zu gelangen?
		\item Beeilt man sich, kann man in 12 Sekunden oben sein. Wie viel müssten Sie dafür leisten?
		\item Wie leistungsfähig können Sie beim Treppensteigen sein?
	\end{enumerate}
	\kloesung{Die Werte hängen von Ihrem Gewicht ab. Annahme Sie wiegen \SI{60}{kg}. a) $E=\SI{5438.7}{J}$, b) $P=\SI{453.2}{W}$}
	\begin{loesung}
		\begin{enumerate} [a)]
			\item Um die Energie berechnen zu können, benötigen Sie die zurückgelegte Höhe: 
		\begin{eqnarray*}
			h=\SI{0.165}{m}\cdot 56 =\SI{9.24}{m}
		\end{eqnarray*}
		Mit Ihrem Körpergewicht (z.B. \SI{60}{kg}) kommen Sie auf die potentielle Energie:
		\begin{eqnarray*}
			\RI{E}{pot}=m\cdot g\cdot h=\SI{60}{kg}\cdot\SI{9.81}{m/s^2}\cdot\SI{9.24}{m}=\SI{5438.7}{J}
		\end{eqnarray*}
	\item Leistung ist verrichtete Arbeit pro Zeit. Damit kommen wir auf:
		\begin{eqnarray*}
			P=\frac{\Delta W}{\Delta t}=\frac{\SI{5438.7}{J}}{\SI{12}{s}}=\SI{453.2}{W}
		\end{eqnarray*}
		\end{enumerate}
	\end{loesung}
\end{aufgabe}
