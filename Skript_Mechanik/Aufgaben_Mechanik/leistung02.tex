
\begin{aufgabe}
	Mit einem Lastenaufzug sollen \SI{50}{kg} Steine in 20 Sekunden zehn Meter hochbefördert werden.
	Für welche Leistung muss der Motor ausgelegt sein?
	\begin{loesung}
		Die Kraft ist die Gewichtskraft
		\begin{eqnarray*}
			\RI{F}{G}=m\cdot g=\SI{50}{kg}\cdot\SI{9.81}{m/s^2}=\SI{490.5}{N}
		\end{eqnarray*}
		Eine Hubarbeit von
		\begin{eqnarray*}
			\RI{W}{Hub}=\RI{F}{G}\cdot h=\SI{490.5}{N}\cdot\SI{10}{m}=\SI{4905}{J}
		\end{eqnarray*}
muss der Motor bewältigen.
Damit ergibt sich eine minimale Leistung von
\begin{eqnarray*}
	P=\frac{\RI{W}{Hub}}{\Delta t}=\frac{\SI{4905}{J}}{\SI{20}{s}}=\SI{245.25}{W}
\end{eqnarray*}
	\end{loesung}
\end{aufgabe}

