
\begin{aufgabe}
	Die ursprüngliche Definition des Meters definiert ihn als das \num{4E-7}-fache des Erdumfangs am Äquator.
	\begin{itemize}
		\item [a)] Welchen Umfang hat die Erde am Äquator (nach dieser Definition)?
		\item [b)] Wäre die Erde eine ideale Kugel, wie gross wäre ihr Radius?
		\item [c)] Welches Volumen hätte die Erde?
	\end{itemize}
	\begin{loesung}
		\begin{enumerate} [a)]
			\item Der Umfang am Äquator wäre \SI{4E7}{m}. Das sind \SI{4E4}{km}, also \SI{40000}{km}.
			\item
				\begin{eqnarray*}
					U=2\cdot\pi\cdot r \to r=\frac{U}{2\cdot\pi}=\frac{\SI{40000}{km}}{2\cdot\pi}=\SI{6366.2}{km}
				\end{eqnarray*}
			\item
				\begin{eqnarray*}
					V=\frac{4}{3}\cdot\pi\cdot r^3=\SI{1.08e+12}{km^3}=\SI{1.08e+21}{m^3}
				\end{eqnarray*}
		\end{enumerate}
	\end{loesung}
	\kloesung{a) \SI{40000}{km}, b) \SI{6366.2}{km}, c) \SI{1.08e+21}{m^3}}
\end{aufgabe}
