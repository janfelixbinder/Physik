
\begin{aufgabe}
Eine Masse von \SI{3}{kg} wird durch eine konstante Kraft in fünf Sekunden um zehn Meter
verrückt. Wie gross ist die Kraft? 

\begin{loesung}
	Eine konstante Kraft bedeutet eine konstante Beschleunigung. Es gilt 
	\[s=\frac{1}{2}\cdot a\cdot t^2 \to a=\frac{2s}{t^2}=\frac{2\cdot\SI{10}{m}}{\SI{25}{s^2}}=\SI{0.8}{m/s^2}\]
	\[F=m\cdot a=\SI{3}{kg}\cdot\SI{0.8}{m/s^2}=\SI{2.4}{N}\]

\end{loesung}
\kloesung{$F=\SI{2.4}{N}$}
\end{aufgabe}
