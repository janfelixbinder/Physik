
\begin{aufgabe}
	Ein Auto mit der Masse von \SI{1500}{kg} beschleunigt nach der roten Ampel auf \SI{50}{km/h}.
	Welche Arbeit muss der Motor dafür leisten?

	\kloesung{\SI{144.68}{kJ}}
	
	\begin{loesung}
		Zuerst sollten alle Grössen in den Grundeinheiten vorliegen.
		\begin{eqnarray*}
			v=\SI{50}{km/h}=\SI{13.89}{m/s}
		\end{eqnarray*}
		\begin{eqnarray*}
			W=\frac{1}{2}\cdot m\cdot v^2=\num{0.5}\cdot\SI{1500}{kg}\cdot(\SI{13.89}{m/s})^2=\SI{144.68}{kJ}
		\end{eqnarray*}
	\end{loesung}
\end{aufgabe}
