\begin{aufgabe}
	Ein Stein (\SI{1.5}{kg}) fällt von einer \SI{40}{m} hohen Brücke.
	\begin{itemize}
		\item [a)] Wie hoch ist die potentielle Energie des Steins auf der Brücke?
		\item [b)] Wie hoch ist die kinetische Energie des Steins auf der Brücke?
		\item [c)] Wie gross ist seine Gesamtenergie?
		\item [d)] Mit welcher Geschwindigkeit kommt der Stein unten auf?
	\end{itemize}

	\kloesung{a) \SI{588.6}{J}, b) \SI{0}{J}, c) \SI{588.6}{J}, d) \SI{28.0}{m/s}}

	\begin{loesung}
		\begin{itemize}
			\item[a)]
				\begin{eqnarray*}
					\RI{E}{pot}=m\cdot g\cdot h=\SI{1.5}{kg}\cdot\SI{9.81}{m/s^2}\cdot\SI{40}{m}=\SI{588.6}{J}
				\end{eqnarray*}
			\item[b)]
				\begin{eqnarray*}
					\RI{E}{kin}=\frac{1}{2}\cdot m\cdot v^2=\SI{0.5}\cdot\SI{1.5}{kg}\cdot{0}{m/s}=\SI{0}{J}
				\end{eqnarray*}
			\item[c)] Die Gesamtenergie ist konstant.
				\begin{eqnarray*}
					\RI{E}{tot}=\RI{E}{pot} + \RI{E}{kin}=\SI{588.6}{J} + \SI{0}{J}=\SI{588.6}{J}
				\end{eqnarray*}
			\item[d)] Am Boden ist die Gesamtenergie genauso gross wie auf der Brücke (Energieerhaltung).
				\begin{gather*}
					\RI{E}{pot}=m\cdot g\cdot h=\SI{1.5}{kg}\cdot\SI{9.81}{m/s^2}\cdot\SI{0}{m}=\SI{0}{J}\\
					\RI{E}{tot}=\RI{E}{pot} + \RI{E}{kin} \to \RI{E}{kin}=\RI{E}{tot}-\RI{E}{pot}=\SI{588.6}{J}-\SI{0}{J}=\SI{588.6}{J}\\
					\RI{E}{kin}=\frac{1}{2}\cdot m\cdot v^2 \to v=\sqrt{\frac{2\cdot \RI{E}{kin}}{m}} =\SI{28.0}{m/s}
				\end{gather*}
		\end{itemize}
	\end{loesung}
\end{aufgabe}
