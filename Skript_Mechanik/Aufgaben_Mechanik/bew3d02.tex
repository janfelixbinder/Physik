
\begin{aufgabe}
Ein Wasserstrahl tritt mit einer konstanten Geschwindigkeit von
\SI{30}{m/s} senkrecht nach oben aus.
\begin{itemize}
	\item[a)] Wie lange braucht ein Wassertropfen ganz nach oben?
	\item[b)] Wie hoch kommt der Wassertropfen dabei?
	\item[c)] Welche Anfangsgeschwindigkeit $v_0$ braucht der Wasserstrahl damit die
		Fontäne \SI{100}{m} hoch ist?
	\item[d)] Wie lange dauert der Aufstieg dann?
	\item[e)] Ein konstanter Seitenwind von \SI{1}{m/s} wirkt auf die Fontäne.
		Wie gross ist der Ablenkungswinkel?
	\item[f)] Wie weit weg vom Ursprung der Fontäne kommen die Wassertropfen 
		auf der Wasseroberfläche an?
	\item[g)] Wie stark muss der Wind wehen, damit die Wassertropfen nach \SI{100}{m}
		wieder aufkommen?
	\item[h)] Wie hoch müsste die Fontäne sein, damit die gleiche Auslenkung (\SI{100}{m}) wie in g)
		mit einem Seitenwind von \SI{1}{m/s} erreicht wird?
\end{itemize}
\begin{loesung}
	\begin{itemize}
		\item[a)]
	\[v=v_0-g\cdot t = 0 \to v_0=g\cdot t \to t=\frac{v_0}{g}=\frac{\SI{30}{m/s}}{\SI{10}{m/s^2}}=\SI{3}{s}\]
		
\item[b)]
	Entweder mit der Formel $s=v_0\cdot t + \frac{1}{2}\cdot a\cdot t^2 = \SI{30}{m}\cdot\SI{3}{s}-\num{0.5}\cdot\SI{10}{m/s^2}\cdot (\SI{3}{s})^2 = \SI{90}{m}-\SI{45}{m}=\SI{45}{m}$\\
	oder man berechnet die Fläche im $v$-$t$-Diagramm $s=\frac{1}{2}\cdot v_0\cdot t=\SI{15}{m/s}\cdot\SI{3}{s}=\SI{45}{m}$.

\item[c)]
	Durch ausprobieren bekommen wir einen Wert zwischen \SI{40}{m/s} und \SI{50}{m/s}.\\
	Mit $v^2=v_0^2 - 2\cdot a \cdot \Delta s = \SI{0}{m^2/s^2} \to v_0^2 = 2\cdot a\cdot \delta s \to v_0=\sqrt{2\cdot g\cdot \Delta s}=\sqrt{2\cdot\SI{10}{m/s^2}\cdot\SI{100}{m}}=\SI{44.72}{m/s}$

\item[d)]
	Wie in Teil a) $\to t=\SI{4.5}{s}$.

\item[e)]
	$s_x=v_x\cdot t = \SI{1}{m/s}\cdot\SI{4.5}{s}=\SI{4.5}{m}$\\
	$\tan{\alpha} = \frac{\SI{4.5}{m}}{\SI{100}{m}}=\num{0.045} \to \alpha=\num{2.6}\degree$

\item[f)]
	$s_x=v_x\cdot t=\SI{1}{m/s}\cdot2\cdot\SI{4.5}{s}= \SI{9}{m}$

\item[g)]
	$v_x=\frac{\Delta s}{\Delta t} =\frac{\SI{100}{m}}{\SI{9}{s}}=\SI{11.1}{m/s}$

\item[h)]
	\[t=\frac{\Delta s}{v_x} = \frac{\SI{100}{m}}{\SI{1}{m/s}}=\SI{100}{s} \to \SI{50}{s}\]
	um den höchsten Punkt zu erreichen.\\
	Jetzt wie in a) $v_0=g\cdot t=\SI{10}{m/s^2}\cdot\SI{50}{s}=\SI{500}{s}$.\\
	Wie in b) $\Delta s= \frac{1}{2}\cdot v_0\cdot t=\frac{1}{2}\cdot\SI{500}{m/s}\cdot\SI{50}{s}=\SI{25000}{m}$.


\end{itemize}
\end{loesung}

\end{aufgabe}


