
%Die Aufgabe in kleinere Schritte zerlegen.
\begin{aufgabe}
	Eine Person (\SI{75}{kg}) geht über einen zehn Meter langen Stahlträger. Der Stahlträger liegt vorne und hinten auf.
 Nehmen Sie den Stahlträger fürs erste als masselos an.
\begin{enumerate} [a)]
\item Skizzieren Sie die Situation.
\item Zeichnen Sie einen Kräfteplan.
\item Welche Bedingung muss für die Kräfte gelten?
\item Was gilt für die Drehmomente?
\item Wählen Sie einen der Auflagepunkte als Drehachse und berechnen Sie die Kraft auf den ersten Auflagepunkt wenn der Mensch direkt darauf steht.
\item Wie gross ist die Kraft auf den ersten Auflagepunkt, wenn der Mensch in der Mitte der Brücke steht?
\item Stellen Sie die Kraft auf den ersten Auflagepunkt für beliebige Positionen des Menschen graphisch dar.
	\item Berücksichtigen Sie die Masse des Stahlträgers von \SI{150}{kg}. Wie verläuft die Kraft nun in Abhängigkeit von der Position der Person?
\end{enumerate}
\end{aufgabe}
\begin{loesung}

\end{loesung}

