
\begin{aufgabe}
	Zwei Federn mit unterschiedlichen Federkonstanten werden hintereinander gehängt.
	Die eine Feder hat eine Federkonstante von \SI{100}{N/m}, die andere Feder hat eine Federkonstante von \SI{50}{N/m}.
	Die Kraft, die auf die Federn wirkt sei \SI{10}{N}. Wie gross ist die gesamte Auslenkung? Wie gross ist die Federkonstante?
	\begin{loesung}
		\begin{eqnarray*}
			F=D\cdot l\to l=\frac{F}{D}
		\end{eqnarray*}
Die Kraft ist für beide Federn gleich gross.
\begin{eqnarray*}
	l=l_1+l_2=\frac{F}{D_1}+\frac{F}{D_2}=\frac{\SI{10}{N}}{\SI{100}{N/m}}+\frac{\SI{10}{N}}{\SI{50}{N/m}}=\SI{0.1}{m}+\SI{0.2}{m}=\SI{0.3}{m}\text{.}
\end{eqnarray*}
Um die Federkonstante zu bestimmen stellen wir das Federgesetz nach ihr um
		\begin{eqnarray*}
			F=D\cdot l\to D=\frac{F}{l}=\frac{\SI{10}{N}}{\SI{0.3}{m}}=\SI{33.3}{N/m}\text{.}
		\end{eqnarray*}
	\end{loesung}
\end{aufgabe}

