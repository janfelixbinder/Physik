%\documentclass[11pt,a4paper,titlepage,twoside]{article}
\documentclass[12pt,a4paper,twoside]{article}

%diese sind neu
\usepackage{style}
\usepackage{aufgaben}

\usepackage{gplot}
\usepackage{foto_v001}
\usepackage{mechanik_v001}

%\author{Felix Binder}
\title{Aufgabensammlung Mechanik}
\date{}


\begin{document}
\maketitle

\tableofcontents

\section{Einheiten}
\Einbinden{\dir/einheiten01.tex}
\Einbinden{\dir/einheiten02.tex}
\Einbinden{\dir/einheiten03.tex}
\Einbinden{\dir/einheiten04.tex}
\Einbinden{\dir/einheiten05.tex}
\Einbinden{\dir/einheiten06.tex}
\Einbinden{\dir/einheiten07.tex}
\Einbinden{\dir/einheiten08.tex}

\section{Dichte}
\Einbinden{\dir/dichte01.tex}
\Einbinden{\dir/dichte02.tex}
\Einbinden{\dir/dichte03.tex}
\Einbinden{\dir/dichte04.tex}


\section{Geschwindigkeit}
\Einbinden{\dir/geschwindigkeit00a.tex}
\Einbinden{\dir/geschwindigkeit00b.tex}
\Einbinden{\dir/geschwindigkeit01.tex}
\Einbinden{\dir/geschwindigkeit02.tex}
\Einbinden{\dir/geschwindigkeit03.tex}
\Einbinden{\dir/geschwindigkeit04.tex}
\Einbinden{\dir/geschwindigkeit05.tex}

\section{Beschleunigung}
\Einbinden{\dir/geschwindigkeit06.tex}
\Einbinden{\dir/geschwindigkeit07.tex}
\Einbinden{\dir/beschleunigung01.tex}
\Einbinden{\dir/beschleunigung02.tex}
\Einbinden{\dir/kinematik01.tex}

\subsection{Aufgaben mit beiden Formeln}

\begin{aufgabe}
Sie schieben ein Spielzeugauto über den Tisch.
Nachdem es die Hand verlassen hat, hat es eine Geschwindigkeit von \SI{3}{m/s}.
Nach \SI{2.5}{m} bleibt das Auto stehen.

\begin{enumerate}[a)]
	\item Wie gross ist die Beschleunigung?
	\item Nach welcher Zeit kommt der Wagen zum stehen?
\end{enumerate}

\begin{loesung}
   Gegeben: $v_0=\SI{3}{m/s}$, $\Delta s=\SI{2.5}{m}$.
   \begin{enumerate}[a)]
	   \item 
   Für die beschleunigte Bewegung haben wir zwei Formeln. Beide enthalten die Zeit $t$.
   Wir formen $v=\nicefrac{v}{t}$ nach $t$ um und setzten in $s=\nicefrac{1}{2}\cdot a\cdot t^2$ ein. 
        \begin{eqnarray*}
			s=\frac{1}{2}\frac{v^2}{a} \to a=\frac{1}{2}\frac{v^2}{s} =\num{0.5}\cdot\frac{\SI{3}{m/s}^2}{\SI{2.5}{m}}=\SI{1.8}{m/s^2}
		\end{eqnarray*}
		Die Geschwindigkeit nimmt ab. Die Beschleunigung muss also negativ sein. $a=\SI{-1.8}{m/s^2}$.
	\item Das Ergebnis von a) in $v=a\cdot t$ einsetzen.
		\begin{eqnarray*}
			v=a\cdot t\to t=\frac{v}{a}=\frac{\SI{3}{m/s}}{\SI{1.8}{m/s^2}}=\SI{1.667}{s}
		\end{eqnarray*}
   \end{enumerate}
\end{loesung}

\kloesung{a) $=a=\SI{-1.8}{m/s^2}$, b) \SI{1.667}{s}}

\end{aufgabe}



\section{Kreisbewegung}
Die Aufgaben \ref{rasenmaeher}, \ref{flugzeugturbine} und \ref{elektromotor} sind alle vom gleichen Typ.
Die Schwierigkeit besteht darin die Frequenz der Kreisbewegung als gegebene Grösse zu erkennen.
Ist dieses Hindernis überwunden muss die Winkelgeschwindigkeit berechnet werden und im Anschluss die Bahngeschwindigkeit.

\Einbinden{\dir/kreis01.tex}
\Einbinden{\dir/kreis02.tex}
\Einbinden{\dir/kreis03.tex}

\section{Bewegung in drei Raumrichtungen}

\Einbinden{\dir/beschleunigung03.tex}
\Einbinden{\dir/kinematik02.tex}


\Einbinden{\dir/bew3d01.tex}
\Einbinden{\dir/bew3d02.tex}

\section{Gravitation}
\Einbinden{\dir/gravitation01.tex}
\Einbinden{\dir/gravitation02.tex}

\section{Federgesetz}
\Einbinden{\dir/federgesetz01.tex}
\Einbinden{\dir/federgesetz02.tex}
\Einbinden{\dir/federgesetz03.tex}

\section{Vektoren}
\Einbinden{\dir/vektoren01.tex}
\Einbinden{\dir/vektoren02.tex}
\Einbinden{\dir/vektoren03.tex}
\Einbinden{\dir/vektoren04.tex}
\Einbinden{\dir/vektoren05.tex}
\Einbinden{\dir/vektoren06.tex}
\Einbinden{\dir/vektoren07.tex}
\Einbinden{\dir/vektoren08.tex}
\Einbinden{\dir/vektoren09.tex}
\Einbinden{\dir/vektoren10.tex}
\Einbinden{\dir/vektoren11.tex}
\Einbinden{\dir/statik02.tex}

\section{Drehmomente}
\Einbinden{\dir/drehmomente01.tex}
\Einbinden{\dir/drehmomente02.tex}
\Einbinden{\dir/drehmomente03.tex}
\Einbinden{\dir/drehmomente04.tex}
\Einbinden{\dir/drehmomente05.tex}
\Einbinden{\dir/statik01.tex}
\Einbinden{\dir/drehmomente06.tex}
\Einbinden{\dir/drehmomente07.tex}
\Einbinden{\dir/drehmomente_ski.tex}
\Einbinden{\dir/waage.tex}

\section{Schwerpunkt}
\Einbinden{\dir/schwerpunkt01.tex}
\Einbinden{\dir/schwerpunkt02.tex}
\Einbinden{\dir/schwerpunkt03.tex}
\Einbinden{\dir/schwerpunkt04.tex}
\Einbinden{\dir/statik03.tex}
\Einbinden{\dir/flaechen_schwerpunkt_buchstaben_T.tex}
\Einbinden{\dir/flaechen_schwerpunkt_buchstaben_U.tex}
\Einbinden{\dir/flaechen_schwerpunkt_buchstaben_I.tex}



\section{Lösen von Statikproblemen mit Hilfe des Drehmomentes}

\Einbinden{\dir/loesen01.tex}
\Einbinden{\dir/loesen02.tex}

\Einbinden{\dir/statik_drehmomente_bruecke01.tex}
\Einbinden{\dir/statik_drehmomente_bruecke02.tex}

\section{Reibung}
\Einbinden{\dir/reibung01.tex}
\Einbinden{\dir/reibung02.tex}
\Einbinden{\dir/reibung03.tex}
\Einbinden{\dir/reibung04.tex}
\Einbinden{\dir/reibung05.tex}
\Einbinden{\dir/reibung06.tex}
\Einbinden{\dir/reibung07.tex}

\section{Schiefe Ebene}
\Einbinden{\dir/schiefeEbene01.tex}
\Einbinden{\dir/dynamik05.tex}


\section{Newton'sche Axiome}
\Einbinden{\dir/dynamik01.tex}
\Einbinden{\dir/dynamik02.tex}
\Einbinden{\dir/dynamik04.tex}
\Einbinden{\dir/newton01.tex}
\Einbinden{\dir/newton02.tex}
\Einbinden{\dir/newton03.tex}

\section{Kreisbewegungen}
\Einbinden{\dir/kreisbewegung_erklaerung.tex}
\Einbinden{\dir/kreisbewegung01.tex}
\Einbinden{\dir/kreisbewegung_eimer.tex}
\Einbinden{\dir/kreisbewegung_reibung.tex}
\Einbinden{\dir/kreisbewegung_kanone.tex}
\Einbinden{\dir/dynamik03.tex}
\Einbinden{\dir/hammerwurf.tex}
\Einbinden{\dir/kreisbewegung_schaukel.tex}

\section{Wechselwirkungsgesetz}
\Einbinden{\dir/wechselwirkung01.tex}
\Einbinden{\dir/wechselwirkung02.tex}

\section{Impuls}
\Einbinden{\dir/impuls01.tex}


\section{Arbeit und Energie}
\Einbinden{\dir/hubarbeit01.tex}
\Einbinden{\dir/beschleunigungsarbeit01.tex}

\Einbinden{\dir/arbeit_schrank.tex}
\Einbinden{\dir/arbeit_schlitten.tex}
\Einbinden{\dir/arbeit_hub.tex}
\Einbinden{\dir/Ekin_auto.tex}
\Einbinden{\dir/arbeit_feder01.tex}
\Einbinden{\dir/arbeit_feder02.tex}
\Einbinden{\dir/arbeit_feder03.tex}



\section{Energieerhaltung}

\subsection{Pendel}
\Einbinden{\dir/energieerhaltung_pendel.tex}
\Einbinden{\dir/energieerhaltung_bruecke.tex}
\Einbinden{\dir/energieerhaltung_feder.tex}
\Einbinden{\dir/energieerhaltung_wagen_feder.tex}



\Einbinden{./arbeit01.tex}
\Einbinden{./arbeit02.tex}
\Einbinden{./arbeit03.tex}
\Einbinden{./arbeit04.tex}
\Einbinden{./energieerhaltung_kugelbahn.tex}

\Einbinden{\dir/stmichael_energie_jane.tex}
\Einbinden{\dir/stmichael_energie_ski.tex}
\Einbinden{\dir/stmichael_energie_schlitten.tex}
\Einbinden{\dir/stmichael_energie_bungee.tex}

\section{Leistung}
\Einbinden{\dir/leistung01.tex}
\Einbinden{\dir/leistung02.tex}
\Einbinden{\dir/leistung03.tex}
\Einbinden{\dir/leistung04.tex}
\Einbinden{\dir/leistung05a.tex}
\Einbinden{./arbeit05.tex}
\Einbinden{\dir/leistung_velo.tex}

\section{Wirkungsgrad}
\Einbinden{\dir/leistung05.tex}


\end{document}
