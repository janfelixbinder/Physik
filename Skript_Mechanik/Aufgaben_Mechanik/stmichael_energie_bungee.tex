\begin{aufgabe}

Eine Bungeespringerin mit einer Masse von \SI{60}{kg} springt von einer Brücke.
Sie ist an einem Bungeeseil befestigt, das im ungedehnten Zustand \SI{12}{m} lang ist,
und fällt insgesamt \SI{31}{m}.

\begin{enumerate} [a)]
\item Berechnen Sie die Federkonstante $D$ des Bungeeseils und nehmen Sie dabei an, dass das Hooke'sche Gesetz gilt.
\item Berechnen Sie die von der Springerin erfahrene maximale Beschleunigung.


\end{enumerate} 

\kloesung{a) \SI{101}{N/m}, b) \SI{22.2}{m/s^2}}

\end{aufgabe}
