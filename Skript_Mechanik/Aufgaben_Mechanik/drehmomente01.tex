
\begin{aufgabe}
	Was beobachten Sie an diesen Wippen?

\begin{center}
\begin{tikzpicture}
	\begin{scope} [xshift=-5cm]
	\Wippe{0}{0}{0}
	\end{scope}
	\begin{scope}[xshift=5cm]
	\Wippe{39}{0}{0}
	\end{scope}
\end{tikzpicture}
\end{center}

	Warum verändert sich die Stellung der Wippe?

	\begin{center}
	\begin{tikzpicture}
		\Wippe{-29}{2}{0}
		\draw (Pr) --+(-90:1cm) node [shape=coordinate] (E) {};
		\Masse{(E)}{0.5}{M1}
		\draw (M1 O) node [right] {$m_1$};
	\end{tikzpicture}
\end{center}
\newpage

Was sieht man hier?	

\begin{center}
	\begin{tikzpicture}
		\Wippe{0}{2.5}{-2.5}
		\draw (Pr) --+(-90:1cm) node [shape=coordinate] (Er) {};
		\draw (Pl) --+(-90:1cm) node [shape=coordinate] (El) {};
		\Masse{(Er)}{0.5}{Mr}
		\Masse{(El)}{0.5}{Ml}
		\draw (Mr O) node [right] {$m_1$};
		\draw (Ml W) node [left] {$m_1$};
	\end{tikzpicture}
\end{center}

Was hat sich hier verändert?	

\begin{center}
	\begin{tikzpicture}
		\Wippe{20}{2.5}{-4.0}
		\draw (Pr) --+(-90:1cm) node [shape=coordinate] (Er) {};
		\draw (Pl) --+(-90:1cm) node [shape=coordinate] (El) {};
		\Masse{(Er)}{0.5}{Mr}
		\Masse{(El)}{0.5}{Ml}
		\draw (Mr O) node [right] {$m_1$};
		\draw (Ml W) node [left] {$m_1$};
	\end{tikzpicture}
\end{center}

Was hat sich hier verändert?	

\begin{center}
	\begin{tikzpicture}
		\Wippe{0}{2.5}{-4.0}
		\draw (Pr) --+(-90:1cm) node [shape=coordinate] (Er) {};
		\draw (Pl) --+(-90:1cm) node [shape=coordinate] (El) {};
		\Masse{(Er)}{0.5}{Mr}
		\Masse{(El)}{0.5}{Ml}
		\draw (Mr O) node [right] {$m_1$};
		\draw (Ml W) node [left] {$m_2$};
	\end{tikzpicture}
\end{center}
\end{aufgabe}

\begin{aufgabe}
	\begin{enumerate}[a)]
		\item Finden Sie mindestens drei unterschiedliche Kombinationen aus Länge des Hebelarms und Gewichten, in denen die Wippe im Gleichgewicht ist.
		\item Bestimmen Sie das Verhältnis der Hebelarme und das Verhältnis der Massen für diese Gleichgewichtsstellungen. Was fällt Ihnen auf?
		\item Formulieren Sie Ihr Hebelgesetz.
	\end{enumerate}
	
\end{aufgabe}

\begin{aufgabe}
	Betrachen Sie die obige Abbildung. Wenn die Masse $m_1$ \SI{200}{g} gross ist, wie gross ist dann die Masse $m_2$?
\end{aufgabe}

