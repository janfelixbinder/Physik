
\begin{aufgabe}
	\label{SchachtelBeschleunigungKraft}
Eine Schachtel mit einer Masse von \SI{100}{g} wird über einen Tisch geschoben.
Nachdem sie die Hand verlassen hat, hat sie eine Geschwindigkeit von \SI{3}{m/s}.
Nach \SI{1.25}{m} bleibt die Schachtel liegen.
\begin{itemize}
\item[a)] Wie gross ist die Beschleunigung?
\item[b)] Wie gross ist die Bremskraft?
\end{itemize}
	
	\begin{loesung}
		Gegeben: $m=\SI{0.1}{kg}$, $v_0=\SI{3}{m/s}$, $\Delta s=\SI{1.25}{m}$. 
		\begin{itemize}
\item[a)] 
	\begin{eqnarray*}
		v^2=v_0^2 + 2\cdot a\cdot s \to a=\frac{v^2-v_0^2}{2\cdot s}=\frac{\SI{0}{m/s}-\SI{9}{m/s}}{2\cdot\SI{1.25}{m}}=\SI{-3.6}{m/s^2}
	\end{eqnarray*}
\item[b)]  
	\begin{eqnarray*}
		F=m\cdot a=\SI{0.1}{kg}\cdot(-\SI{3.6}{m/s^2})=\SI{-0.36}{N}
	\end{eqnarray*}
\end{itemize}
	\end{loesung}
	\kloesung{a) $=a=\SI{-3.6}{m/s^2}$, b) $F=\SI{-0.36}{N}$}
\end{aufgabe}

