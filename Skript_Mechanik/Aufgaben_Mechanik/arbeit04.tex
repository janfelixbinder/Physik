
\begin{aufgabe}
	Um die Geschwindigkeit einer Gewehrkugel zu bestimmen, schiesst ein Balistiker mit einem Gewehr auf einen grossen Gummiblock.
	Der Gummiblock ist an einem Haken aufgehängten.
	Die Kugel wird vom Gummiblock aufgenommen und abgebremst, dadurch wir dieser ausgelenkt.
	\begin{enumerate} [a)]
		\item Skizzieren Sie den Versuch.
		\item Welches Prinzip können Sie nutzen zum die Geschwindigkeit der Kugel zu berechnen?
		\item Die Gewehrkugel hat eine Masse von \SI{50}{g}. Der \SI{500}{kg} schwere Block wird insgesamt um \SI{40}{cm} angehoben.
			Wie gross ist die Geschwindigkeit der Kugel?
	\end{enumerate}
\end{aufgabe}


