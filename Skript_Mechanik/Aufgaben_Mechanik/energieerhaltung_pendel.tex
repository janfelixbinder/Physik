
\begin{aufgabe}
	\label{pendel_energieerhaltung}
Begründen und Argumentieren Sie Ihre Antworten schriftlich.
	\begin{enumerate}[a)]
		\item	Was müssen Sie tun, um ein Fadenpendel aus seiner Ruhelage auszulenken?
		\item Wie viel müssen Sie für diese Auslenkung arbeiten?
		\item Wie viel potentielle Energie hat das Pendel durch diese Auslenkung erhalten?

		\item Sie lassen das Pendel nun schwingen. Geben Sie die Geschwindigkeit für einige ausgewählte Positionen des Pendels an.
		\item Was passiert mit der potentiellen Energie während der Schwingung?

	\end{enumerate}
\end{aufgabe}

\begin{aufgabe}
	Berechnen Sie Zahlenwerte für die Fragestellungen aus Aufgabe \ref{pendel_energieerhaltung}.
	Dazu lenken Sie den \SI{1}{m} langen Faden des Pendels um \SI{30}{\degree} aus.
	Die Masse des Pendels ist \SI{0.1}{kg}.

	\kloesung{b) $\RI{W}{Hub} =\SI{0.13}{J}$, c) $\RI{E}{pot}=\SI{0.13}{J}$, d) am tiefsten Punkt $v=\SI{1.62}{m/s}$}

	\begin{loesung}
		\begin{enumerate}[a)]
			\item Sie müssen Hubarbeit an der Masse des Fadenpendels verrichten. 
				Der genaue Weg, den die Masse zum Auslenkpunkt nimmt ist nicht entscheidend für die zu verrichtende Arbeit.
				Sie müssen nur wissen, um welche Höhe Sie die Masse anheben.
			\item Um die Höhe zu bestimmen um die Sie das Pendel auslenken, können Sie eine massstabsgetreue Zeichnung anfertigen und abmessen,
				oder Sie berechnen die Höhe mit
				\begin{eqnarray*}
					h=\SI{1}{m} - \cos(\SI{30}{\degree}) = \SI{0.134}{m}\text{.}
				\end{eqnarray*}

				Die zu verrichtende Hubarbeit ist damit
				\begin{eqnarray*}
					\RI{W}{Hub} = m\cdot g\cdot h =\SI{0.1}{kg}\cdot\SI{9.81}{m/s^2}\cdot\SI{0.134}{m} =\SI{0.13}{J}\text{.}
				\end{eqnarray*}
			\item Das Pendel hat nun eine zusätzliche potentielle Energie von
				\begin{eqnarray*}
					\RI{E}{pot} = m\cdot g\cdot h =\SI{0.1}{kg}\cdot\SI{9.81}{m/s^2}\cdot\SI{0.134}{m} =\SI{0.13}{J}\text{.}
				\end{eqnarray*}
				Das ist genau die Menge Energie, die vorher in Form von Hubarbeit erbracht wurde.
			\item Während der Schwingung des Pendels wird potentielle Energie in kinetische Energie und umgekehrt umgewandelt.
				Durch den Energieerhaltungssatz wissen Sie, dass die Summe aus potentieller und kinetischer Energie gleich
				bleibt. Damit können Sie die Geschwindigkeit für jeden beliebigen Punkt des Pendels berechnen.
				Zum Beispiel ist an den Wendepunkten die Geschwindigkeit Null. Damit ist die kinetische Energie Null. Die Gesamtenergie ist
				\begin{eqnarray*}
					E =\RI{E}{pot} + \RI{E}{kin} = m\cdot g\cdot h + \frac{1}{2} m\cdot v^2 = \SI{0.13}{J}\text{.} 
				\end{eqnarray*}

				Am tiefsten Punkt ist die potentielle Energie Null. Damit muss die kinetische Energie gleich der Gesamtenergie sein.
				\begin{eqnarray*}
					E =\RI{E}{pot} + \RI{E}{kin} \to \RI{E}{kin} = E - \RI{E}{pot}= \SI{0.13}{J}\text{.} 
				\end{eqnarray*}
				Damit können wir jetzt die Geschwindigkeit berechnen.
				\begin{eqnarray*}
					\RI{E}{kin} =\frac{1}{2} m\cdot v^2 \to v^2 =\frac{2 \cdot\RI{E}{kin}}{m}=\frac{2\cdot\SI{0.13}{J}}{\SI{0.1}{kg}} = \SI{2.6}{m^2/s^2}
				\end{eqnarray*}
				Die Geschwindigkeit erhalten Sie nun, indem Sie die Wurzel ziehen $v=\SI{1.61}{m/s}$.


		\end{enumerate}
	\end{loesung}

\end{aufgabe}
