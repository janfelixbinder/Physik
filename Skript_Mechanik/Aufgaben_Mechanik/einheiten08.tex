

\begin{aufgabe}
	Eine Idee das Kilogramm neu zu definieren, besteht im Abzählen von Atomen. Jedes Atom hat eine bestimmte Masse.
	Gesucht ist nun die richtige Anzahl eines bestimmten Atomtyps.
	Wie viele Atome $^{28}_{14}\text{Si}$ (Silizium, mit atomarer Masse 28) benötigt man, für ein Kilogramm?
	\begin{loesung}
		1 Mol dieses Silizium Isotops wiegt \SI{28}{g}. Also benötigt man $\frac{\SI{1000}{g}}{\SI{28}{g/mol}}=\num{35.714}$ mol dieses Isotops.
		Das sind $\SI{35.714}{mol}\cdot\SI{6.02E23}{mol^{-1}}=\num{2.1500e25}$ Atome.
	\end{loesung}
	\kloesung{\num{2.1500e25}}
\end{aufgabe}
