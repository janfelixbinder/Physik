
\begin{aufgabe}
	Eine Masse von \SI{5}{kg} wird um \SI{3}{m} angehoben.
	\begin{itemize}
		\item [a)] Berechnen Sie die erforderliche Hubarbeit?
		\item [b)] Um wie viel hat sich die potentielle Energie der Masse vergrössert?
	\end{itemize}
	\kloesung{a) \SI{147.15}{J}, b) \SI{147.15}{J}}
	\begin{loesung}
		\begin{itemize}
			\item [a)] 
				\begin{eqnarray*}
					\RI{W}{Hub} = \RI{F}{G}\cdot s=m\cdot g\cdot s= \SI{5}{kg}\cdot\SI{9.81}{m/s^2}\SI{3}{m}=\SI{147.15}{J}
				\end{eqnarray*}
			\item [b)] 
				\begin{eqnarray*}
					\RI{E}{pot} = m\cdot g\cdot h= \SI{5}{kg}\cdot\SI{9.81}{m/s^2}\SI{3}{m}=\SI{147.15}{J}
				\end{eqnarray*}
		\end{itemize}
	\end{loesung}
\end{aufgabe}
