
\begin{aufgabe}
	Ein Holzklotz mit dem Gewicht von \SI{300}{Gramm} liegt auf einer schiefen Ebene.
	Bei einer Steigung von \SI{35}{\degree} beginnt der Block zu rutschen.
	\begin{enumerate}[a)]
		\item Machen Sie sich eine Skizze der Situation und vervollständigen sie diese mit den auftretenden Kräften.
		\item Bestimmen Sie die Normalkraft und die Reibungskraft.
		\item Wie gross ist die Haftreibungszahl?
	\end{enumerate}
	\begin{loesung}
		\begin{enumerate} [a)]
			\item Skizze wie über der Aufgabe. 
			\item Um Normalkraft und Reibungskraft zu erhalten, muss die Gewichtskraft in eine Komponente senkrecht zur Ebene (Normalkraft)
				und eine parallel zur Ebene (Reibungskraft) zerlegt werden.
				\begin{eqnarray*}
					\RI{F}{N}=\cos\alpha\RI{F}{G}
				\end{eqnarray*}
				und
				\begin{eqnarray*}
					\RI{F}{R}=\sin\alpha\RI{F}{G}\text{.}
				\end{eqnarray*}
			\item Mit der Formel für die Reibungskraft ergibt sich die Haftreibungszahl:
				\begin{eqnarray*}
					\RI{F}{R}=\mu\cdot\RI{F}{N}\to\mu=\frac{\RI{F}{R}}{\RI{F}{N}}=\frac{\sin\alpha}{\cos\alpha}=\num{0.7}\text{.}
				\end{eqnarray*}
		\end{enumerate}
	\end{loesung}<++>
	\kloesung{c) $\mu=\num{0.7}$}
\end{aufgabe}

