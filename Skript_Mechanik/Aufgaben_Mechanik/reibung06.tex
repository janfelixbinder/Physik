
\begin{aufgabe}
	Ein Schlitten hat eine Masse von \SI{75}{kg}.
	Berechnen Sie
	\begin{itemize}
		\item [a)] die Haftreibung.
		\item [b)] die Gleitreibung.
	\end{itemize}
	\begin{loesung}
		Die Normalkraft ist gleich der Gewichtskraft. $F_N=m\cdot g=\SI{75}{kg}\cdot\SI{9.81}{m/s^2}=\SI{735.75}{N}$.
		Die Reibungszahlen entnimmt man einer Tabelle.
		\begin{itemize}
			\item[a)] Haftreibung Stahl auf Eis $\mu= 0.027$ aus Tabelle.%~\ref{tab:reibung}.
				\begin{eqnarray*}
					F_R=\mu\cdot F_N= \num{0.027}\cdot\SI{735.75}{N}=\SI{19.865}{N}
				\end{eqnarray*}
			\item[b)] Gleitreibung: Stahl auf Eis $\mu= 0.014$ aus Tabelle.%~\ref{tab:reibung}.
				\begin{eqnarray*}
					F_R=\mu\cdot F_N=\num{0.014}\cdot\SI{735.75}{N}=\SI{10.30}{N}
				\end{eqnarray*}
		\end{itemize}
	\end{loesung}
\end{aufgabe}
