
\begin{aufgabe}
	Zeigen die folgenden Abbildungen immer eine Gleichgewichtslage des Körpers? Wenn ja, ist das Gleichgewicht stabil, labil oder indifferent?
	Wenn nicht, gibt es ein Drehmoment. Berechnen Sie dieses. Die Gewichtskraft der Dreiecke soll in jedem Fall \SI{10}{N} betragen.
	\vspace*{2cm}

	\begin{center}
	\begin{tikzpicture}
		\coordinate (P1) at (0,0);
		\coordinate (P2) at (3,2);
		\coordinate (P3) at (4,-2);

		\fill [color=black!30] (P1)--(P2)--(P3)--(P1);
		\SPDreieck{(P1)}{(P2)}{(P3)}{Drei}
		\draw [fill] (Drei SP) circle (0.1cm) node [right] {SP};
		\draw (Drei SP) ++(90:1) node [shape=coordinate] (Drei D){};
		\draw [fill] (Drei D) circle (0.1cm) node [right] {D};

		\coordinate (P1) at (6,-1);
		\coordinate (P2) at (9,2);
		\coordinate (P3) at (14,-2);

		\fill [color=black!30] (P1)--(P2)--(P3)--(P1);
		\SPDreieck{(P1)}{(P2)}{(P3)}{Drei}
		\draw [fill] (Drei SP) circle (0.1cm) node [right] {SP};
		\draw (Drei SP) ++(120:1.4) node [shape=coordinate] (Drei D){};
		\draw [fill] (Drei D) circle (0.1cm) node [right] {D};

	\end{tikzpicture}
	\end{center}

	\vspace*{2cm}

	\begin{center}
	\begin{tikzpicture}
		\coordinate (P1) at (0,0);
		\coordinate (P2) at (5,2);
		\coordinate (P3) at (3,-5);

		\fill [color=black!30] (P1)--(P2)--(P3)--(P1);
		\SPDreieck{(P1)}{(P2)}{(P3)}{Drei}
		\draw [fill] (Drei SP) circle (0.1cm) node [right] {SP};
		\draw (Drei SP) ++(180:1.3) node [shape=coordinate] (Drei D){};
		\draw [fill] (Drei D) circle (0.1cm) node [right] {D};

		\coordinate (P1) at (8,2);
		\coordinate (P2) at (15,0);
		\coordinate (P3) at (11.5,-5);

		\fill [color=black!30] (P1)--(P2)--(P3)--(P1);
		\SPDreieck{(P1)}{(P2)}{(P3)}{Drei}
		\draw [fill] (Drei SP) circle (0.1cm) node [right] {SP};
		\draw (Drei SP) ++(270:2.2) node [shape=coordinate] (Drei D){};
		\draw [fill] (Drei D) circle (0.1cm) node [right] {D};

	\end{tikzpicture}
	\end{center}

\end{aufgabe}

