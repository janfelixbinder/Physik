\begin{aufgabe}
	Ein Auto mit einer Masse von \SI{1.5}{T} überträgt über die Pneus eine Kraft auf die Strasse.
	\SI{60}{\percent} der Wagenmasse liegen auf der Antriebswelle (und damit auch auf den Antriebsrädern, der Wagen hat kein Allrad).
	\begin{enumerate} [a)]
		\item Was passiert, wenn diese Kraft überschritten wird?
		\item Wie viel Kraft kann maximal auf die Strasse übertragen werden?
	\end{enumerate}
	\begin{loesung}
		Die maximale Kraft, die auf die Strasse übertragen werden kann ist begrenzt durch die Reibung.
		Die Reibung ist proportional zur Normalkraft. In der Formelsammlung findet man die Proportionalitätskonstante für dieses Problem.
		Die Haftreibungszahl ist
	\end{loesung}
\end{aufgabe}


