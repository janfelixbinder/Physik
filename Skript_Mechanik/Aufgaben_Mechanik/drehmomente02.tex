
\begin{center}
	\Bildeinbinden{\dir/nussknacker2b.jpg}{0.7}
\end{center}

\begin{aufgabe}
	Auf dem Foto ist ein Nussknacker zu sehen. %Seine Länge ist \SI{15}{Zentimeter}.
	\begin{enumerate} [a)]
		\item Finden Sie die Drehachse des Nussknackers sowie die zwei Hebelarme im Foto.
		\item Beschreiben Sie warum man den Nussknacker zum Knacken von Nüssen verwendet.
		\item Das Foto zeigt den Nussknacker nicht massstabsgetreu. 
			Die Länge des Nussknackers ist etwa 15 Zentimeter. Warum ist diese Angabe für die Berechnung der Kraftverstärkung nicht wichtig?
		\item Zum Knacken einer Wallnuss wird eine Kraft von etwa \SI{1000}{N} benötigt. 
			Wie viel Kraft brauchen Sie mit diesem Nussknacker um die Nuss zu knacken?
	\end{enumerate}
\end{aufgabe}
