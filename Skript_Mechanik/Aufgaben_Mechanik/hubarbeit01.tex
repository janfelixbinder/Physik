
\begin{aufgabe}
	Eine Masse von \SI{5}{kg} wird um \SI{3}{m} angehoben.
	Berechnen Sie die erforderliche Hubarbeit?

	\kloesung{\SI{147.15}{J}}

	\begin{loesung}
		\begin{eqnarray*}
			\RI{W}{Hub} = \RI{F}{G}\cdot s=m\cdot g\cdot s= \SI{5}{kg}\cdot\SI{9.81}{m/s^2}\SI{3}{m}=\SI{147.15}{J}
		\end{eqnarray*}
	\end{loesung}
\end{aufgabe}
