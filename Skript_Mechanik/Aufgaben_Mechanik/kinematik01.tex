
\begin{aufgabe}
	Ein Auto fährt geradlinig gleichförmig mit einer Geschwindigkeit von \SI{120}{km/h} auf der Autobahn.  
\begin{enumerate}[a)]
	\item Wie weit kommt es in drei Sekunden?
	\item Vor einem Tunnel bremst der Fahrer das Auto in zwei Sekunden auf \SI{100}{km/h} ab.
		Wie gross ist die Beschleunigung?
	\item Der Tunnel ist \SI{140}{m} lang. Wie lange braucht das Auto durch den Tunnel?
	\item Zeichnen Sie ein $v$-$t$-Diagramm und ein $a$-$t$-Diagramm der Aufgabe.
\end{enumerate}

\kloesung{a) \SI{100}{m}, b) \SI{-2.8}{m/s^2}, c) \SI{5}{s}}

\begin{loesung}
	\begin{enumerate} [a)]
		\item Das Auto kommt \SI{100}{m} weit.
	\begin{eqnarray*}
		v=\frac{\Delta s}{\Delta t} \to \Delta s =v\cdot\Delta s = \SI{33.3}{m/s}\cdot\SI{3}{s}=\SI{100}{m}
	\end{eqnarray*}
\item Die Beschleunigung ist \SI{-2.8}{m/s^2}.
	\begin{eqnarray*}
		a=\frac{\Delta v}{\Delta t} =\frac{\SI{-5.5}{m/s}}{\SI{2}{s}}=\SI{-2.8}{m/s} 
	\end{eqnarray*}
\item Das Auto braucht 5 Sekunden durch den Tunnel.
	\begin{eqnarray*}
		v=\frac{\Delta s}{\Delta t} \to \Delta t =\frac{\Delta s}{v} =\frac{\SI{140}{m}}{\SI{27.8}{m/s}} = \SI{5}{s}
	\end{eqnarray*}
	\end{enumerate}
\end{loesung}

\end{aufgabe}

