%Die letzte Aufgabe bedient sich eines historischen Experiments von Galileo Galilei. Für diese Aufgabe soll die Energieerhaltung
%angewandt werden. Da die Berechnung das ursprüngliche Experiments für die SuS zu schwierig ist, habe ich die Aufgabe vereinfacht.
%Die Schwierigkeit dieser Aufgabe lässt sich sehr gut variieren. Ausserdem könnte man das lösen der Aufgabe mit dem Ausführen eines
%Experiments verknüpfen.

\begin{aufgabe}
	Galileo Galilei (*\,1564 in Pisa, $\dagger$\,1641 in Arcetri bei Florenz) untersuchte das Fallen von Körpern 
	und fand dabei als erster das Fallgesetz.
	Zu seiner Zeit gab es noch keine Uhren, die Bruchteile von Sekunden messen konnten, daher war er darauf angewiesen,
	dass Fallen stark zu verlangsamen um den Zusammenhang von Weg und Zeit beim Fallen von Körpern trotzdem messen zu können.
	Um dies zu erreichen, benutzte er eine schiefe Ebene um die Beschleunigung zu verringern.
	Für die schiefe Ebene hängt die Beschleunigung vom Steigungswinkel ab. 
	Bei kleinen Steigungen, ist die Beschleunigung klein,
	im Grenzfall eines Winkels von \SI{90}{\degree} erhält man die Fallbeschleunigung.
	Benutzten Sie die Energieerhaltung (dieses praktische Konzept kannte Galilei noch nicht),
	um die Beschleunigung in Abhängigkeit vom Steigungswinkel zu bestimmen.
	\begin{enumerate} [a)]
		\item Machen Sie eine Skizze des Versuchsaufbaus.
			\TAX{Zeichnerisch K1}
		\item Finden Sie eine Formel für die potentielle Energie, die vom Steigungswinkel abhängt.
			\TAX{Formale Herleitung K1}
		\item Berechnen Sie für verschiedene Steigungswinkel (\SI{15}{\degree}, \SI{45}, {\degree}, \SI{75}{\degree} und \SI{90}{\degree}) 
			die Endgeschwindigkeit eines Schlittens, der \SI{1.5}{m} auf einer schiefen Ebene gleitet.
			\TAX{Berechnung mit numerischem Resultat K2}
		\item Wie stark wurde der Schlitten für die oben genannten Winkel beschleunigt? Tragen Sie die Werte in ein Diagramm ein. 
			Stellen Sie eine allgemeine	Formel für die Beschleunigung in Abhängigkeit vom Steigungswinkel auf.
			\TAX{Berechnung mit numerischem Resultat K2}
	\end{enumerate}
	Galilei nutzt anstatt eines Schlittens eine Kugeln, die er die schiefe Ebene runter rollen liess. Er ignorierte dabei, dass die Kugel rollend
	die schiefe Ebene herunterkommt. Dadurch kam er auf einen falschen Wert für die Fallbeschleunigung. Benutzt man eine Kugel, ist die kinetische Energie
	$\RI{E}{kin}=\nicefrac{7}{10}\cdot m\cdot v^2$.
	\begin{enumerate} [e)]
		\item Auf welchen Wert für die Fallbeschleunigung ist Galilei mit einer Messingkugel gekommen?
			\TAX{Berechnung mit numerischem Resultat K2}
	\end{enumerate}
	\begin{loesung}
		\begin{enumerate} [a)]
			\item Eine Skizze könnte so aussehen:
				\begin{center}
					
				\begin{tikzpicture}

\draw (-1,0)--(10,0);
\draw (0,0) --(30:10) node [midway, above] {$l$};

\draw [->] (2,0) arc(0:30:2);
\draw (1.5,0.4) node {$\alpha$};

\draw [<->] (8,0) --(8,4.619) node [midway,right] {$h=l\cdot\sin\alpha$};

\draw [fill] (8,4.7) -- (8,4.9) -- ++(30:0.5)--++(0,-0.2) node [right] {Schlitten};

\end{tikzpicture}
				\end{center}
			\item Für die potentielle Energie gilt:
				\begin{eqnarray*}
					\RI{E}{pot}=m\cdot g\cdot h = m\cdot g\cdot l\cdot \sin \alpha\text{.}
				\end{eqnarray*}
			\item Es gilt die Energieerhaltung. Das heisst, die Gesamtenergie bleibt konstant, während der Schlitten die Ebene herunterrutscht.
				Die potentielle Energie, die der Schlitten oben mehr hat, wandelt sich beim runterrutschen in kinetische Energie um. Oben war die
				kinetische Energie Null, damit ist sie unten gleich der potentiellen Energie oben.
				\begin{eqnarray*}
					\begin{split}
						\RI{E}{pot}&=\RI{E}{kin}\\	
					m\cdot g\cdot l\cdot\sin\alpha&=\frac{1}{2}\cdot m\cdot v^2\to v=\sqrt{2\cdot g\cdot l\cdot\sin\alpha}\text{.}
					\end{split}
				\end{eqnarray*}
				Damit erhält man für \SI{15}{\degree} eine Geschwindigkeit von \SI{2.76}{m/s}, für
\SI{45}{\degree} eine Geschwindigkeit von \SI{4.56}{m/s}, für
\SI{75}{\degree} eine Geschwindigkeit von \SI{5.33}{m/s} und für 
\SI{90}{\degree} eine Geschwindigkeit von \SI{5.42}{m/s}.
\item Für diesen Aufgabenteil kann eine Formel aus der Kinematik verwendet werden:
	\begin{eqnarray*}
		v^2=v_0^2+2\cdot a\cdot\Delta s\to a=\frac{v^2-v_0^2}{2\cdot\Delta s}\text{.}
	\end{eqnarray*}
	Damit bekommen wir Werte für die Beschleunigung. Bei einem Winkel von \SI{15}{\degree} ist die Beschleunigung \SI{2.54}{m/s^2},
	bei einem Winkel von \SI{45}{\degree} ist die Beschleunigung \SI{6.94}{m/s^2},
	bei einem Winkel von \SI{75}{\degree} ist die Beschleunigung \SI{9.48}{m/s^2} und
	bei einem Winkel von \SI{90}{\degree} ist die Beschleunigung \SI{9.81}{m/s^2}.

	\begin{tikzpicture}[xscale=0.1,yscale=0.5]
		\Xachse{0}{90}{15}{Winkel $\alpha$}
		\Yachse{0}{10}{2}{a (\SI{}{m/s^2})}
		\SPoint{0,0}
		\SPoint{15,2.54}
		\SPoint{30,4.91}
		\SPoint{45,6.94}
		\SPoint{60,8.50}
		\SPoint{75,9.48}
		\SPoint{90,9.81}
	\end{tikzpicture}

	Setzt man die Formel für die Geschindigkeit (haben wir in Aufgabenteil c) erhalten) in die obige Formel ein, bekommen wir eine allgemeine
	Formel für die Beschleunigung an der schiefen Ebene
	\begin{eqnarray*}
		a=\sin\alpha\cdot g\text{.}
	\end{eqnarray*}

	\item Wiederholt man die Rechnung, und setzt dabei die kinetische Energie einer Kugel ein, so erhält man 
		die folgend winkelabhängige Beschleunigung
		\begin{eqnarray*}
			a=\frac{5}{7}\cdot g\cdot\sin\alpha\text{.}
		\end{eqnarray*}
		Galilei muss also etwa \SI{7}{m/s^2} für die Fallbeschleunigung erhalten haben.
		\end{enumerate}
	\end{loesung}

\end{aufgabe}

%	Galilei nutzte eine Art Kugelbahn, deren Steigung er verändern konnte. Bei kleiner Steigung,
%	rollen die Kugeln langsamer die Bahn herunter, erhöht man die Steigung ist die Beschleunigung
%	grösser. Um die Fallbeschleunigung zu erhalten, untersuchte er, wie die Beschleunigung vom
%	Steigungswinkel abhängt. Dabei vernachlässigte er, dass beim Fallen (\SI{90}{\degree}) keine
%	Rollbewegung mehr stattfindet. Dadurch ist die von ihm ermittelte Fallbeschleunigung kleiner als die
%	richtige. Benutzen Sie die Energieerhaltung um Galileis Fallbeschleunigung zu ermitteln.
%		\item Wie setzt sich die kinetische Energie in diesem Problem zusammen?
%		\item Wie hängen Winkelgeschwindigkeit und Bahngeschwindigkeit bei einer Kugel zusammen?
%		\item Stellen Sie eine Formel für die kinetische Energie auf.

%\section{Formeln}
%			Die Formel für den Rollwiderstand ist:
%			\begin{eqnarray*}
%				\RI{F}{Roll}=c\cdot m\cdot g
%			\end{eqnarray*}
%
%			\begin{eqnarray*}
%				\RI{F}{Luft}=\num{0.5}\cdot c_w \cdot A \cdot \rho\cdot v^2
%			\end{eqnarray*}
%

