%\documentclass[11pt,a4paper,titlepage,twoside]{article}
\documentclass[12pt,a4paper,twoside]{article}
%\documentclass[12pt,a4paper,twoside]{scrartcl}
\usepackage{mystyle}

%\author{Felix Binder}
\title{Ziele für den Mechaniktest}
\date{}



\def\dir{../Aufgaben_Mechanik/}
\newcommand{\Einbinden}[1]{\input{#1}}


\begin{document}
\maketitle

\section*{Kräfte}

\begin{enumerate}
	\item Sie können Kräfte in Skizzen einzeichnen.
	\item Sie können Kräfte graphisch und rechnerisch addieren.
	\item Sie können Kräfte graphisch und rechnerisch zerlegen.
	\item Sie kennen die Reibung und können zwischen Haft-, Gleit- und Rollreibung unterscheiden.
	\item Sie können das zweite Newtonsche Gesetz erklären und dieses in Aufgaben anwenden.
	\item Sie können Berechnungen an der schiefen Ebene durchführen.

\end{enumerate}

Der Test wird fünf Aufgaben zu unterschiedlichen Themen umfassen. Jede Aufgabe gibt gleich viele Punkte.
Um eine 4 zu erreichen müssen Sie also drei Aufgaben richtig bearbeiten.

\end{document}
