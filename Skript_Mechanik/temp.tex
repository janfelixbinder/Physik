%\documentclass[11pt,a4paper,titlepage,twoside]{article}
%\documentclass[12pt,a5paper,landscape]{article}
\documentclass[12pt,a5paper]{article}
%\documentclass[12pt,a4paper,twoside]{scrartcl}
\usepackage{mystyle}

%\author{Felix Binder}
\title{Mechanik}
\date{}



\def\dir{../Aufgaben_Mechanik/}
\newcommand{\Einbinden}[1]{\input{#1}}

\addtocounter{page}{4}
\addtocounter{section}{9}
\addtocounter{aufgabe}{48}

\begin{document}

\begin{tikzpicture}
	
\def\Radius{4.5cm}
\def\dx{2cm}

\usetikzlibrary{calc,intersections,through,backgrounds}

\draw [dotted] (0,\Radius+1)--(0,-\Radius-1);
\draw [dotted] (\Radius+1, 0)--(-\Radius-1,0);

\draw [name path = Kreis] (0,0) circle(\Radius);
%\draw (0,0) --++(0,\Radius) node [midway, left] {$R$};

\draw [->,force] (0,\Radius) --++(\dx,0) node [midway, above] {$x$} node [shape=coordinate] (A){};
\path [name path = Radius] (A) --++ (0,-\Radius);

\path [name intersections={of=Kreis and Radius, by={G}}];

\draw [->,force] (A) -- (G) node [midway,right]{$r-y$};
\draw (0,0) -- (G) node [midway, above] {$r$};

%\draw(0,0)--(0,-\Radius)--(G)--(0,\Radius)--(0,0);

%\draw (0,0) -- (G) node [midway, above] {$r$};

%\draw (G)--++(-\Radius,0);

\draw [fill] (0,0) circle(0.1);

\end{tikzpicture}


An einem Kreis gilt
\begin{eqnarray*}
	r=\sqrt{x^2 + y^2}\text{.}
\end{eqnarray*}

Dies kann man auch schreiben als

\begin{eqnarray*}
	r^2 = x^2 + y^2 \qquad \text{oder} \qquad y^2=r^2-x^2\text{.}
\end{eqnarray*}

Nun soll ein Gegenstand vom Punkt $P_0$ in $x$-Richtung bewegt werden.
Der Gegenstand hat die Geschwindigkeit $v$.
Wir bewegen den Gegenstand nur ein sehr kleines Stück $\Delta x$.
\begin{eqnarray*}
	\Delta x = v\cdot \Delta t
\end{eqnarray*}

In der Zeit $\Delta t$ fällt der Gegenstand in Richtung Kreismittelpunkt (siehe Skizze)
\begin{eqnarray*}
	r-y = \frac{1}{2}\cdot a\cdot (\Delta t)^2 \to y = r - \frac{1}{2}\cdot a\cdot(\Delta t)^2\text{.}
\end{eqnarray*}

\begin{eqnarray*}
	\begin{split}
		r^2 = & x^2 + y^2\\
		    = & (v\cdot \Delta t)^2 + (r-\frac{1}{2}\cdot a\cdot (\Delta t)^2)^2\\
			= & (v\cdot \Delta t)^2 + r^2 - r\cdot a\cdot (\Delta t)^2 + \frac{1}{4}\cdot a^2 \cdot (\Delta t)^4\\
	0		= & (v\cdot \Delta t)^2 - r\cdot a\cdot (\Delta t)^2 + \frac{1}{4}\cdot a^2 \cdot (\Delta t)^4\\
	r\cdot a\cdot (\Delta t)^2		= & (v\cdot \Delta t)^2 + \frac{1}{4}\cdot a^2 \cdot (\Delta t)^4\\
	r\cdot a	= & v^2 + \frac{1}{4}\cdot a^2 \cdot (\Delta t)^2\\
	\end{split}
\end{eqnarray*}

Wir schauen nach sehr kurzer Zeit. $\Delta t$ ist also fast Null. 
Der zweite Term ist damit auch Null und es folgt

\begin{eqnarray*}
	a=\frac{v^2}{r}
\end{eqnarray*}

$a$ nennt man Zentrifugalbeschleunigung.

\end{document}
