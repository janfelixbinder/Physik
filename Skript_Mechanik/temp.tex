%\documentclass[11pt,a4paper,titlepage,twoside]{article}
\documentclass[12pt,a5paper,landscape]{article}
%\documentclass[12pt,a4paper,twoside]{scrartcl}
\usepackage{mystyle}

%\author{Felix Binder}
\title{Mechanik}
\date{}



\def\dir{../Aufgaben_Mechanik/}
\newcommand{\Einbinden}[1]{\input{#1}}

\addtocounter{page}{4}
\addtocounter{section}{9}
\addtocounter{aufgabe}{17}

\begin{document}

\begin{aufgabe}
	Lassen Sie einen Gegenstand (Stein, Metallmutter oder etwas anderes kleines schweres) aus verschiedenen
	Höhen fallen und messen Sie die Zeit, die es zum Herunterfallen benötigt (Fallzeit).
	Erstellen Sie eine Tabelle, in die Sie die Messwerte eintragen.
	Wiederholen Sie die Messungen einige Male (Sie bekommen dadurch Übung und die Messwerte werden genauer).
	Berechnen Sie für jede Fallhöhe den Mittelwert der Fallzeit und tragen Sie diese in ein
	Weg-Zeit-Diagramm ($x$-Achse Zeit, $y$-Achse Weg) ein.

	Können Sie einen Zusammenhang zwischen Fallhöhe und Fallzeit feststellen?
	Was müsste man in diesem Experiment verbessern?
\end{aufgabe}

\begin{aufgabe}
	Schauen Sie sich das Video an und übernehmen Sie die Messwerte in eine Tabelle.
	Zwischen jedem Bild vergeht \SI{0.1}{Sekunden}.
	Zeichnen Sie die Messwerte in ein Weg-Zeit-Diagramm ein.

	Kennen Sie die Form dieser Kurve? Wie gross ist die Konstante?
\end{aufgabe}



\end{document}
