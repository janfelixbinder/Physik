%\documentclass[11pt,a4paper,titlepage,twoside]{article}
\documentclass[12pt,a4paper,twoside]{article}
%\documentclass[12pt,a4paper,twoside]{scrartcl}
\usepackage{mystyle}

%\author{Felix Binder}
\title{Aufgaben für die Lernkontrolle Mechanik}
\date{}

\def\dir{/home/felix/GITHUB/Aufgaben_Mechanik/}
\newcommand{\Einbinden}[1]{\input{#1}}



\begin{document}
\maketitle

\begin{aufgabe}
	Nennen Sie mindestens drei Merkmale der gradlinig gleichförmig beschleunigten Bewegung.
	\begin{loesung}
		\begin{eqnarray*}
			\text{Weg}\aprox\text{Zeit}^2
		\end{eqnarray*}
		\begin{eqnarray*}
			\text{Geschwindigkeit}\aprox\text{Zeit}
		\end{eqnarray*}
		\begin{eqnarray*}
			\text{Beschleunigung} = \text{konstant}
		\end{eqnarray*}
	\end{loesung}
\end{aufgabe}



\begin{aufgabe}
	Im Urlaub in Kalifornien kaufen Sie eine Dose Cola für \SI{0.88}{\$}. 
	Die Dose hat ein Volumen von \SI{10}{fl\ oz}.
	Wie viel kostet ein Liter dieser Cola in Schweizer Franken?

	\SI{1}{fl\ oz} entspricht \SI{29,5735}{cm^3}.
	\SI{1}{SFr} entspricht \SI{1.0493}{\$}.

\end{aufgabe}

\begin{aufgabe}
In der Tabelle befinden sich Messwerte aus einer Messreihe.
\begin{enumerate} [a)]
	\item Tragen Sie die Daten in ein passendes Koordinatensystem ein.
	\item Um was für eine Art Bewegung handelt es sich?
	\item Bestimmen Sie die charakteristische Grösse dieser Bewegung.
\end{enumerate}

\begin{center}
	
	\begin{tabular}{l r |c c c c c c}
		Zeit & $t$ (s)	& 0         & 1         & 2         & 3          & 4          & 5\\\hline
		Weg  & $s$ (m) & \num{2.5} & \num{5.0} & \num{7.5} & \num{10.0} & \num{12.5} & \num{15.0} \\
	\end{tabular}
\end{center}
\end{aufgabe}

\Einbinden{\dir/kreis01.tex}


\begin{aufgabe}
	Sie lassen von der \SI{70}{m} hohen Poyabrücke einen Stein fallen.
	\begin{enumerate} [a)]
		\item Wie lange dauert es bis der Stein den Boden erreicht?
		\item Welche Geschwindigkeit hat der Stein beim Aufkommen?
	\end{enumerate}
\end{aufgabe}

\end{document}
