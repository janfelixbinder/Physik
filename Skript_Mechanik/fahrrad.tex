%\documentclass[11pt,a4paper,titlepage,twoside]{article}
\documentclass[12pt,a5paper,twoside]{article}
%\documentclass[12pt,a4paper,twoside]{scrartcl}
\usepackage{mystyle}
\usepackage{foto_v001}

%\author{Felix Binder}
\title{Das Fahrrad\\Bewerteter Schreibauftrag}
\date{}

\def\dir{/home/felix/GITHUB/Aufgaben_Mechanik/}
\newcommand{\Einbinden}[1]{\input{#1}}



\begin{document}
\maketitle


\begin{aufgabe}
	Welche Erfahrungen haben Sie mit dem Fahrrad fahren? Schreiben Sie einen kleinen Bericht über Ihre Erfahrungen.
	Betrachten Sie die folgenden Fragen als Hilfestellung.
	\begin{itemize}
		\item	Sie wollen eine gerade flache Strasse befahren. Welchen Gang wählen Sie dafür? Warum nehmen Sie diesen Gang?
		Welche Zahnkränze gehören zu diesem Gang? Der vordere Zahnkranz (am Pedal) hat einen grossen oder kleinen Radius?
		Der hintere Zahnkranz hat einen kleinen oder grossen Radius?

		\item Sie wollen eine steile Bergstrasse hochfahren. Welchen Gang wählen Sie nun? Welche Zahnkränze kommen nun zum Einsatz?
			Erklären Sie.

		\item Sie üben eine Kraft von \SI{100}{N} auf das Pedal aus.
			Berechnen Sie die Kraft, die am hinteren Rad auf die Strasse wirkt. Entnehmen Sie die nötigen Angaben aus der Zeichnung.
	
	\end{itemize}


	\begin{center}
	\Bildeinbinden{Velo.jpeg}{0.9}
	\end{center}

\end{aufgabe}

Bearbeiten Sie diese Aufgabe in einer halben Stunde und geben Sie Ihren Bericht in der nächsten Stunde ab.

%\newpage
%
%\begin{aufgabe}
%	An der Antriebsachse eines Autos erzeugt der Motor ein Drehmoment von \SI{2000}{Nm}.
%	Die Räder des Autos haben einen Radius von \SI{33}{cm}.
%
%	Wie gross ist die Kraft die über die Reifen auf die Strasse wirkt um das Auto zu beschleunigen?
%
%	\kloesung{\SI{6061}{N}}
%\end{aufgabe}
%
%\begin{aufgabe}
%	Auf einer Achse stecken zwei Scheiben mit unterschiedlichen Durchmessern.
%	An der ersten Scheibe mit 50 Zentimeter Durchmesser ist ein Schnur aufgewickelt.
%	Am Ende der Schnur hängt eine Masse von einem Kilogramm frei zu Boden.
%
%	\begin{enumerate} [a)]
%		\item Wie gross ist das Drehmoment an der Achse, das von der herunter hängenden Masse erzeugt wird.
%		\item An der zweiten Scheibe (\SI{10}{cm} Durchmesser) ist ebenfalls ein Faden befestigt an dessen Ende ein zweites Gewicht hängt.
%			Sinkt das erste Gewicht, steigt das zweite an. Wie schwer darf das zweite Gewicht maximal sein, damit es noch angehoben wird?
%	\end{enumerate}
%
%	\kloesung{a) \SI{2.45}{Nm}, b) \SI{5}{kg}}
%\end{aufgabe}
%
%
%
\end{document}
