%\documentclass[11pt,a4paper,titlepage,twoside]{article}
\documentclass[12pt,a4paper,twoside]{article}
%\documentclass[12pt,a4paper,twoside]{scrartcl}
\usepackage{mystyle}

%\author{Felix Binder}
\title{Ziele für den Mechaniktest}
\date{}



\def\dir{../Aufgaben_Mechanik/}
\newcommand{\Einbinden}[1]{\input{#1}}


\begin{document}
\maketitle

\section*{Experimente und Einheiten}
\begin{enumerate}
	\item Sie können ein Experiment detailliert erklären.
		\begin{itemize}
			\item Was soll mit dem Experiment untersucht werden?
			\item Sie können die Durchführung beschreiben.
			\item Mit welchen Schwierigkeiten ist bei der Durchführung zu rechnen?
			\item Sie können die Messwerte in einem Graphen darstellen.
		\end{itemize}
	\item Sie können die Einheiten physikalischer Grössen umrechnen.
\end{enumerate}

\section*{Kinematik}

\begin{enumerate}
	\item Sie können beschreiben, wie viel Weg ein Gegenstand während einer gleichmässig beschleunigten Bewegung zurücklegt.
	\item Sie können beschreiben, wie sich die Geschwindigkeit während einer gleichmässig beschleunigten Bewegung ändert.
%	\item Sie kennen die Definition der Geschwindigkeit und der Beschleunigung und können Sie in Aufgaben anwenden.
	\item Sie können die Formeln der Kinematik in Aufgaben anwenden.
	\item Sie können Weg-Zeit-Diagramme, Geschwindigkeit-Zeit-Diagramme und Beschleunigung-Zeit-Diagramme lesen und erstellen.
	
	\item Sie können erklären, warum man für Kreisbewegungen neue Grössen einführt.
	\item Sie können Frequenz, Umlaufzeit und Winkelgeschwindigkeit an Beispielen erklären.
	\item Sie können Frequenz, Umlaufzeit und Winkelgeschwindigkeit in Aufgaben anwenden.
\end{enumerate}

Der Test wird fünf Aufgaben zu unterschiedlichen Themen umfassen. Jede Aufgabe gibt gleich viele Punkte.
Um eine 4 zu erreichen müssen Sie also drei Aufgaben richtig bearbeiten.

\end{document}
