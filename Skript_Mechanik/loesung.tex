\documentclass[12pt,a4paper,twoside]{article}
%\documentclass[12pt,a4paper,twoside]{scrartcl}
\usepackage{mystyle}
\usepackage{gplot}
\usepackage{mechanik_v001}
\usepackage{foto_v001}


\usepackage[shell]{gnuplottex}
\usepackage{pgf}
\usepackage{pgfplots}
\usepackage{gnuplot-lua-tikz}
\pgfplotsset{compat=1.9}




%\author{Felix Binder}
\title{Mechanik}
\date{}



\def\dir{./Aufgaben_Mechanik/}
\newcommand{\Einbinden}[1]{\input{#1}}


\begin{document}
\begin{ex@loesung}{75}
\begin{enumerate}[a)]
\item Sie müssen Hubarbeit an der Masse des Fadenpendels verrichten.
Der genaue Weg, den die Masse zum Auslenkpunkt nimmt ist nicht entscheidend für die zu verrichtende Arbeit.
Sie müssen nur wissen, um welche Höhe Sie die Masse anheben.
\item Um die Höhe zu bestimmen um die Sie das Pendel auslenken, können Sie eine massstabsgetreue Zeichnung anfertigen und abmessen,
oder Sie berechnen die Höhe mit
\begin{eqnarray*}
h=\SI{1}{m} - \cos(\SI{30}{\degree}) = \SI{0.134}{m}\text{.}
\end{eqnarray*}

Die zu verrichtende Hubarbeit ist damit
\begin{eqnarray*}
\RI{W}{Hub} = m\cdot g\cdot h =\SI{0.1}{kg}\cdot\SI{9.81}{m/s^2}\cdot\SI{0.134}{m} =\SI{0.13}{J}\text{.}
\end{eqnarray*}
\item Das Pendel hat nun eine zusätzliche potentielle Energie von
\begin{eqnarray*}
\RI{E}{pot} = m\cdot g\cdot h =\SI{0.1}{kg}\cdot\SI{9.81}{m/s^2}\cdot\SI{0.134}{m} =\SI{0.13}{J}\text{.}
\end{eqnarray*}
Das ist genau die Menge Energie, die vorher in Form von Hubarbeit erbracht wurde.
\item Während der Schwingung des Pendels wird potentielle Energie in kinetische Energie und umgekehrt umgewandelt.
Durch den Energieerhaltungssatz wissen Sie, dass die Summe aus potentieller und kinetischer Energie gleich
bleibt. Damit können Sie die Geschwindigkeit für jeden beliebigen Punkt des Pendels berechnen.
Zum Beispiel ist an den Wendepunkten die Geschwindigkeit Null. Damit ist die kinetische Energie Null. Die Gesamtenergie ist
\begin{eqnarray*}
E =\RI{E}{pot} + \RI{E}{kin} = m\cdot g\cdot h + \frac{1}{2} m\cdot v^2 = \SI{0.13}{J}\text{.}
\end{eqnarray*}

Am tiefsten Punkt ist die potentielle Energie Null. Damit muss die kinetische Energie gleich der Gesamtenergie sein.
\begin{eqnarray*}
E =\RI{E}{pot} + \RI{E}{kin} \to \RI{E}{kin} = E - \RI{E}{pot}= \SI{0.13}{J}\text{.}
\end{eqnarray*}
Damit können wir jetzt die Geschwindigkeit berechnen.
\begin{eqnarray*}
\RI{E}{kin} =\frac{1}{2} m\cdot v^2 \to v^2 =\frac{2 \cdot\RI{E}{kin}}{m}=\frac{2\cdot\SI{0.13}{J}}{\SI{0.1}{kg}} = \SI{2.6}{m^2/s^2}
\end{eqnarray*}
Die Geschwindigkeit erhalten Sie nun, indem Sie die Wurzel ziehen $v=\SI{1.61}{m/s}$.


\end{enumerate}
\hrule \end{ex@loesung}
\begin{ex@loesung}{76}
\begin{itemize}
\item[a)]
\begin{eqnarray*}
\RI{E}{pot}=m\cdot g\cdot h=\SI{1.5}{kg}\cdot\SI{9.81}{m/s^2}\cdot\SI{40}{m}=\SI{588.6}{J}
\end{eqnarray*}
\item[b)]
\begin{eqnarray*}
\RI{E}{kin}=\frac{1}{2}\cdot m\cdot v^2=\SI{0.5}\cdot\SI{1.5}{kg}\cdot{0}{m/s}=\SI{0}{J}
\end{eqnarray*}
\item[c)] Die Gesamtenergie ist konstant.
\begin{eqnarray*}
\RI{E}{tot}=\RI{E}{pot} + \RI{E}{kin}=\SI{588.6}{J} + \SI{0}{J}=\SI{588.6}{J}
\end{eqnarray*}
\item[d)] Am Boden ist die Gesamtenergie genauso gross wie auf der Brücke (Energieerhaltung).
\begin{gather*}
\RI{E}{pot}=m\cdot g\cdot h=\SI{1.5}{kg}\cdot\SI{9.81}{m/s^2}\cdot\SI{0}{m}=\SI{0}{J}\\
\RI{E}{tot}=\RI{E}{pot} + \RI{E}{kin} \to \RI{E}{kin}=\RI{E}{tot}-\RI{E}{pot}=\SI{588.6}{J}-\SI{0}{J}=\SI{588.6}{J}\\
\RI{E}{kin}=\frac{1}{2}\cdot m\cdot v^2 \to v=\sqrt{\frac{2\cdot \RI{E}{kin}}{m}} =\SI{28.0}{m/s}
\end{gather*}
\end{itemize}
\hrule \end{ex@loesung}
\begin{ex@loesung}{77}
\begin{itemize}
\item [a)] Die Federkraft ist nicht konstant, sondern steigt linear mit der Auslenkung.
Die Fläche unter dem Arbeitsdiagramm ist
\begin{eqnarray*}
W=\frac{1}{2}\cdot\RI{F}{F}\cdot(\Delta x)=\frac{1}{2}\cdot D\cdot (\Delta x)^2=\num{0.5}\cdot\SI{200}{N/m}\cdot(\SI{0.15}{m})^2=\SI{2.25}{J}
\end{eqnarray*}
\item[b)] Das spannen der Feder hat \SI{2.25}{J} gekostet, damit ist die Federenergie $\RI{E}{Feder}=\SI{2.25}{J}$.
\item[c)] Es gilt Energieerhaltung. Die Federenergie wird vollständig in kinetische Energie umgewandelt.
\begin{eqnarray*}
\RI{E}{kin}=\frac{1}{2}\cdot m\cdot v^2 \to v=\sqrt{\frac{2\cdot \RI{E}{kin}}{m}} =\sqrt{\frac{2\cdot\SI{2.25}{J}}{\SI{1.7}{kg}}}=\SI[dp=2]{1.6270}{m/s}
\end{eqnarray*}
\item[d)] Wir rechnen zuerst die Reibungskraft aus. Aus der Tabelle finden wir die Reibungszahl $\mu$ für Gleitreibung Stahl auf Stahl.
Es wirkt nur die Gewichtskraft und die Normalkraft in vertikaler Richtung auf den Schlitten,
daraus folgt, dass die Normalkraft gleich
der Gewichtskraft ist.
\begin{eqnarray*}
\RI{F}{R}=\mu\cdot\RI{F}{N}=\mu\cdot m\cdot g=\num{0.1}\cdot\SI{1.7}{kg}\cdot\SI{9.81}{m/s^2}=\SI{1.67}{N}
\end{eqnarray*}
Es gibt nun zwei Möglichkeiten um zu berechnen, wie weit der Schlitten noch kommt.
\begin{itemize}
\item [(1)] Im ersten Fall berechnen wir die Arbeit, die die Oberfläche gegen den Schlitten verrichtet.
Wenn die kinetische Energie des Wagens vollständig in innere Energie $U$ umgewandelt wurde, kommt dieser zum Stehen.
\begin{eqnarray*}
W=F\cdot s =\RI{F}{R}\cdot s \to s =\frac{W}{\RI{F}{R}}=\frac{\SI{2.25}{J}}{\SI{1.67}{N}}=\SI[dp=2]{1.3473}{m}
\end{eqnarray*}
\item[(2)]
Mit $F=m\cdot a$ kann man nun die Beschleunigung ausrechnen. \RI{F}{R} wirkt entgegen der Bewegungsrichtung, also $-\RI{F}{R}$.
\begin{gather*}
a=\frac{F}{m} =\frac{\RI{F}{R}}{m}=\frac{\SI{-1.67}{N}}{\SI{1.7}{kg}}=\SI{-0.981}{m/s^2}\\
v^2=v_0^2+2\cdot a\cdot s \to s=\frac{v^2-v_0^2}{2\cdot a}=\frac{(\SI{0}{m/s})^2-(\SI[dp=2]{1.6270}{m/s})^2}{2\cdot(\SI{-0.981}{m/s^2})}=\SI[dp=2]{1.3492}{m}
\end{gather*}

\end{itemize}
\end{itemize}
\hrule \end{ex@loesung}
\begin{ex@loesung}{78}
\begin{itemize}
\item [a)] Der Wagen wandelt potentielle Energie in kinetische Energie.
\begin{eqnarray*}
\RI{E}{pot}= m\cdot g\cdot h=m\cdot g\cdot\sin(\alpha)\cdot \Delta s =\SI{3}{kg}\cdot\SI{9.81}{m/s^2}\cdot\num[dp=2]{0.42262}\cdot\SI{7.5}{m}=\SI{93.282}{J}
\end{eqnarray*}
Diese potentielle Energie hat der Wagen verloren. Gleichzeitig hat er denselben Betrag an kinetischer Energie gewonnen (Energieerhaltung).
Durch umstellen der kinetischen Energie nach $v$ erhalten wir
\begin{eqnarray*}
\RI{E}{kin}=\frac{1}{2}\cdot m\cdot v^2 \to v=\sqrt{\frac{2\cdot \RI{E}{kin}}{m}}=\sqrt{\frac{2\cdot\SI{93.28}{J}}{\SI{3}{kg}}}=\SI[dp=2]{7.8860}{m/s}
\end{eqnarray*}

\item[b)] Nun wird die kinetische Energie in Federenergie umgewandelt.
Da die kinetische Energie am Startpunkt Null war, ist die kinetische Energie unten, gleich der potentiellen Energie am Startpunkt.
\begin{eqnarray*}
\RI{E}{tot}=\RI{E}{pot}=m\cdot g\cdot h=m\cdot g\cdot\sin(\SI{25}{\degree})\cdot\SI{15}{m}=\SI{3}{kg}\cdot\SI{9.81}{m/s^2}\cdot\num[dp=2]{0.42262}\cdot\SI{15}{m}=\SI{186.56}{J}
\end{eqnarray*}
Aus der Federenergie kann die Auslenkung $\Delta x$ bestimmt werden
\begin{eqnarray*}
\RI{E}{Feder}=\frac{1}{2}\cdot D\cdot(\Delta x)^2 \to \Delta x = \sqrt{\frac{2\cdot\RI{E}{Feder}}{D}}=\sqrt{\frac{2\cdot\SI{186.56}{J}}{\SI{100}{N/m}}}=\sqrt{\SI[dp=2]{3.7313}{m^2}}=\SI[dp=2]{1.9317}{m}
\end{eqnarray*}

\item [c)] Zuerst berechnen wir die Reibungskraft. Die Gewichtskraft und die Normalkraft sind die einzigen Kräfte in vertikale Richtung.
Daher müssen beide gleich gross sein.
\begin{eqnarray*}
\RI{F}{R} = \mu\cdot\RI{R}{N} = \mu\cdot m\cdot g =\num{0.1}\cdot\SI{3}{m}\cdot\SI{9.81}{m/s^2}=\SI[dp=2]{2.9430}{N}
\end{eqnarray*}
Nun gibt es zwei Möglichkeiten auszurechen, wie weit der Schlitten noch fährt.
\begin{itemize}
\item [(1)] Wir wandeln die kinetische Energie in innere Energie $U$ um.
\begin{eqnarray*}
U = \RI{F}{R}\cdot s \to s=\frac{U}{\RI{F}{R}}=\frac{\SI{186.56}{J}}{\SI[dp=2]{2.9430}{N}} =\SI[dp=2]{63.393}{m}
\end{eqnarray*}
\item [(2)] Wir berechnen aus der Reibungskraft die Beschleunigung und dann damit den Bremsweg.
\begin{eqnarray*}
F=m\cdot a \to a=\frac{F}{m}=\frac{\SI[dp=2]{2.9430}{N}}{\SI{3}{kg}}=\SI[dp=2]{0.98100}{m/s^2}
\end{eqnarray*}
Aus der Gesamtenergie bestimmen wir die Geschwindikeit
\begin{eqnarray*}
\RI{E}{kin}=\frac{1}{2}\cdot m\cdot v^2 \to v=\sqrt{\frac{2\cdot \RI{E}{kin}}{m}}=\sqrt{\frac{2\cdot\SI{186.56}{J}}{\SI{3}{kg}}}=\SI[dp=2]{11.152}{m/s}
\end{eqnarray*}
Damit bekommen wir
\begin{eqnarray*}
v^2 = v_0^2 +2\cdot a\cdot s \to s=\frac{v^2 -v_0^2}{2\cdot a}=\frac{(\SI{0}{m/s})^2-(\SI[dp=2]{11.152}{m/s})^2}{2\cdot(\SI[dp=2]{-0.98100}{m/s^2})}=\frac{\SI[dp=2]{124.38}{m^2/s^2}}{\SI[dp=2]{1.9620}{m/s^2}}=\SI[dp=2]{63.393}{m}
\end{eqnarray*}
\end{itemize}
\end{itemize}
\hrule \end{ex@loesung}

\end{document}
