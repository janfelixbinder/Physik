%\documentclass[11pt,a4paper,titlepage,twoside]{article}
\documentclass[12pt,a4paper,twoside]{article}
%\documentclass[12pt,a4paper,twoside]{scrartcl}
\usepackage{mystyle}

%\author{Felix Binder}
\title{Ziele für die Mechanikprüfung}
\date{}



\def\dir{../Aufgaben_Mechanik/}
\newcommand{\Einbinden}[1]{\input{#1}}


\begin{document}
\maketitle

\section*{Energie}

\begin{enumerate}
	\item Sie können die verschiedenen Formen von Arbeit (Hubarbeit, Beschleunigungsarbeit, Verformungsarbeit)
		erklären und in Rechnungen anwenden.
	\item Sie können den Satz der Energieerhaltung erklären und in Rechnungen anwenden.
	\item Sie können den Begriff Leistung erklären und in Rechnungen anwenden.
	\item Sie können verschiedene Experimente erklären und den Zusammenhang mit dem Satz der Energieerhaltung aufzeigen.
		\begin{itemize}
			\item Fadenpendel
			\item Manneken Pis
			\item Jo-Jo
		\end{itemize}

\end{enumerate}


\end{document}
