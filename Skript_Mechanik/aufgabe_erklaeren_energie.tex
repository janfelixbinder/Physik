%\documentclass[11pt,a4paper,titlepage,twoside]{article}
\documentclass[12pt,a4paper,twoside]{article}
%\documentclass[12pt,a4paper,twoside]{scrartcl}
\usepackage{mystyle}
\usepackage{foto_v001}
\usepackage{gplot}
\usepackage{mechanik_v001}

%\author{Felix Binder}
\title{Aufgabe erklären und lösen\\Bewerteter Schreibauftrag}
\date{}

\def\dir{./Aufgaben_Mechanik/}
\newcommand{\Einbinden}[1]{\input{#1}}

\newcommand{\OBEN}[1]{%
\begin{tikzpicture}
	\draw (0,0) node [below right] {\Pruefung}rectangle(\textwidth,-3);
	\draw (\textwidth,0) node [below left] {\Datum};
	\draw (0,-3) node [above right]{Name: \ldots\ldots\ldots\ldots\ldots\ldots\ldots\ldots\ldots\ldots\ldots\ldots\ldots};
	\draw (\textwidth,-3) node [above left] {Klasse: \Klasse};
\end{tikzpicture}


Zur Vorbereitung auf die Prüfung (nach den Ferien) bereiten Sie eine Aufgabe für die ganze Klasse vor.
Die vorbereiteten Aufgaben schaue ich durch. Ich werde sie scannen und für alle zugänglich auf educanet2 bereitstellen.
Diese können Sie dann für die Vorbereitung auf die Prüfung benutzen.

Beachten Sie dazu die folgenden Reihenfolge:

\begin{itemize}
	\item Erklären Sie in eigenen Worten worum es in der Aufgabe geht.
	\item Zeigen Sie die Schwierigkeiten der Aufgabe auf.
	\item Lösen Sie die Aufgabe Schritt für Schritt.
\end{itemize}


Bitte schreiben Sie leserlich!

Stellen Sie bitte die folgende Aufgabe für alle vor. Benutzen Sie dafür
den vorgesehenen Platz auf der Rückseite!

#1
%\Einbinden{\dir/drehmomente03.tex}%Wippe

\newpage
}
\newcommand{\UNTEN}[0]{
\begin{itemize}
	\item Erklären Sie in eigenen Worten worum es in der Aufgabe geht.
		\par
		\Karo{4}

	\item Welche Schwierigkeiten gibt es beim Lösen dieser Aufgabe?
		\par
		\Karo{2}
	\item Lösung
		\par
		\Karo{14}
\end{itemize}

\newpage
}

\newcommand{\TAX}[1]{}


\begin{document}
\def\Datum{15.06.2015}
\def\Pruefung{Vorbereitung auf die Physikprüfung}
\def\Klasse{\ldots\ldots\ldots\ldots}

\OBEN{\Einbinden{\dir/hubarbeit01.tex}}
\UNTEN

\OBEN{\Einbinden{\dir/beschleunigungsarbeit01.tex}}
\UNTEN

\OBEN{\Einbinden{\dir/arbeit_feder01.tex}}
\UNTEN

\OBEN{\Einbinden{\dir/energieerhaltung_pendel.tex}}
\UNTEN

\OBEN{\Einbinden{\dir/energieerhaltung_bruecke.tex}}
\UNTEN

\OBEN{\Einbinden{\dir/energieerhaltung_feder.tex}}
\UNTEN

\OBEN{\Einbinden{\dir/energieerhaltung_wagen_feder.tex}}
\UNTEN


\OBEN{\Einbinden{\dir/leistung01.tex}}
\UNTEN

\OBEN{\Einbinden{\dir/leistung02.tex}}
\UNTEN

\OBEN{\Einbinden{\dir/leistung03.tex}}
\UNTEN

\OBEN{\Einbinden{\dir/leistung04.tex}}
\UNTEN

\OBEN{\Einbinden{\dir/leistung05.tex}}
\UNTEN

\OBEN{\Einbinden{\dir/arbeit05.tex}}
\UNTEN


\OBEN{\Einbinden{\dir/arbeit01.tex}}
\UNTEN

\OBEN{\Einbinden{\dir/arbeit03.tex}}
\UNTEN

\OBEN{\Einbinden{\dir/arbeit04.tex}}
\UNTEN

\OBEN{\Einbinden{\dir/stmichael_energie_jane.tex}}
\UNTEN

\OBEN{\Einbinden{\dir/stmichael_energie_ski.tex}}
\UNTEN

\OBEN{\Einbinden{\dir/stmichael_energie_schlitten.tex}}
\UNTEN

\OBEN{\Einbinden{\dir/stmichael_energie_bungee.tex}}
\UNTEN
\end{document}
