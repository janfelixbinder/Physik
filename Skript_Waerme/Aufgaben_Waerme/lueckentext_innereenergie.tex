
\StartLueckentext
Fügt man einem Gas \gl{Energie} zu, z.B. durch Wärme, so erhöht sich die Geschwindigkeit mit der sich die Gasteilchen bewegen.
Es gilt das Gesetz der kinetischen Energie aus der Mechanik: \gl{$E =\frac{1}{2}\cdot m\cdot v^2$}.

Wird einer \gl{Flüssigkeit} Energie zugeführt, dann werden die Bindungen zwischen den einzelnen Flüssigkeitsteilchen \gl{schwächer}.
Positionswechsel zwischen den Flüssigkeitsteilchen sind nun \gl{einfach} und kommen daher \gl{häufiger} vor.
Eine zähflüssige Flüssigkeit fliesst nun \gl{leichter}.

Auch in \gl{festen} Körpern schwächen sich die Bindungen zwischen den Atomen ab.
Diese werden aber nie so schwach, dass die Atome ihre \gl{Nachbarn} wechseln können.
Die Länge der Bindungen wird aber \gl{grösser}. Damit wird die \gl{Dichte} des Körpers kleiner.


Durch Zufuhr von Energie werden die Teilchen in einem Medium \gl{beweglicher}. Die innere Energie $U$ des Materials \gl{erhöht} sich.

\StoppLueckentext
