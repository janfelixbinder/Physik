
\def\MASS{10}
%\def\MASS{25}
\begin{aufgabe}
	Ein Silberbarren von \SI{\MASS}{kg} Gewicht, fällt aus einer Höhe von \SI{100}{m} auf eine unelastische Unterlage.
	\begin{itemize}
		\item [a)] Welche Energie-Umwandlung spielt sich ab?
		\item [b)] Wie gross ist die erzeugte Wärmemenge?
		\item [c)] Welche Temperaturerhöhung würde der Barren erfahren, wenn man von jeder Wärmeabgabe an die Umgebung absähe?
	\end{itemize}
\end{aufgabe}
