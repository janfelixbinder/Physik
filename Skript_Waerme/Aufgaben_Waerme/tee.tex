
\begin{aufgabe}
Sie wollen sich eine Tasse Tee (\SI{1.5}{dl}) kochen. 
Anfangs hat das Wasser eine Temperatur von \SI{20}{\celsius}.
Wie viel Wärme brauchen Sie zum Tee kochen?

\kloesung{\SI{50184}{J}}

\begin{loesung}
\begin{eqnarray*}
Q=\RI{c}{Wasser}\cdot m\cdot \Delta\vartheta=\SI{4182}{J/kg K}\cdot\SI{0.15}{kg}\cdot\SI{80}{K}=\SI{50184}{J}
\end{eqnarray*}
\end{loesung}

\end{aufgabe}

