
\def\DRUCK{0.15}
%\def\DRUCK{0.25}
\begin{aufgabe}
	Der Gasdruck in einem Gasthermometer mit konstantem Volumen betrage \SI{0.4}{bar} am Gefrierpunkt
	und \SI{0.546}{bar} am Siedepunkt des Wassers.
	\begin{itemize}
		\item [a)] Bei welcher Temperatur beträgt der Druck \SI{\DRUCK}{bar}?
		\item [b)] Wie hoch ist der Druck bei der Temperatur bei der Schwefel siedet? ($\vartheta=\SI{444.6}{\celsius}$)
	\end{itemize}
	\begin{loesung}
		\begin{itemize}
			\item [a)]
				\begin{eqnarray*}
					\frac{p\cdot V}{T}=\text{konstant}\to\frac{p_1\cdot V_1}{T_1}=\frac{p_2\cdot V_2}{T_2}\to T_2=\frac{p_2}{p_1}\cdot\frac{V_2}{V_1}\cdot T_1 =\frac{1}{4}\cdot 1\cdot \SI{273}{K}=\SI{68.25}{K} 
				\end{eqnarray*}
			\item[b)]
				\begin{eqnarray*}
					\frac{p\cdot V}{T}=\text{konstant}\to\frac{p_1\cdot V_1}{T_1}=\frac{p_2\cdot V_2}{T_2}\to p_2=\frac{V_1}{V_2}\cdot\frac{T_2}{T_1}\cdot p_1=1\cdot\frac{\SI{717.75}{K}}{\SI{273.15}{K}}\cdot\SI{400}{hPa}=\SI{1051.1}{hPa} 
				\end{eqnarray*}
		\end{itemize}
	\end{loesung}
\end{aufgabe}
