
\begin{aufgabe}
	Ein \num{1.5} Meter langer Kupferstab dehnt sich um \SI{1}{mm} aus. Um wie viele Grad Celsius wurde der Stab erwärmt?
	Der Längenausdehnungskoeffizient von Kupfer ist \SI{16.5E-6}{K^{-1}}.
	\begin{loesung}
		Für die Ausdehnung eines Stabes gilt:
		\begin{eqnarray*}
			\Delta l = l_1\cdot\Delta\vartheta\cdot\alpha\text{.}
		\end{eqnarray*}
		Damit folgt:
		\begin{eqnarray*}
			\Delta\vartheta=\frac{\Delta l}{l_1\cdot\alpha}=\frac{\SI{1E-3}{m}}{\SI{1.5}{m}\cdot\SI{16.5E-6}{K^{-1}}}=\SI{40.4}{\celsius}
		\end{eqnarray*}
	\end{loesung}
	\kloesung{\SI{40.4}{\celsius}}
\end{aufgabe}
