
\begin{aufgabe}
Sie wollen eine Flasche Weisswein kühlen. Der Wein hat Anfangs eine Temperatur von \SI{23}{\celsius}, gekühlt ist er \SI{8}{\celsius} kalt.
Wie viel Wärme müssen Sie der Flasche (\SI{0.7}{l}, \SI{10}{\percent} Alkohol, Gewicht des Glases \SI{400}{g}) entziehen?


\begin{loesung}
%\begin{eqnarray*}
%Q=\RI{c}{Wasser}\cdot m\cdot \Delta\vartheta=\SI{4182}{J/kg K}\cdot\SI{0.15}{kg}\cdot\SI{80}{K}=\SI{50184}{J}
%\end{eqnarray*}
\end{loesung}

\end{aufgabe}

