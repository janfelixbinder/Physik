
\StartLueckentext

Die Temperatur kennen Sie schon sehr lange.
Im Sommer ist es wärmer als im Winter, dies bedeutet die Temperatur ist im Sommer \gl{höher} als im Winter.
Haben Sie \gl{Fieber} ist Ihre Körpertemperatur um ein paar \gl{Grad Celsius} höher als normal.

Um die Temperatur eines Stoffs zu messen benutzt man ein \gl{Thermometer}.
Zum Messen bringt man das Thermometer und den Stoff dessen \gl{Temperatur} man messen möchte in thermischen \gl{Kontakt}.
Die Temperatur des Thermometers ändert sich dadurch so lange, bis die \gl{Temperatur} des Stoffs erreicht ist.
Nun kann man die Temperatur am Thermometer ablesen.

Es gibt verschiedene Temperaturskalen, Sie kennen \gl{Grad Celsius}, \gl{Grad Fahrenheit} und \gl{Kelvin}.
Diese \gl{Skalen} unterscheiden sich durch unterschiedliche Temperaturfixpunkte.
Um eine Temperaturskala zu definieren, braucht man zwei \gl{Fixpunkte}. 
Bei der Celsius Skala benutzt man den \gl{Gefrierpunkt} und den \gl{Siedepunkt} von Wasser als Fixpunkte und setzt diese auf \SI{0}{\celsius} und \SI{100}{\celsius}.

Eine modernere Temperaturskala ist die \gl{Kelvinskala}. 
Diese orientiert sich am absoluten \gl{Nullpunkt} der Temperatur.
Durch diese \gl{Definition} sind nur positive Temperaturen möglich.
Die Kelvinskala hat die selbe Schrittweite wie die \gl{Celsiusskala}.
\SI{0}{K} entspricht einer \gl{Temperatur} von \SI{-273.15}{\celsius}.

\StoppLueckentext

