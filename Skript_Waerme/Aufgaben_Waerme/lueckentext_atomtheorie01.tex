
\StartLueckentext

Materie, egal in welchem Aggregatzustand, besteht aus \gl{Atomen}, die sich unterschiedlich binden.

In Gasen formen Atome oft \gl{Moleküle}. Diese Gasteilchen können sich \gl{frei} bewegen.

Bei \gl{festen} Körpern, ordnen sich die Atome in einem Kristallgitter an.
Jedes Atom hat einen festen Platz mit festen benachbarten \gl{Atomen}, den es nicht verlassen kann.

In \gl{Flüssigkeiten} sind Atome oder Moleküle so dicht beieinander, dass die einzelnen Flüssigkeitsteilchen \gl{Bindungen} untereinander eingehen.
Werden diese \gl{schwachen} Bindungen gebrochen, 
können die Flüssigkeitsteilchen ihre \gl{Position} verändern und an einer anderen Stelle neue \gl{Bindungen} formen.

\StoppLueckentext

