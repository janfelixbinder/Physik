
\begin{aufgabe}
Seit einigen Jahren sind selbstkühlende Getränkefässer im Handel erhältlich.
Sie funktionieren auf dem Prinzip der Verdampfungskühlung.
Wasser wird verdampft und die erforderliche Energie für diesen Vorgang wird dem zu kühlenden Getränk entzogen.

\begin{itemize}
	\item [a)] Wie viel Wärme muss man \SI{20}{Litern} (\SI{20}{kg}) Bier (\SI{95}{\percent} Wasser, \SI{5}{\percent} Ethanol) entziehen, um es von \SI{25}{\celsius} auf \SI{8}{\celsius} zu kühlen?
	\item [b)] Wie viel Wasser muss man verdunsten, um die in a) berechnete Wärmemenge zu entziehen?
\end{itemize}
\end{aufgabe}

