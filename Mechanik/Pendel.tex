\documentclass[12pt,a4paper,twoside,DIV13,BCOR1cm,landscape]{scrartcl}


\usepackage{style}


\newcommand{\QRcode}[2]{\tikz \node (Bild) at (0,0) {\includegraphics[width=0.3\textwidth] {#1}}; \tikz \node [below] (Code) at (Bild) {#2};}
%\node (Bild) at (0,0) {Bild};
%\node [below] (Code) at (Bild) {Code};




%Kopfzeile
%gerade   linke  seiten mit \lehead, \cehead und \rehead
%ungerade rechte seiten mit \lohead, \cohead und \rohead

%Fusszeile wie bei der Kopfzeile nur statt head foot
%\lefoot 
\usepackage{scrlayer-scrpage}
\lohead{Felix Binder}
\rohead{Mechanik}
\lofoot{Verlag Naseblau, Irgendwo}

\pagestyle{scrheadings}




\begin{document}

\begin{aufgabe}
Überlegen Sie sich, von welchen Grössen die Periodendauer eines Fadenpendels abhängen könnte.
Rufen Sie dann mit Ihrem Handy den Fragebogen auf und beantworten Sie die Fragen.
\QRcode{/home/felix/Schule1415/qr-code_fadenpendel.png}{http://goo.gl/mE3Uh6}

\end{aufgabe}

\newpage
\includesolutions

\end{document}

