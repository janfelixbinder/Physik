\section*{Die schiefe Ebene}
In dieser Lektion werden wir uns mit der schiefen Ebene beschäftigen. In einem Versuch werden wir sehen, was mit einem
Klotz passiert, der auf einer angewinkelten Ebene liegt. Machen Sie sich neben den Skizzen Notizen und
zeichnen Sie alle auftretenden Kräfte ein (Gewichtskraft, Normalkraft, Reibungskraft).
Dies ist ein Beispiel, in dem Gewichtskraft und Normalkraft nicht gleich gross sind.

\subsection*{Keine Steigung der Ebene}

\Spalten{0.5}{
\begin{center}
\begin{tikzpicture}

\def\angle{0}%Winkel

\coordinate (P0) at (6,0);

\draw (0,0)--(7,0);
\draw (P0)--++(180-\angle:6);


%winkel
%\draw [->] (5,0) arc(180:180-\angle:1cm);%winkelbogen
%\draw  (P0) ++(180-0.5*\angle:0.75) node {$\alpha$};%alpha

\def\hoehe{1}
\def\breite{2}
\draw (P0) ++(180-\angle:2.25cm)--++(90-\angle:\hoehe)--++(180-\angle:\breite)--++(270-\angle:\hoehe);

%\draw [color=white] (0,-3)--(2,-3);%abstand nach unten
	\end{tikzpicture}
\end{center}
\vspace*{1.5cm}
}{0.5}{}

\subsection*{Geringe Steigung der Ebene}
\Spalten{0.5}{
\begin{center}
\begin{tikzpicture}

\def\angle{15}%Winkel

\coordinate (P0) at (6,0);

\draw (0,0)--(7,0);
\draw (P0)--++(180-\angle:6);


%winkel
\draw [->] (5,0) arc(180:180-\angle:1cm);%winkelbogen
\draw  (P0) ++(180-0.5*\angle:0.75) node {$\alpha$};%alpha

\def\hoehe{1}
\def\breite{2}
\draw (P0) ++(180-\angle:2.25cm)--++(90-\angle:\hoehe)--++(180-\angle:\breite)--++(270-\angle:\hoehe);
	\end{tikzpicture}
\end{center}
\vspace*{1.5cm}
}{0.5}{}

\subsection*{Grosse Steigung der Ebene}
\Spalten{0.5}{
\begin{center}
\begin{tikzpicture}

\def\angle{65}%Winkel

\coordinate (P0) at (6,0);

\draw (0,0)--(7,0);
\draw (P0)--++(180-\angle:6);


%winkel
\draw [->] (5,0) arc(180:180-\angle:1cm);%winkelbogen
\draw  (P0) ++(180-0.5*\angle:0.75) node {$\alpha$};%alpha

\def\hoehe{1}
\def\breite{2}
\draw (P0) ++(180-\angle:2.25cm)--++(90-\angle:\hoehe)--++(180-\angle:\breite)--++(270-\angle:\hoehe);
	\end{tikzpicture}
\end{center}
\vspace*{1.5cm}
}{0.5}{}

\Einbinden{\dir/schiefeEbene01.tex}
