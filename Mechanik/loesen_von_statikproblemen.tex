
\section*{Lösen von Statikproblemen mit dem Drehmoment}

%In der letzten Lektion haben wir uns mit dem Hebelgesetz beschäftigt.
%Eine bekannte Formulierung des Hebelgesetzes lautet:
%\begin{center}
%%	\begin{large}
%Kraft mal Kraftarm gleich Last mal Lastarm.
%%	\end{large}
%\end{center}

%Weiter sind wir auf das Drehmoment gestossen.
Das Drehmoment ist allgemein als Kreuzprodukt zweier Vektoren definiert:

\begin{cbox}
\begin{gather*}
	\text{Drehmoment} = \text{Kreuzprodukt von Hebelarm und Kraft} \quad\text{oder}\quad \vec{M} = \vec{r}\times\vec{F}\\
	\text{Einheit}: [\vec{M}]=\text{Meter}\cdot\text{Kraft}=\si{m}\cdot\si{N}=\si{Nm}
\end{gather*}
\end{cbox}

Das bedeutet, dass die Komponente der Kraft, die senkrecht auf dem Hebelarm steht mal dem Hebelarm gerechnet werden muss.

\Spalten{0.5}{
\Beispiel Ein realer Hebelarm hat immer auch ein Eigengewicht. Seine Gewichtskraft greift am Schwerpunkt an.
Die Richtung der Gewichtskraft ist im allgemeinen nicht senkrecht zum Hebelarm. Deshalb muss man die Gewichtskraft
zerlegen, in eine Komponente senkrecht zum Hebelarm und eine Komponente parallel zum Hebelarm.
Rechnerisch ist dass hier $\sin\alpha\cdot\RI{F}{G}$. Damit ergibt sich für das Drehmoment
\begin{eqnarray*}
	M=r\cdot \RI{F}{G}\cdot \sin\alpha\text{.}
\end{eqnarray*}
}{0.5}{
\begin{center}
	\begin{tikzpicture}

\coordinate (P0) at (30:2cm);
\draw [fill] (0,0) circle (0.1); \draw (0,0) node [left] {Drehachse};
\draw [fill] (P0) circle (0.1); \draw (P0) node [left] {Schwerpunkt};
\draw (0,0) --(30:4cm) node [ above,sloped] {Hebelarm};
\draw [Kraft,->] (P0)--++(-90:3cm) node [midway, left] {\RI{F}{G}}node [shape=coordinate] (EK){};

\draw [Winkel,->] (30:1cm) arc (210:270:1cm); \draw (P0) ++(240:0.8cm) node  {$\alpha$};

\draw [dotted] (P0)--++(-60:3cm)  node [midway,right] {$\sin\alpha\cdot\RI{F}{G}$};
\draw [dotted] (EK)--++(30:2.5cm) node [shape=coordinate] (EK){};
	\end{tikzpicture}
\end{center}
}

Zum Lösen von Problemen aus der Statik kann man das Drehmoment auch benutzen. Im statischen Fall muss die Summe der
Kräfte und die Summe der Drehmomente an jedem beliebigen Punkt Null sein.
\begin{align*}
	\vec{F_1} + \vec{F_2} + \vec{F_3} + \cdots + \vec{F_n} = 0\\
	\vec{M_1} + \vec{M_2} + \vec{M_3} + \cdots + \vec{M_n} = 0
\end{align*}


\Einbinden{\dir/loesen01.tex}
\Einbinden{\dir/loesen02.tex}

