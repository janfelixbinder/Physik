\section{Geschwindigkeit}
Die Geschwindigkeit ist eine abgeleitete Grösse. Sie gibt an, wie viel Weg $\Delta s$ in einem Zeitintervall $\Delta t$
zurückgelegt werden.
Wenn sich weder Betrag noch Richtung der Geschwindigkeit ändern, so spricht man von einer \emph{gradlinig gleichförmig} Geschwindigkeit. 
\begin{cbox}
\begin{gather*}
	\text{Geschwindigkeit} = \frac{\text{Weg}}{\text{Zeit}}\quad\text{oder}\quad \bar{v}=\frac{\Delta s}{\Delta t}\\
		\text{Einheit}: [v] = \frac{\text{Meter}}{\text{Sekunde}}=\frac{\si{m}}{\si{s}}
\end{gather*}
\end{cbox}

Mit dem Strich über dem $\bar{v}$ wird angedeutet, dass eine durchschnittliche Geschwindigkeit
gemeint ist. Diese kann von der \emph{Momentangeschwindigkeit} abweichen, wenn das bewegte Teilchen beschleunigt wird.

\Einbinden{\dir/geschwindigkeit01.tex}
\Einbinden{\dir/geschwindigkeit02.tex}
\Einbinden{\dir/geschwindigkeit03.tex}
\Einbinden{\dir/geschwindigkeit04.tex}
\Einbinden{\dir/geschwindigkeit05.tex}

