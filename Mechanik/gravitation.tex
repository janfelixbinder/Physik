
\section*{Kräfte}
In der Physik gibt es heute vier fundamentale Kräfte. Die starke und die schwache Kernkraft, die elektromagnetische Kraft und die Gravitation.
Kräfte werden in der Physik mit dem Buchstaben $F$ abgekürzt.
Die Einheit für alle Kräfte ist das Newton N. Das Newton ist eine zusammengesetzte Einheit
\begin{eqnarray*}
	\text{N}=\frac{\text{kg}\cdot\text{m}}{\text{s}^2}\text{.}
\end{eqnarray*}

\subsection*{Gravitation}
In diesem Kapitel wollen wir uns mit der Gravitation beschäftigen. Sie wirkt auf alle Körper mit einer Masse.
Die Anziehungskraft, die ein Körper mit der Masse $m_1$ auf einen Körper mit der Masse $m_2$ ausübt, ist proportional zu den Massen $m_1$ und $m_2$.
Je grösser der Abstand zwischen den zwei Massen, desto kleiner ist die Anziehungskraft zwischen ihnen. Die Abnahme der Anziehungskraft geht quadratisch
mit dem Abstand zwischen den Körpern ein ($F\sim\nicefrac{1}{r^2}$).


\begin{aufgabe}
	Lesen Sie die obige Einführung in die Kräfte, und stellen Sie eine Gleichung für die Anziehungskraft zwischen zwei Massen her. 
\end{aufgabe}
\vspace*{2.5cm}
\begin{aufgabe}
	Welche Einheit hat die Gravitationskonstante? 
\end{aufgabe}

\vspace*{2.5cm}
%Die Gravitationskonstante G hat einen Wert von \SI{6.674E-11}{Nm^2/kg^2}
Die Gravitationskonstante G hat einen Wert von \num{6.674E-11} \gl.

\Einbinden{\dir/gravitation01.tex}
\Einbinden{\dir/gravitation02.tex}

Da es recht aufwendig ist, die Gravitationskraft zwischen Erde und den Objekten auf der Erde zu berechnen, führt man eine Vereinfachung ein.
Den Ausdruck für die Gravitationskraft formen wir um und erhalten die Gewichtskraft. 
\begin{eqnarray*}
	\RI{F}{G} = m\cdot g
\end{eqnarray*}
Der Faktor $g$ ist die Fallbeschleunigung. Es ist ein Ortsfaktor, der nicht überall auf der Erde gleich gross ist, aber immer etwa
\SI{9.81}{m/s^2} gross ist.


\Einbinden{\dir/gravitation03.tex}
