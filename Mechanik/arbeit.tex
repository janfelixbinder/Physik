\section*{Arbeit und Energie}
In der Umgangssprache ist der Begriff \emph{Arbeit} nicht klar definiert.
Viele Menschen arbeiten im Büro und schreiben, telefonieren und diskutieren.
Andere arbeiten auf Baustellen und tragen Steine und schweres Gerät.

In der Physik ist die \emph{Arbeit} klar definiert.
\emph{Arbeit} ist das Skalarprodukt aus \emph{Kraft} und \emph{Weg}.
Die Einheit der Arbeit ist das {\bf Joule}.

\begin{cbox}
\begin{gather*}
    \text{Arbeit} = \text{Kraft}\cdot{\text{Weg}}\quad\text{oder}\quad W=F\cdot s\cdot \cos \alpha\\
	\text{Einheit}: [W] = \text{Newton}\cdot\text{Meter}=\si{N}\cdot\si{m}=\frac{\si{kg}\cdot\si{m}}{\si{s^2}}\cdot \si{m}=\text{Joule}=\si{J}
\end{gather*}
\end{cbox}

\Einbinden{\dir/arbeit_schrank.tex}
\Einbinden{\dir/arbeit_schlitten.tex}


\subsection*{Hubarbeit}
Um einen Körper der Masse $m$ um eine Höhe $h$ anzuheben, ist eine \emph{Hubarbeit} erforderlich.
\begin{equation*}
	W_{\text{Hub}} = F \cdot h = F_{\text{G}}\cdot h = m\cdot g\cdot h
\end{equation*}

Im Vergleich zu seiner ursprünglichen Lage hat der Körper eine höhere potentielle Energie.

\begin{cbox}
\begin{equation*}
	E_{\text{pot}} = m\cdot g\cdot h
\end{equation*}
\end{cbox}


\Einbinden{\dir/arbeit_hub.tex}




\subsection*{Beschleunigungsarbeit}
Um einen Körper zu beschleunigen ist eine Kraft nötig. Es gilt $F=m\cdot a$ wie wir auf Seite ~\pageref{newton2} gesehen haben.
Damit ist klar, dass es auch eine Beschleunigungsarbeit gibt.
\begin{eqnarray*}
	\RI{W}{Bew}=F\cdot s=m\cdot a\cdot s=\frac{1}{2}\cdot m\cdot\left(v^2-v_0^2\right)
\end{eqnarray*}

Dabei ist $m$ die Masse, $a$ die Beschleunigung, $s$ der Weg und $v$ die Geschwindigkeit.
Im letzten Schritt haben wir für $a\cdot s$ einen Ausdruck eingesetzt, den wir durch umstellen aus der Formel
$v^2=v_0^2+2\cdot a\cdot s$ bekommen haben.

Hat der Körper die Geschwindigkeit $v$ erreicht, so besitzt er eine \emph{kinetische Energie}.

\begin{cbox}
\begin{equation*}
	\RI{E}{kin} = \frac{1}{2}\cdot m\cdot v^2
\end{equation*}
\end{cbox}

\Einbinden{\dir/arbeit_velo.tex}
\Einbinden{\dir/Ekin_auto.tex}

\newpage

\subsection*{Verformungsarbeit}
In diesem Abschnitt werden wir lernen, wie man die Arbeit für eine nicht konstante Kraft bestimmen kann.
Arbeit ist Kraft mal Weg. Ändert sich die Kraft über den Weg, muss man dies berücksichtigen.

\begin{figure}[h!]
\begin{center}
\begin{tikzpicture}[gnuplot]
%% generated with GNUPLOT 4.6p1 (Lua 5.1; terminal rev. 99, script rev. 100)
%% Mit 05 Dez 2012 16:08:51 CET
\path (0.000,0.000) rectangle (15.000,5.000);
\gpcolor{color=gp lt color border}
\gpsetlinetype{gp lt border}
\gpsetlinewidth{1.00}
\draw[gp path] (1.320,4.631)--(1.320,0.985)--(6.947,0.985)--(6.947,4.631)--cycle;
\node[gp node center,rotate=-270] at (0.246,2.808) {Kraft (N)};
\node[gp node center] at (4.133,0.215) {Weg (m)};
\gpfill{rgb color={0.063,0.318,0.502}} (1.320,0.985)--(1.320,0.985)--(1.377,1.022)--(1.434,1.059)%
    --(1.491,1.095)--(1.547,1.132)--(1.604,1.169)--(1.661,1.206)--(1.718,1.243)%
    --(1.775,1.280)--(1.832,1.316)--(1.888,1.353)--(1.945,1.390)--(2.002,1.427)%
    --(2.059,1.464)--(2.116,1.501)--(2.173,1.537)--(2.229,1.574)--(2.286,1.611)%
    --(2.343,1.648)--(2.400,1.685)--(2.457,1.722)--(2.514,1.758)--(2.570,1.795)%
    --(2.627,1.832)--(2.684,1.869)--(2.741,1.906)--(2.798,1.943)--(2.855,1.979)%
    --(2.911,2.016)--(2.968,2.053)--(3.025,2.090)--(3.082,2.127)--(3.139,2.164)%
    --(3.196,2.200)--(3.253,2.237)--(3.309,2.274)--(3.366,2.311)--(3.423,2.348)%
    --(3.480,2.384)--(3.537,2.421)--(3.594,2.458)--(3.650,2.495)--(3.707,2.532)%
    --(3.764,2.569)--(3.821,2.605)--(3.878,2.642)--(3.935,2.679)--(3.991,2.716)%
    --(4.048,2.753)--(4.105,2.790)--(4.162,2.826)--(4.219,2.863)--(4.276,2.900)%
    --(4.332,2.937)--(4.389,2.974)--(4.446,3.011)--(4.503,3.047)--(4.560,3.084)%
    --(4.617,3.121)--(4.673,3.158)--(4.730,3.195)--(4.787,3.232)--(4.844,3.268)%
    --(4.901,3.305)--(4.958,3.342)--(5.014,3.379)--(5.071,3.416)--(5.128,3.452)%
    --(5.185,3.489)--(5.242,3.526)--(5.299,3.563)--(5.356,3.600)--(5.412,3.637)%
    --(5.469,3.673)--(5.526,3.710)--(5.583,3.747)--(5.640,3.784)--(5.697,3.821)%
    --(5.753,3.858)--(5.810,3.894)--(5.867,3.931)--(5.924,3.968)--(5.981,4.005)%
    --(6.038,4.042)--(6.094,4.079)--(6.151,4.115)--(6.208,4.152)--(6.265,4.189)%
    --(6.322,4.226)--(6.379,4.263)--(6.435,4.300)--(6.492,4.336)--(6.549,4.373)%
    --(6.606,4.410)--(6.663,4.447)--(6.720,4.484)--(6.776,4.521)--(6.833,4.557)%
    --(6.890,4.594)--(6.947,4.631)--(6.947,0.985)--cycle;
\gpcolor{rgb color={0.063,0.318,0.502}}
\gpsetlinetype{gp lt plot 0}
\gpsetlinewidth{4.00}
\draw[gp path] (1.320,0.985)--(1.377,1.022)--(1.434,1.059)--(1.491,1.095)--(1.547,1.132)%
  --(1.604,1.169)--(1.661,1.206)--(1.718,1.243)--(1.775,1.280)--(1.832,1.316)--(1.888,1.353)%
  --(1.945,1.390)--(2.002,1.427)--(2.059,1.464)--(2.116,1.501)--(2.173,1.537)--(2.229,1.574)%
  --(2.286,1.611)--(2.343,1.648)--(2.400,1.685)--(2.457,1.722)--(2.514,1.758)--(2.570,1.795)%
  --(2.627,1.832)--(2.684,1.869)--(2.741,1.906)--(2.798,1.943)--(2.855,1.979)--(2.911,2.016)%
  --(2.968,2.053)--(3.025,2.090)--(3.082,2.127)--(3.139,2.164)--(3.196,2.200)--(3.253,2.237)%
  --(3.309,2.274)--(3.366,2.311)--(3.423,2.348)--(3.480,2.384)--(3.537,2.421)--(3.594,2.458)%
  --(3.650,2.495)--(3.707,2.532)--(3.764,2.569)--(3.821,2.605)--(3.878,2.642)--(3.935,2.679)%
  --(3.991,2.716)--(4.048,2.753)--(4.105,2.790)--(4.162,2.826)--(4.219,2.863)--(4.276,2.900)%
  --(4.332,2.937)--(4.389,2.974)--(4.446,3.011)--(4.503,3.047)--(4.560,3.084)--(4.617,3.121)%
  --(4.673,3.158)--(4.730,3.195)--(4.787,3.232)--(4.844,3.268)--(4.901,3.305)--(4.958,3.342)%
  --(5.014,3.379)--(5.071,3.416)--(5.128,3.452)--(5.185,3.489)--(5.242,3.526)--(5.299,3.563)%
  --(5.356,3.600)--(5.412,3.637)--(5.469,3.673)--(5.526,3.710)--(5.583,3.747)--(5.640,3.784)%
  --(5.697,3.821)--(5.753,3.858)--(5.810,3.894)--(5.867,3.931)--(5.924,3.968)--(5.981,4.005)%
  --(6.038,4.042)--(6.094,4.079)--(6.151,4.115)--(6.208,4.152)--(6.265,4.189)--(6.322,4.226)%
  --(6.379,4.263)--(6.435,4.300)--(6.492,4.336)--(6.549,4.373)--(6.606,4.410)--(6.663,4.447)%
  --(6.720,4.484)--(6.776,4.521)--(6.833,4.557)--(6.890,4.594)--(6.947,4.631);
\gpcolor{color=gp lt color axes}
\gpsetlinetype{gp lt axes}
\gpsetlinewidth{1.00}
\draw[gp path] (1.320,0.985)--(6.947,0.985);
\gpcolor{color=gp lt color border}
\gpsetlinetype{gp lt border}
\draw[gp path] (1.320,0.985)--(1.500,0.985);
\draw[gp path] (6.947,0.985)--(6.767,0.985);
\node[gp node right] at (1.136,0.985) { 0};
\gpcolor{color=gp lt color axes}
\gpsetlinetype{gp lt axes}
\draw[gp path] (1.320,1.593)--(6.947,1.593);
\gpcolor{color=gp lt color border}
\gpsetlinetype{gp lt border}
\draw[gp path] (1.320,1.593)--(1.500,1.593);
\draw[gp path] (6.947,1.593)--(6.767,1.593);
\node[gp node right] at (1.136,1.593) { 5};
\gpcolor{color=gp lt color axes}
\gpsetlinetype{gp lt axes}
\draw[gp path] (1.320,2.200)--(6.947,2.200);
\gpcolor{color=gp lt color border}
\gpsetlinetype{gp lt border}
\draw[gp path] (1.320,2.200)--(1.500,2.200);
\draw[gp path] (6.947,2.200)--(6.767,2.200);
\node[gp node right] at (1.136,2.200) { 10};
\gpcolor{color=gp lt color axes}
\gpsetlinetype{gp lt axes}
\draw[gp path] (1.320,2.808)--(6.947,2.808);
\gpcolor{color=gp lt color border}
\gpsetlinetype{gp lt border}
\draw[gp path] (1.320,2.808)--(1.500,2.808);
\draw[gp path] (6.947,2.808)--(6.767,2.808);
\node[gp node right] at (1.136,2.808) { 15};
\gpcolor{color=gp lt color axes}
\gpsetlinetype{gp lt axes}
\draw[gp path] (1.320,3.416)--(6.947,3.416);
\gpcolor{color=gp lt color border}
\gpsetlinetype{gp lt border}
\draw[gp path] (1.320,3.416)--(1.500,3.416);
\draw[gp path] (6.947,3.416)--(6.767,3.416);
\node[gp node right] at (1.136,3.416) { 20};
\gpcolor{color=gp lt color axes}
\gpsetlinetype{gp lt axes}
\draw[gp path] (1.320,4.023)--(6.947,4.023);
\gpcolor{color=gp lt color border}
\gpsetlinetype{gp lt border}
\draw[gp path] (1.320,4.023)--(1.500,4.023);
\draw[gp path] (6.947,4.023)--(6.767,4.023);
\node[gp node right] at (1.136,4.023) { 25};
\gpcolor{color=gp lt color axes}
\gpsetlinetype{gp lt axes}
\draw[gp path] (1.320,4.631)--(6.947,4.631);
\gpcolor{color=gp lt color border}
\gpsetlinetype{gp lt border}
\draw[gp path] (1.320,4.631)--(1.500,4.631);
\draw[gp path] (6.947,4.631)--(6.767,4.631);
\node[gp node right] at (1.136,4.631) { 30};
\gpcolor{color=gp lt color axes}
\gpsetlinetype{gp lt axes}
\draw[gp path] (1.320,0.985)--(1.320,4.631);
\gpcolor{color=gp lt color border}
\gpsetlinetype{gp lt border}
\draw[gp path] (1.320,0.985)--(1.320,1.165);
\draw[gp path] (1.320,4.631)--(1.320,4.451);
\node[gp node center] at (1.320,0.677) { 0};
\gpcolor{color=gp lt color axes}
\gpsetlinetype{gp lt axes}
\draw[gp path] (2.445,0.985)--(2.445,4.631);
\gpcolor{color=gp lt color border}
\gpsetlinetype{gp lt border}
\draw[gp path] (2.445,0.985)--(2.445,1.165);
\draw[gp path] (2.445,4.631)--(2.445,4.451);
\node[gp node center] at (2.445,0.677) { 0.05};
\gpcolor{color=gp lt color axes}
\gpsetlinetype{gp lt axes}
\draw[gp path] (3.571,0.985)--(3.571,4.631);
\gpcolor{color=gp lt color border}
\gpsetlinetype{gp lt border}
\draw[gp path] (3.571,0.985)--(3.571,1.165);
\draw[gp path] (3.571,4.631)--(3.571,4.451);
\node[gp node center] at (3.571,0.677) { 0.1};
\gpcolor{color=gp lt color axes}
\gpsetlinetype{gp lt axes}
\draw[gp path] (4.696,0.985)--(4.696,4.631);
\gpcolor{color=gp lt color border}
\gpsetlinetype{gp lt border}
\draw[gp path] (4.696,0.985)--(4.696,1.165);
\draw[gp path] (4.696,4.631)--(4.696,4.451);
\node[gp node center] at (4.696,0.677) { 0.15};
\gpcolor{color=gp lt color axes}
\gpsetlinetype{gp lt axes}
\draw[gp path] (5.822,0.985)--(5.822,4.451)--(5.822,4.631);
\gpcolor{color=gp lt color border}
\gpsetlinetype{gp lt border}
\draw[gp path] (5.822,0.985)--(5.822,1.165);
\draw[gp path] (5.822,4.631)--(5.822,4.451);
\node[gp node center] at (5.822,0.677) { 0.2};
\gpcolor{color=gp lt color axes}
\gpsetlinetype{gp lt axes}
\draw[gp path] (6.947,0.985)--(6.947,4.631);
\gpcolor{color=gp lt color border}
\gpsetlinetype{gp lt border}
\draw[gp path] (6.947,0.985)--(6.947,1.165);
\draw[gp path] (6.947,4.631)--(6.947,4.451);
\node[gp node center] at (6.947,0.677) { 0.25};
\draw[gp path] (1.320,4.631)--(1.320,0.985)--(6.947,0.985)--(6.947,4.631)--cycle;
%% coordinates of the plot area
\gpdefrectangularnode{gp plot 1}{\pgfpoint{1.320cm}{0.985cm}}{\pgfpoint{6.947cm}{4.631cm}}
\draw[gp path] (9.004,4.631)--(9.004,0.985)--(14.447,0.985)--(14.447,4.631)--cycle;
\node[gp node center,rotate=-270] at (7.746,2.808) {Kraft (N)};
\node[gp node center] at (11.725,0.215) {Weg (m)};
\gpfill{rgb color={0.063,0.318,0.502}} (9.004,0.985)--(9.004,0.985)--(9.059,1.048)--(9.114,1.111)%
    --(9.169,1.174)--(9.224,1.237)--(9.279,1.300)--(9.334,1.363)--(9.389,1.425)%
    --(9.444,1.488)--(9.499,1.550)--(9.554,1.612)--(9.609,1.673)--(9.664,1.734)%
    --(9.719,1.795)--(9.774,1.856)--(9.829,1.916)--(9.884,1.976)--(9.939,2.035)%
    --(9.994,2.094)--(10.049,2.152)--(10.104,2.210)--(10.159,2.267)--(10.214,2.324)%
    --(10.269,2.380)--(10.324,2.435)--(10.378,2.490)--(10.433,2.544)--(10.488,2.598)%
    --(10.543,2.650)--(10.598,2.703)--(10.653,2.754)--(10.708,2.804)--(10.763,2.854)%
    --(10.818,2.903)--(10.873,2.952)--(10.928,2.999)--(10.983,3.045)--(11.038,3.091)%
    --(11.093,3.136)--(11.148,3.180)--(11.203,3.223)--(11.258,3.265)--(11.313,3.307)%
    --(11.368,3.347)--(11.423,3.386)--(11.478,3.425)--(11.533,3.463)--(11.588,3.499)%
    --(11.643,3.535)--(11.698,3.570)--(11.753,3.603)--(11.808,3.636)--(11.863,3.668)%
    --(11.918,3.699)--(11.973,3.729)--(12.028,3.758)--(12.083,3.786)--(12.138,3.813)%
    --(12.193,3.839)--(12.248,3.865)--(12.303,3.889)--(12.358,3.912)--(12.413,3.935)%
    --(12.468,3.956)--(12.523,3.977)--(12.578,3.997)--(12.633,4.016)--(12.688,4.034)%
    --(12.743,4.051)--(12.798,4.067)--(12.853,4.083)--(12.908,4.097)--(12.963,4.111)%
    --(13.018,4.124)--(13.073,4.136)--(13.127,4.148)--(13.182,4.159)--(13.237,4.169)%
    --(13.292,4.178)--(13.347,4.187)--(13.402,4.195)--(13.457,4.203)--(13.512,4.210)%
    --(13.567,4.216)--(13.622,4.222)--(13.677,4.227)--(13.732,4.231)--(13.787,4.236)%
    --(13.842,4.239)--(13.897,4.243)--(13.952,4.245)--(14.007,4.248)--(14.062,4.250)%
    --(14.117,4.252)--(14.172,4.253)--(14.227,4.255)--(14.282,4.255)--(14.337,4.256)%
    --(14.392,4.257)--(14.447,4.257)--(14.447,0.985)--cycle;
\gpcolor{rgb color={0.063,0.318,0.502}}
\gpsetlinetype{gp lt plot 0}
\gpsetlinewidth{4.00}
\draw[gp path] (9.004,0.985)--(9.059,1.048)--(9.114,1.111)--(9.169,1.174)--(9.224,1.237)%
  --(9.279,1.300)--(9.334,1.363)--(9.389,1.425)--(9.444,1.488)--(9.499,1.550)--(9.554,1.612)%
  --(9.609,1.673)--(9.664,1.734)--(9.719,1.795)--(9.774,1.856)--(9.829,1.916)--(9.884,1.976)%
  --(9.939,2.035)--(9.994,2.094)--(10.049,2.152)--(10.104,2.210)--(10.159,2.267)--(10.214,2.324)%
  --(10.269,2.380)--(10.324,2.435)--(10.378,2.490)--(10.433,2.544)--(10.488,2.598)--(10.543,2.650)%
  --(10.598,2.703)--(10.653,2.754)--(10.708,2.804)--(10.763,2.854)--(10.818,2.903)--(10.873,2.952)%
  --(10.928,2.999)--(10.983,3.045)--(11.038,3.091)--(11.093,3.136)--(11.148,3.180)--(11.203,3.223)%
  --(11.258,3.265)--(11.313,3.307)--(11.368,3.347)--(11.423,3.386)--(11.478,3.425)--(11.533,3.463)%
  --(11.588,3.499)--(11.643,3.535)--(11.698,3.570)--(11.753,3.603)--(11.808,3.636)--(11.863,3.668)%
  --(11.918,3.699)--(11.973,3.729)--(12.028,3.758)--(12.083,3.786)--(12.138,3.813)--(12.193,3.839)%
  --(12.248,3.865)--(12.303,3.889)--(12.358,3.912)--(12.413,3.935)--(12.468,3.956)--(12.523,3.977)%
  --(12.578,3.997)--(12.633,4.016)--(12.688,4.034)--(12.743,4.051)--(12.798,4.067)--(12.853,4.083)%
  --(12.908,4.097)--(12.963,4.111)--(13.018,4.124)--(13.073,4.136)--(13.127,4.148)--(13.182,4.159)%
  --(13.237,4.169)--(13.292,4.178)--(13.347,4.187)--(13.402,4.195)--(13.457,4.203)--(13.512,4.210)%
  --(13.567,4.216)--(13.622,4.222)--(13.677,4.227)--(13.732,4.231)--(13.787,4.236)--(13.842,4.239)%
  --(13.897,4.243)--(13.952,4.245)--(14.007,4.248)--(14.062,4.250)--(14.117,4.252)--(14.172,4.253)%
  --(14.227,4.255)--(14.282,4.255)--(14.337,4.256)--(14.392,4.257)--(14.447,4.257);
\gpcolor{color=gp lt color axes}
\gpsetlinetype{gp lt axes}
\gpsetlinewidth{1.00}
\draw[gp path] (9.004,0.985)--(14.447,0.985);
\gpcolor{color=gp lt color border}
\gpsetlinetype{gp lt border}
\draw[gp path] (9.004,0.985)--(9.184,0.985);
\draw[gp path] (14.447,0.985)--(14.267,0.985);
\node[gp node right] at (8.820,0.985) { 0};
\gpcolor{color=gp lt color axes}
\gpsetlinetype{gp lt axes}
\draw[gp path] (9.004,1.506)--(14.447,1.506);
\gpcolor{color=gp lt color border}
\gpsetlinetype{gp lt border}
\draw[gp path] (9.004,1.506)--(9.184,1.506);
\draw[gp path] (14.447,1.506)--(14.267,1.506);
\node[gp node right] at (8.820,1.506) { 0.5};
\gpcolor{color=gp lt color axes}
\gpsetlinetype{gp lt axes}
\draw[gp path] (9.004,2.027)--(14.447,2.027);
\gpcolor{color=gp lt color border}
\gpsetlinetype{gp lt border}
\draw[gp path] (9.004,2.027)--(9.184,2.027);
\draw[gp path] (14.447,2.027)--(14.267,2.027);
\node[gp node right] at (8.820,2.027) { 1};
\gpcolor{color=gp lt color axes}
\gpsetlinetype{gp lt axes}
\draw[gp path] (9.004,2.548)--(14.447,2.548);
\gpcolor{color=gp lt color border}
\gpsetlinetype{gp lt border}
\draw[gp path] (9.004,2.548)--(9.184,2.548);
\draw[gp path] (14.447,2.548)--(14.267,2.548);
\node[gp node right] at (8.820,2.548) { 1.5};
\gpcolor{color=gp lt color axes}
\gpsetlinetype{gp lt axes}
\draw[gp path] (9.004,3.068)--(14.447,3.068);
\gpcolor{color=gp lt color border}
\gpsetlinetype{gp lt border}
\draw[gp path] (9.004,3.068)--(9.184,3.068);
\draw[gp path] (14.447,3.068)--(14.267,3.068);
\node[gp node right] at (8.820,3.068) { 2};
\gpcolor{color=gp lt color axes}
\gpsetlinetype{gp lt axes}
\draw[gp path] (9.004,3.589)--(14.447,3.589);
\gpcolor{color=gp lt color border}
\gpsetlinetype{gp lt border}
\draw[gp path] (9.004,3.589)--(9.184,3.589);
\draw[gp path] (14.447,3.589)--(14.267,3.589);
\node[gp node right] at (8.820,3.589) { 2.5};
\gpcolor{color=gp lt color axes}
\gpsetlinetype{gp lt axes}
\draw[gp path] (9.004,4.110)--(14.447,4.110);
\gpcolor{color=gp lt color border}
\gpsetlinetype{gp lt border}
\draw[gp path] (9.004,4.110)--(9.184,4.110);
\draw[gp path] (14.447,4.110)--(14.267,4.110);
\node[gp node right] at (8.820,4.110) { 3};
\gpcolor{color=gp lt color axes}
\gpsetlinetype{gp lt axes}
\draw[gp path] (9.004,4.631)--(14.447,4.631);
\gpcolor{color=gp lt color border}
\gpsetlinetype{gp lt border}
\draw[gp path] (9.004,4.631)--(9.184,4.631);
\draw[gp path] (14.447,4.631)--(14.267,4.631);
\node[gp node right] at (8.820,4.631) { 3.5};
\gpcolor{color=gp lt color axes}
\gpsetlinetype{gp lt axes}
\draw[gp path] (9.004,0.985)--(9.004,4.631);
\gpcolor{color=gp lt color border}
\gpsetlinetype{gp lt border}
\draw[gp path] (9.004,0.985)--(9.004,1.165);
\draw[gp path] (9.004,4.631)--(9.004,4.451);
\node[gp node center] at (9.004,0.677) { 0};
\gpcolor{color=gp lt color axes}
\gpsetlinetype{gp lt axes}
\draw[gp path] (9.911,0.985)--(9.911,4.631);
\gpcolor{color=gp lt color border}
\gpsetlinetype{gp lt border}
\draw[gp path] (9.911,0.985)--(9.911,1.165);
\draw[gp path] (9.911,4.631)--(9.911,4.451);
\node[gp node center] at (9.911,0.677) { 0.5};
\gpcolor{color=gp lt color axes}
\gpsetlinetype{gp lt axes}
\draw[gp path] (10.818,0.985)--(10.818,4.631);
\gpcolor{color=gp lt color border}
\gpsetlinetype{gp lt border}
\draw[gp path] (10.818,0.985)--(10.818,1.165);
\draw[gp path] (10.818,4.631)--(10.818,4.451);
\node[gp node center] at (10.818,0.677) { 1};
\gpcolor{color=gp lt color axes}
\gpsetlinetype{gp lt axes}
\draw[gp path] (11.726,0.985)--(11.726,4.631);
\gpcolor{color=gp lt color border}
\gpsetlinetype{gp lt border}
\draw[gp path] (11.726,0.985)--(11.726,1.165);
\draw[gp path] (11.726,4.631)--(11.726,4.451);
\node[gp node center] at (11.726,0.677) { 1.5};
\gpcolor{color=gp lt color axes}
\gpsetlinetype{gp lt axes}
\draw[gp path] (12.633,0.985)--(12.633,4.631);
\gpcolor{color=gp lt color border}
\gpsetlinetype{gp lt border}
\draw[gp path] (12.633,0.985)--(12.633,1.165);
\draw[gp path] (12.633,4.631)--(12.633,4.451);
\node[gp node center] at (12.633,0.677) { 2};
\gpcolor{color=gp lt color axes}
\gpsetlinetype{gp lt axes}
\draw[gp path] (13.540,0.985)--(13.540,4.451)--(13.540,4.631);
\gpcolor{color=gp lt color border}
\gpsetlinetype{gp lt border}
\draw[gp path] (13.540,0.985)--(13.540,1.165);
\draw[gp path] (13.540,4.631)--(13.540,4.451);
\node[gp node center] at (13.540,0.677) { 2.5};
\gpcolor{color=gp lt color axes}
\gpsetlinetype{gp lt axes}
\draw[gp path] (14.447,0.985)--(14.447,4.631);
\gpcolor{color=gp lt color border}
\gpsetlinetype{gp lt border}
\draw[gp path] (14.447,0.985)--(14.447,1.165);
\draw[gp path] (14.447,4.631)--(14.447,4.451);
\node[gp node center] at (14.447,0.677) { 3};
\draw[gp path] (9.004,4.631)--(9.004,0.985)--(14.447,0.985)--(14.447,4.631)--cycle;
%% coordinates of the plot area
\gpdefrectangularnode{gp plot 2}{\pgfpoint{9.004cm}{0.985cm}}{\pgfpoint{14.447cm}{4.631cm}}
\end{tikzpicture}
%% gnuplot variables

\end{center}
\caption{\label{fig:arbeitsdiagramm} Arbeitsdiagramm für eine ideale Feder und für eine reale Feder.
}
\end{figure}

In Abbildung~\ref{fig:arbeitsdiagramm} sehen wir zwei verschiedene Arbeitsdiagramme. 
Das erste Arbeitsdiagramm zeigt die Arbeit zum spannen einer idealen Feder.
Die Federkraft ist ein Beispiel für eine nicht konstante Kraft. 
%Wie wir schon auf Seite \pageref{hook} gesehen haben, 
Wie wir schon früher einmal gesehen haben,
gilt für die Federkraft $\RI{F}{F} = -D\cdot l$. Dabei ist $D$ die Federkonstante
und $l$ ist die Auslenkung aus der Ruhelage der Feder.
Die Federkraft nimmt linear mit der Auslenkung der Feder zu. Das heisst, am Anfang ist relativ wenig
Arbeit nötig um die Feder auszulenken. Die Arbeit, die nötig ist, um die Feder auszulenken ist
die Fläche unter dem Arbeitsdiagramm. Im Fall der idealen Feder, können wir eine Formel dafür angeben.


\begin{eqnarray*}
	\RI{W}{Feder}= \frac{1}{2}\cdot\RI{F}{F}\cdot l=\frac{1}{2}\cdot D\cdot l^2
\end{eqnarray*}

Um die Arbeit für das zweite Diagramm zu bestimmen gibt es keine Formel. Es gilt aber immer, dass die Fläche
im Arbeitsdiagramm die verrichtete Arbeit repräsentiert. Die Fläche im zweiten Diagramm lässt sich näherungsweise
durch auszählen der Kästchen bestimmen.

Die Energie, die in einer gespannten Feder gespeichert ist kann wie folgt berechnet werden:

\begin{cbox}
\begin{eqnarray*}
	\RI{E}{Feder}=\frac{1}{2}\cdot D\cdot l^2
\end{eqnarray*}
\end{cbox}

\newpage

\Einbinden{\dir/arbeit_feder01.tex}
\Einbinden{\dir/arbeit_feder02.tex}
\Einbinden{\dir/arbeit_feder03.tex}





\newpage

\section*{Energieerhaltung}
Die Energie ist in der Physik eine der wichtigsten Grössen. Die Energie ist eine \emph{Erhaltungsgrösse}.
Das heisst Energie lässt sich weder erzeugen noch vernichten.
Wenn man das mit den eigenen Erfahrungen vergleicht, klingt das auf den ersten Blick falsch.
Man liest oft über Energieerzeugung durch z. B. Kohlekraftwerke. Aber diese Energie wird nicht erzeugt,
sondern nur umgewandelt. Wenn Kohle verbrannt wird, brechen chemische Bindungen, in denen Energie gespeichert
war. Kohle entsteht, wenn Pflanzen unter Druck kompostieren. Diese Pflanzen sind dank der Sonnenenergie gewachsen.
Auf der Sonne verschmelzen zwei Wasserstoffatome in ein Heliumatom, dadurch wird Energie freigesetzt.


Reale Systeme ``verlieren'' immer Energie durch Reibung. Auch in diesem Fall gilt die Energieerhaltung.
Kinetische Energie wird in \emph{innere Energie} $U$ umgewandelt. Das heisst praktisch, dass Energie
durch Reibung zu Wärme wird. Innere Energie lässt sich nicht mehr
wirkungsvoll in andere Energieformen umwandeln.

\begin{cbox}
\begin{equation*}
	E = \RI{E}{pot} +\RI{E}{kin} + \RI{E}{Feder} + U = \text{konstant}
\end{equation*}
\end{cbox}

\begin{aufgabe}
	Ein Stein (\SI{1.5}{kg}) fällt von einer \SI{40}{m} hohen Brücke.
	\begin{itemize}
		\item [a)] Wie hoch ist die potentielle Energie des Steins auf der Brücke?
		\item [b)] Wie hoch ist die kinetische Energie des Steins auf der Brücke?
		\item [c)] Wie gross ist seine Gesamtenergie?
		\item [d)] Mit welcher Geschwindigkeit kommt der Stein unten auf?
	\end{itemize}
	\begin{loesung}
		\begin{itemize}
			\item[a)]
				\begin{eqnarray*}
					\RI{E}{pot}=m\cdot g\cdot h=\SI{1.5}{kg}\cdot\SI{9.81}{m/s^2}\cdot\SI{40}{m}=\SI{588.6}{J}
				\end{eqnarray*}
			\item[b)]
				\begin{eqnarray*}
					\RI{E}{kin}=\frac{1}{2}\cdot m\cdot v^2=\SI{0.5}\cdot\SI{1.5}{kg}\cdot{0}{m/s}=\SI{0}{J}
				\end{eqnarray*}
			\item[c)] Die Gesamtenergie ist konstant.
				\begin{eqnarray*}
					\RI{E}{tot}=\RI{E}{pot} + \RI{E}{kin}=\SI{588.6}{J} + \SI{0}{J}=\SI{588.6}{J}
				\end{eqnarray*}
			\item[d)] Am Boden ist die Gesamtenergie genauso gross wie auf der Brücke (Energieerhaltung).
				\begin{gather*}
					\RI{E}{pot}=m\cdot g\cdot h=\SI{1.5}{kg}\cdot\SI{9.81}{m/s^2}\cdot\SI{0}{m}=\SI{0}{J}\\
					\RI{E}{tot}=\RI{E}{pot} + \RI{E}{kin} \to \RI{E}{kin}=\RI{E}{tot}-\RI{E}{pot}=\SI{588.6}{J}-\SI{0}{J}=\SI{588.6}{J}\\
					\RI{E}{kin}=\frac{1}{2}\cdot m\cdot v^2 \to v=\sqrt{\frac{2\cdot \RI{E}{kin}}{m}} =\SI{28.0}{m/s}
				\end{gather*}
		\end{itemize}
	\end{loesung}
\end{aufgabe}

\begin{aufgabe}
	Eine Feder mit einer Federkonstanten von \SI{200}{N/m} wird gestaucht.
	\begin{itemize}
		\item [a)] Wie viel Arbeit ist nötig um die Feder um \SI{15}{cm} zu stauchen?
		\item [b)] Wie viel Energie ist nun in der Feder gespeichert?
		\item [c)] Nun wird ein Schlitten ($m=\SI{1.7}{kg}$) vor die gespannte Feder gesetzt, und die Feder entspannt.
			Auf welche Geschwindigkeit wird der Schlitten beschleunigt, wenn Reibung vernachlässigt wird?
		\item [d)] Der Schlitten hat Stahlkufen und gleitet auf einer Stahloberfläche. Wie weit kommt der Schlitten, wenn
			die Reibung nach entspannen der Feder einsetzt?
	\end{itemize}
	\begin{loesung}
		\begin{itemize}
			\item [a)] Die Federkraft ist nicht konstant, sondern steigt linear mit der Auslenkung.
				Die Fläche unter dem Arbeitsdiagramm ist 
				\begin{eqnarray*}
				W=\frac{1}{2}\cdot\RI{F}{F}\cdot(\Delta x)=\frac{1}{2}\cdot D\cdot (\Delta x)^2=\num{0.5}\cdot\SI{200}{N/m}\cdot(\SI{0.15}{m})^2=\SI{2.25}{J}
				\end{eqnarray*}
			\item[b)] Das spannen der Feder hat \SI{2.25}{J} gekostet, damit ist die Federenergie $\RI{E}{Feder}=\SI{2.25}{J}$.
			\item[c)] Es gilt Energieerhaltung. Die Federenergie wird vollständig in kinetische Energie umgewandelt.
				\begin{eqnarray*}
					\RI{E}{kin}=\frac{1}{2}\cdot m\cdot v^2 \to v=\sqrt{\frac{2\cdot \RI{E}{kin}}{m}} =\sqrt{\frac{2\cdot\SI{2.25}{J}}{\SI{1.7}{kg}}}=\SI[dp=2]{1.6270}{m/s}
				\end{eqnarray*}
			\item[d)] Wir rechnen zuerst die Reibungskraft aus. Aus der Tabelle finden wir die Reibungszahl $\mu$ für Gleitreibung Stahl auf Stahl.
				Es wirkt nur die Gewichtskraft und die Normalkraft in vertikaler Richtung auf den Schlitten, 
				daraus folgt, dass die Normalkraft gleich
				der Gewichtskraft ist.
				\begin{eqnarray*}
					\RI{F}{R}=\mu\cdot\RI{F}{N}=\mu\cdot m\cdot g=\num{0.1}\cdot\SI{1.7}{kg}\cdot\SI{9.81}{m/s^2}=\SI{1.67}{N}
				\end{eqnarray*}
	Es gibt nun zwei Möglichkeiten um zu berechnen, wie weit der Schlitten noch kommt.
	\begin{itemize}
		\item [(1)] Im ersten Fall berechnen wir die Arbeit, die die Oberfläche gegen den Schlitten verrichtet.
			Wenn die kinetische Energie des Wagens vollständig in innere Energie $U$ umgewandelt wurde, kommt dieser zum Stehen.
			\begin{eqnarray*}
				W=F\cdot s =\RI{F}{R}\cdot s \to s =\frac{W}{\RI{F}{R}}=\frac{\SI{2.25}{J}}{\SI{1.67}{N}}=\SI[dp=2]{1.3473}{m}
			\end{eqnarray*}
		\item[(2)]			
				Mit $F=m\cdot a$ kann man nun die Beschleunigung ausrechnen. \RI{F}{R} wirkt entgegen der Bewegungsrichtung, also $-\RI{F}{R}$.
\begin{gather*}
	a=\frac{F}{m} =\frac{\RI{F}{R}}{m}=\frac{\SI{-1.67}{N}}{\SI{1.7}{kg}}=\SI{-0.981}{m/s^2}\\
	v^2=v_0^2+2\cdot a\cdot s \to s=\frac{v^2-v_0^2}{2\cdot a}=\frac{(\SI{0}{m/s})^2-(\SI[dp=2]{1.6270}{m/s})^2}{2\cdot(\SI{-0.981}{m/s^2})}=\SI[dp=2]{1.3492}{m}
\end{gather*}

	\end{itemize}
		\end{itemize}
	\end{loesung}

\end{aufgabe}



\begin{aufgabe}
	Ein Wagen (\SI{3}{kg}) rollt eine schiefe Ebene herunter. Die Ebene ist mit einem Winkel von \SI{25}{\degree}
	gegen die Horizontale geneigt. 
	\begin{itemize}
		\item [a)] Wie schnell ist der Wagen nach \SI{7.5}{m}?
		\item [b)] Nach \SI{15}{m} fährt der Wagen auf eine Feder mit einer Federkonstante von \SI{100}{N/m}
			wie stark wird die Feder gestaucht?
		\item [c)] Der Wagen wird auf eine horizontale Ebene mit Reibung ($\mu=0.1$) umgelenkt.
			Wie weit kommt der Wagen?
	\end{itemize}
	\begin{loesung}
		\begin{itemize}
			\item [a)] Der Wagen wandelt potentielle Energie in kinetische Energie.
				\begin{eqnarray*}
					\RI{E}{pot}= m\cdot g\cdot h=m\cdot g\cdot\sin(\alpha)\cdot \Delta s =\SI{3}{kg}\cdot\SI{9.81}{m/s^2}\cdot\num[dp=2]{0.42262}\cdot\SI{7.5}{m}=\SI{93.282}{J}
				\end{eqnarray*}
Diese potentielle Energie hat der Wagen verloren. Gleichzeitig hat er denselben Betrag an kinetischer Energie gewonnen (Energieerhaltung).
Durch umstellen der kinetischen Energie nach $v$ erhalten wir
\begin{eqnarray*}
	\RI{E}{kin}=\frac{1}{2}\cdot m\cdot v^2 \to v=\sqrt{\frac{2\cdot \RI{E}{kin}}{m}}=\sqrt{\frac{2\cdot\SI{93.28}{J}}{\SI{3}{kg}}}=\SI[dp=2]{7.8860}{m/s}
\end{eqnarray*}

\item[b)] Nun wird die kinetische Energie in Federenergie umgewandelt.
Da die kinetische Energie am Startpunkt Null war, ist die kinetische Energie unten, gleich der potentiellen Energie am Startpunkt. 
	\begin{eqnarray*}
		\RI{E}{tot}=\RI{E}{pot}=m\cdot g\cdot h=m\cdot g\cdot\sin(\SI{25}{\degree})\cdot\SI{15}{m}=\SI{3}{kg}\cdot\SI{9.81}{m/s^2}\cdot\num[dp=2]{0.42262}\cdot\SI{15}{m}=\SI{186.56}{J}
	\end{eqnarray*}
Aus der Federenergie kann die Auslenkung $\Delta x$ bestimmt werden
\begin{eqnarray*}
	\RI{E}{Feder}=\frac{1}{2}\cdot D\cdot(\Delta x)^2 \to \Delta x = \sqrt{\frac{2\cdot\RI{E}{Feder}}{D}}=\sqrt{\frac{2\cdot\SI{186.56}{J}}{\SI{100}{N/m}}}=\sqrt{\SI[dp=2]{3.7313}{m^2}}=\SI[dp=2]{1.9317}{m}
\end{eqnarray*}

\item [c)] Zuerst berechnen wir die Reibungskraft. Die Gewichtskraft und die Normalkraft sind die einzigen Kräfte in vertikale Richtung.
	Daher müssen beide gleich gross sein.
	\begin{eqnarray*}
		\RI{F}{R} = \mu\cdot\RI{R}{N} = \mu\cdot m\cdot g =\num{0.1}\cdot\SI{3}{m}\cdot\SI{9.81}{m/s^2}=\SI[dp=2]{2.9430}{N}
	\end{eqnarray*}
	Nun gibt es zwei Möglichkeiten auszurechen, wie weit der Schlitten noch fährt.
	\begin{itemize}
		\item [(1)] Wir wandeln die kinetische Energie in innere Energie $U$ um.
			\begin{eqnarray*}
				U = \RI{F}{R}\cdot s \to s=\frac{U}{\RI{F}{R}}=\frac{\SI{186.56}{J}}{\SI[dp=2]{2.9430}{N}} =\SI[dp=2]{63.393}{m}
			\end{eqnarray*}
		\item [(2)] Wir berechnen aus der Reibungskraft die Beschleunigung und dann damit den Bremsweg.
			\begin{eqnarray*}
				F=m\cdot a \to a=\frac{F}{m}=\frac{\SI[dp=2]{2.9430}{N}}{\SI{3}{kg}}=\SI[dp=2]{0.98100}{m/s^2}
			\end{eqnarray*}
Aus der Gesamtenergie bestimmen wir die Geschwindikeit
\begin{eqnarray*}
	\RI{E}{kin}=\frac{1}{2}\cdot m\cdot v^2 \to v=\sqrt{\frac{2\cdot \RI{E}{kin}}{m}}=\sqrt{\frac{2\cdot\SI{186.56}{J}}{\SI{3}{kg}}}=\SI[dp=2]{11.152}{m/s}
\end{eqnarray*}
Damit bekommen wir
\begin{eqnarray*}
	v^2 = v_0^2 +2\cdot a\cdot s \to s=\frac{v^2 -v_0^2}{2\cdot a}=\frac{(\SI{0}{m/s})^2-(\SI[dp=2]{11.152}{m/s})^2}{2\cdot(\SI[dp=2]{-0.98100}{m/s^2})}=\frac{\SI[dp=2]{124.38}{m^2/s^2}}{\SI[dp=2]{1.9620}{m/s^2}}=\SI[dp=2]{63.393}{m}
\end{eqnarray*}
	\end{itemize}
		\end{itemize}
	\end{loesung}

\end{aufgabe}

