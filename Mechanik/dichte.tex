\section{Die Dichte}

\begin{table}
	\centering
	\begin{tabular}{c r | c r | c r}
		Gold                             & 19290  & 	Sandstein     & 2400 & Ethanol  & 789\\
		Quecksilber                      & 13546  & 	Glas          &2500  & Diesel   & 830\\
		Aluminium                        &  2700  & 	Diamant       & 3510 & Olivenöl & 910\\
		Wasser (\SI{0}{\degreeCelsius})  &  1000  & 	Silber        & 10490& Meerwasser& 1025 \\
        Eis (\SI{0}{\degreeCelsius})     &   917  & 	Uran          & 19050& Milch    & 1030\\
		Holz (Kiefer)                    &   520  & 	Platin        & 21450& Helium (\SI{0}{\degreeCelsius})   & \num{0,1785}\\
		Luft                             &  1.2041& 	Blei          & 11340& Wasserstoff (\SI{0}{\degreeCelsius}) & \num{0,0899}\\
	\end{tabular}
	\caption{Dichte verschiedener Materialien in (\si{kg}/\si{m}$^3$).}
	\label{tab:dichte}
\end{table}


Die Dichte ($\rho$) ist das Verhältnis zwischen Masse (m) und Volumen (V).


\begin{cbox}
\begin{equation*}
	\rho = \frac{\text{Masse}}{\text{Volumen}} = \frac{m}{V}\text{,}\quad\text{Einheit:} [\rho]=\frac{\text{kg}}{\text{m}^3} 
\end{equation*}
\end{cbox}

Die Dichte ist eine Materialkonstante und kann zur Unterscheidung verschiedener Materialien
verwendet werden. Sie ist unabhängig von Form und Grösse des Gegenstands. 
Die Dichte von Festkörpern ist grösser als die Dichte von Gasen.
In Tabelle \ref{tab:dichte} sind die Dichten einiger Materialien angegeben.

\Einbinden{\dir/dichte01.tex}
\Einbinden{\dir/dichte02.tex}
\Einbinden{\dir/dichte03.tex}
\Einbinden{\dir/dichte04.tex}
