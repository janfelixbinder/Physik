\section*{Der Impuls}
Der Impuls $\vec{p}$ eines Teilchens ist definiert als das Produkt aus seiner Masse und seiner Geschwindigkeit:
\begin{cbox}
\begin{gather*}
	\text{Impuls} = \text{Produkt von Masse und Geschwindigkeit} \quad\text{oder}\quad \vec{p} = m\cdot\vec{v}\\
	\text{Einheit}: [\vec{p}]=\text{Kilogramm}\cdot\nicefrac{\text{Meter}}{\text{Sekunde}}=\si{kg}\cdot\si{m/s}=\si{N s}
\end{gather*}
\end{cbox}

\Einbinden{\dir/impuls01.tex}

Wirken keine äusseren Kräfte auf ein System, so bleibt der  Gesamtimpuls erhalten. 
Schauen Sie sich dazu noch einmal Aufgabe \ref{skateboard} an.
Nehmen wir an, der Ball habe eine Masse von \SI{2}{kg}, Mensch und Skateboard zusammen wiegen \SI{50}{kg}.
Reibung soll vernachlässigt werden.
Der Gesamtimpuls vor dem Werfen ist Null, da das Skateboard in Ruhe ist.
Nun wird der Ball beschleunigt und erhält eine Geschwindigkeit von \SI{10}{m/s}.
Der Impuls des Balls ist damit
\begin{eqnarray*}
	p=m\cdot v=\SI{2}{kg}\cdot\SI{10}{m/s}=\SI{20}{kg m/s}\text{.}
\end{eqnarray*}
Da der Gesamtimpuls Null bleibt, rollt das Skateboard mit Fahrer in die entgegengesetzte Richtung weg.
Die Geschwindigkeit ist: 
\begin{eqnarray*}
	p=m\cdot v\to v=\frac{p}{m}=\frac{\SI{20}{kg m/s}}{\SI{50}{kg}}=\SI{0.4}{m/s}\text{.}
\end{eqnarray*}
