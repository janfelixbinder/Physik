
\section*{Das Federgesetz}
Kräfte kann man nicht direkt beobachten. Eine Kraft muss auf einen Körper wirken, damit sie ``sichtbar'' wird.
Wirkt eine Kraft zum Beispiel auf eine Metallfeder, dann wird diese aus ihrer Ruhestellung ausgelenkt.
Wirkt die Kraft nicht mehr, nimmt die Feder wieder ihre ursprüngliche Form an. Die Verlängerung $l$ der Feder
unter Krafteinwirkung wird durch das \emph{Federgesetz} beschrieben.

\Einbinden{\dir/federgesetz01.tex}

\newcommand\leereZ{\phantom{x} & \phantom{x} & \phantom{x} \\\hline}

\subsection*{Experiment}
\begin{center}
%\begin{large}
	\begin{tabular}{p{0.3\textwidth}|p{0.3\textwidth}|p{0.3\textwidth}}
    Masse (kg) & Gewichtskraft (N) & Auslenkung (cm) \\\hline
	\leereZ
	\leereZ
	\leereZ
	\leereZ
	\leereZ
	\end{tabular}
%\end{large}
\end{center}

\begin{center}
\begin{tikzpicture}
    \mmPapier{(0,0)}{(15.5,7)}{0.5}{0.5}
\end{tikzpicture}
\end{center}

\newpage
\begin{cbox}
	\begin{center}
		\bf{Federgesetz}
	\end{center}
\begin{gather*}
	\text{Kraft} \phantom{=\text{Federkonstante}\cdot\text{Verlängerung}}\quad\text{oder}\quad F=\phantom{-D\cdot\Delta x}\\
	\text{Einheit}: [D] = \phantom{\frac{\text{Newton}}{\text{Meter}}=\frac{\si{N}}{\si{m}}}
\end{gather*}
\end{cbox}


\Einbinden{\dir/federgesetz02.tex}
\Einbinden{\dir/federgesetz03.tex}

