
\section*{Addition von Kräften}

In einem Knoten laufen drei Fäden zusammen. An dem jeweils anderen Ende der Fäden ist eine Masse angehängt.
Zwei der drei Fäden werden durch Rollen umgelenkt.
Wird der Knoten aus seiner Ruhelage (Gleichgewichtslage) ausgelenkt, so pendelt sich das System wieder in diese Lage zurück.
Die Rollen lenken die Fadenkräfte um, ändern aber nicht ihren Betrag.
In der Ruhelage ist die Summe aller angreifenden Kräfte Null.

\Einbinden{\dir/vektoren01.tex}
\Einbinden{\dir/vektoren02.tex}
\Einbinden{\dir/vektoren03.tex}
\Einbinden{\dir/vektoren04.tex}


\newpage
\section*{Zerlegung von Kräften}

Vektoren lassen sich nicht nur addieren, sondern man kann sie auch in Komponenten zerlegen. Das ist im Prinzip die Umkehrung
der Vektoraddition.

\Einbinden{\dir/vektoren05.tex}
\Einbinden{\dir/vektoren06.tex}


Ist die Richtung eines Vektors vorgegeben, lässt sich der Vektor noch nicht eindeutig zerlegen. Es gibt immer
noch viele verschiedene Möglichkeiten den Vektor zu zerlegen.

\newpage


Sind die Richtungen von zwei Vektoren gegeben (in 2D), ist die Zerlegung eindeutig möglich. Allgemein gilt, dass man
für jede Raumdimension eine eindeutige Richtung benötigt.

\Einbinden{\dir/vektoren07.tex}


In der Praxis zerlegt man Vektoren oft in Kartesische Koordinaten, also in eine $x$-, eine $y$- und im Fall von drei Dimensionen in eine $z$-Koordinate.

\Einbinden{\dir/vektoren08.tex}

\newpage

\Einbinden{\dir/vektoren09.tex}

\newpage
\Einbinden{\dir/vektoren10.tex}

\Einbinden{\dir/vektoren11.tex}

