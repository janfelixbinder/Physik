\section*{Leistung $P$}

Die \emph{Leistung} gibt den Energieverbrauch pro Zeiteinheit an. Das Formelzeichen 
für die Leistung ist $P$ (engl. power). Die Grundeinheit der Leistung ist das {\bf Watt}.
Die Einheit ist zu Ehren von James Watt (1736 - 1819), dem Erfinder der ersten Dampfmaschine benannt.
An nahezu allen technischen Geräten ist angegeben, wie viel Watt sie verbrauchen.
Im Sprachgebrauch wird die Leistung von Autos oft in PS angegeben (\(\SI{1}{PS}= \SI{735.5}{W}\)).


\begin{cbox}
\begin{gather*}
	\text{Leistung} = \frac{\text{verrichtete Arbeit}}{\text{erforderliche Zeit}}\quad\text{oder}\quad P=\frac{\Delta W}{\Delta t}\\
	\text{Einheit}: [P] = \frac{\text{Joule}}{\text{Sekunde}}=\frac{\si{J}}{\si{s}}=\frac{\si{kg}\cdot\si{m^2}}{\si{s^2}}\cdot \frac{1}{\si{s}}=\text{Watt}=\si{W}
\end{gather*}
\end{cbox}

\Einbinden{\dir/leistung01.tex}
\Einbinden{\dir/leistung02.tex}
\Einbinden{\dir/leistung03.tex}


\newpage

\Einbinden{\dir/leistung04.tex}
\Einbinden{\dir/leistung05.tex}



\Einbinden{\dir/energieerhaltung_kugelbahn.tex}
\newpage


\Einbinden{\dir/leistung_velo.tex}

