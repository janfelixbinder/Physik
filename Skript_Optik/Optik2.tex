%\documentclass[11pt,a4paper,titlepage,twoside]{article}
\documentclass[12pt,a4paper,twoside]{article}
\usepackage{mystyle}
\usepackage{optik_v001}
\usepackage{gplot}
%\usepackage{schueler}
%\usepackage{lehrer}
\date{}
%\author{Felix Binder}
\title{Optik}

%\usepackage{enumerate}

\newcommand{\Obox}{\draw (0,0) rectangle (\textwidth,0.7\textwidth);}




\begin{document}
%\maketitle

\addtocounter{section}{7}
\addtocounter{aufgabe}{23}


\section{Linsen}

Optische Linsen sind aus lichtdurlässigen Materialien, wie Glas oder Plastik, manchmal auch aus durchsichtigen Kristallen.
Einfache (sphärische) Linsen kann man sich aus einer Kugel geschnitten vorstellen. Der Kugelradius kann dabei verschieden gross sein.
Beim Durchlauf des Lichtstrahl durch die Linse, passiert er zwei Grenzflächen, an denen er nach dem Brechungsgesetz gebrochen wird.

\begin{aufgabe}
Zeichnen Sie den vollständigen Strahlengang für mindestens zwei parallel einlaufende Lichtstrahlen.
Die optische Dichte der Linse ist \num{1.5}.
\end{aufgabe}

\begin{center}
\begin{tikzpicture}
	\draw [mmPapier, xstep=0.5, ystep=0.5, shift={(-7,-5)}](0,0) grid (14,10);
	\def\RI{8}
	\def\RII{7}
	\Linse{(0,0)}{\RI}{\RII}{4}{0}{Lin}{Glas}

	%Mittelpunkte
	\draw [fill] (Lin Cl) circle(0.1cm) node [below] {C};
	\draw [dotted] (Lin POSr) arc (0:90:\RI);
	\draw [dotted] (Lin POSr) arc (0:-90:\RI);
	\draw [LS,->] (Lin Cl) --+(65:\RI) node [above, sloped, midway] {$R_1$};
	
	\draw [fill] (Lin Cr) circle(0.1cm) node [below] {C};
	\draw [dotted] (Lin POSl) arc (180:90:\RII);
	\draw [dotted] (Lin POSl) arc (180:270:\RII);
	\draw [LS,->] (Lin Cr) --+(125:\RII) node [above, sloped, midway] {$R_2$};

	
	\draw [dotted] (Lin POSo) -- (Lin POSu);
	\draw [dotted] (-7,0)--(7,0);
\end{tikzpicture}
\end{center}


Linsen werden in sehr vielen optischen Geräten verwendet. Um deren Funktionsweise verstehen zu können, ist es meistens nicht
nötig den genauen Stahlengang eines Lichtstrahl durch die Linse zu kennen. Für technische Anwendungen, und auch für unseren weiteren
Unterricht, läuft ein parallel zur optischen Achse einlaufender Lichtstrahl, bis zur \emph{Mittelebene} der Linse, und wird erst dort gebrochen.

\begin{aufgabe}
	Zeichnen Sie den Strahlengang für einen parallel zur optischen Achse einlaufenden Lichtstrahl, für einen Lichtstrahl, der durch
	den Brennpunkt auf die Linse fällt und für einen Lichtstrahl, der durch den Mittelpunkt der Linse verläuft.
\end{aufgabe}

\begin{center}
\begin{tikzpicture}
%	\draw [mmPapier, xstep=0.5, ystep=0.5, shift={(-7,-5)}](0,0) grid (14,10);
	\def\RI{12}
	\def\RII{12}
	\Linse{(0,0)}{\RI}{\RII}{4}{0}{Lin}{Glas}

	%Mittelpunkte
	\draw [fill] (-5,0) circle(0.1cm) node [below] {F};
	
	\draw [fill] (5,0) circle(0.1cm) node [below] {F};

	
	\draw [dotted] (Lin POSo) -- (Lin POSu);
	\draw [dotted] (-7,0)--(7,0);
\end{tikzpicture}
\end{center}

\begin{aufgabe}
	Zeichnen Sie den Strahlengang für parallel zur optischen Achse einlaufende Lichtstrahlen ein.
\end{aufgabe}

\begin{center}
\begin{tikzpicture}
%	\draw [mmPapier, xstep=0.5, ystep=0.5, shift={(-7,-5)}](0,0) grid (14,10);
	\def\RI{-12}
	\def\RII{-12}
	\Linse{(0,0)}{\RI}{\RII}{4}{2}{Lin}{Glas}

	%Mittelpunkte
	\draw [fill] (-5,0) circle(0.1cm) node [below] {F};
	
	\draw [fill] (5,0) circle(0.1cm) node [below] {F};

	
	\draw [dotted] (Lin POSo) -- (Lin POSu);
	\draw [dotted] (-7,0)--(7,0);
\end{tikzpicture}
\end{center}


%\newpage
%\includesolutions





\end{document}
