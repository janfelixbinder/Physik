\subsection{Abbildungen mit gewölbten Spiegeln}
\begin{aufgabe}
	Betrachten Sie sich in einem Hohlspiegel. Was sehen Sie? Ist das Bild abhängig vom Abstand zum Spiegel?
\end{aufgabe}

%Objekt ausserhalb des Kreispittelpunktes
\begin{center}
	\begin{tikzpicture}
		\AbbHS{IO3}
	\end{tikzpicture}
\end{center}

%Objekt im Mittelpunkt des Kreises
\begin{center}
	\begin{tikzpicture}
		\AbbHS{Spiegel C}
	\end{tikzpicture}
\end{center}

%Objekt zwischen Mittelpunkt und Brennpunkt
\begin{center}
	\begin{tikzpicture}
		\path (Spiegel C) -- (Spiegel F) node [pos=0.5, shape=coordinate] (M1) {};
		\AbbHS{M1}
	\end{tikzpicture}
\end{center}

%Objekt im Brennpunkt
\begin{center}
	\begin{tikzpicture}
		\AbbHS{Spiegel F}
	\end{tikzpicture}
\end{center}

%Objekt zwischen Brennpunkt und Spiegel
\begin{center}
	\begin{tikzpicture}
		\path (Spiegel F) -- (IO) node [pos=0.5, shape=coordinate] (M1) {};
		\AbbHS{M1}
	\end{tikzpicture}
\end{center}



\begin{loesung}

%Objekt ausserhalb des Kreispittelpunktes
\begin{center}
	\begin{tikzpicture}
		\AbbHSLoesung{IO3}
	\end{tikzpicture}
\end{center}

%Objekt im Mittelpunkt des Kreises
\begin{center}
	\begin{tikzpicture}
		\AbbHSLoesung{Spiegel C}
	\end{tikzpicture}
\end{center}

%Objekt zwischen Mittelpunkt und Brennpunkt
\begin{center}
	\begin{tikzpicture}
		\path (Spiegel C) -- (Spiegel F) node [pos=0.5, shape=coordinate] (M1) {};
		\AbbHSLoesung{M1}
	\end{tikzpicture}
\end{center}

%Objekt im Brennpunkt
\begin{center}
	\begin{tikzpicture}
		\AbbHS{Spiegel F}
		\AbbHSLoesung{Spiegel F}
	\end{tikzpicture}
\end{center}

%Objekt zwischen Brennpunkt und Spiegel
\begin{center}
	\begin{tikzpicture}
		\path (Spiegel F) -- (IO) node [pos=0.5, shape=coordinate] (M1) {};
		\AbbHSLoesung{M1}
	\end{tikzpicture}
\end{center}


\end{loesung}

%\begin{center}
%\begin{tikzpicture}
%\NObox
%
%\HSpiegel{IO}{Spiegel}
%
%\draw [OAchse, name path=OAchse] (IO) -- (IW);
%
%\draw [OObjekt, ->] (IO3) --++(90:2cm) node [shape=coordinate] (Spitze) {};
%\path [OObjekt, ->] (IO3) --++(-90:2cm) node [shape=coordinate] (MSpitze) {};%der Gegenstand an der optischen Achse gespiegelt
%
%%Loesung
%%Parallelstrahl wird Brennstrahl
%\path [name path = IParallelstrahl] (Spitze) --++(20,0);
%\path [name intersections={of=IParallelstrahl and Spiegel, by={IPS}}];
%\draw [color=red] (Spitze)--(IPS)--($(IPS)!3!(Spiegel F)$);
%\path [name path = OBrennstrahl] (IPS)--($(IPS)!3!(Spiegel F)$);
%
%
%%Brennstrahl wird Parallelstrahl
%\path [name path = IBrennstrahl] (Spitze) --($(Spitze)!10!(Spiegel F)$);
%\path [name intersections={of=IBrennstrahl and Spiegel, by={IBS}}];%Intersection Brennstrahl Spiegel IBS
%\draw [color=red] (Spitze)--(IBS)--++(-10,0);
%
%%Mittelpunktstrahl
%\draw [color=red] (Spitze)--(IO)--($(IO)!1.2!(MSpitze)$);
%
%\path [name path = OMittelpunktstrahl] (IO)--($(IO)!1.2!(MSpitze)$);
%\path [name intersections={of=OBrennstrahl and OMittelpunktstrahl, by={SBild}}];
%\path [name path = Bild] (SBild)--++(0,10);
%\path [name intersections={of=Bild and OAchse, by={UBild}}];
%\draw [OObjekt, ->] (UBild)--(SBild) node [midway, right] {B};
%
%\end{tikzpicture}
%\end{center}
%
