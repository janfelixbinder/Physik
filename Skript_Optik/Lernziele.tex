%\documentclass[11pt,a4paper,titlepage,twoside]{article}
\documentclass[12pt,a4paper,twoside]{article}
\usepackage{mystyle}
\usepackage{foto_v001}
\usepackage{gplot}
\usepackage{tabelle_v001}
\usepackage{mechanik_v001}
%\usepackage{import}

\usepackage{url}

%oben und unten
\usepackage{fancyhdr}
\pagestyle{fancy}
\lhead{}
\rhead{}
\rfoot{Felix Binder}
\lfoot{Hydrostatik Mai 2014}
\renewcommand\headrulewidth{0pt}
\renewcommand\footrulewidth{1pt}
%ende oben und unten

\date{}
%\author{Felix Binder}
\title{Lernziele Strahlenoptik}


\begin{document}
\maketitle

%\addtocounter{page}{5}
%\addtocounter{section}{9}
%\addtocounter{aufgabe}{27}

%\section*{Lernziele}

\begin{itemize}
	\item Sie kennen die drei Modelle für Lichtquellen und können mit diesen Schattenwürfe konstruieren.

	\item Sie kennen die Begriffe Gegenstandsweite $g$ und Bildweite $b$ sowie Gegenstandsgrösse $G$ und Bildgrosse $B$ 
		im Zusammenhang mit dem Schattenwurf und können Aufgaben dazu lösen.

	\item Sie können das Prinzip der Lochkamera erklären, den Strahlengang zeichnen und Rechnungen 
		mit Bildweite, Gegenstandsweite, Bildgrösse und Gegenstandsgrösse durchführen.
	\item Sie können erklären wie das Bild bei einer Lochkamera mit der Lochgrösse zusammenhängt.


	\item Sie können das Reflexionsgesetz erklären und sowohl an Planspiegeln als auch an Wölbspiegeln anwenden.
	\item Sie können erklären, was beim Auftreffen von Licht auf eine raue Oberfläche passiert.

	\item Sie können das Brechungsgesetz erklären und rechnerisch anwenden. 
		Ausserdem können Sie den Strahlengang beim Übergang von einem Material in ein anderes einzeichnen.

	\item Sie können erklären wie es zu reellen und virtuellen Bildern sowohl bei Linsen als auch bei Spiegeln (Planspiegel, Hohlspiegel, Wölbspiegel) kommt.
	\item Sie können die Strahlengänge (Parallelstrahl, Brennstrahl, Mittelpunktstrahl) einzeichnen und ein Bild konstruieren.

	\item Sie können erklären, wie das Brechungsgesetz mit der Ausbreitung von Licht durch Linsen zusammenhängt.
	\item Sie kennen die Linsengleichung und können Sie anwenden.

\end{itemize}

\end{document}
