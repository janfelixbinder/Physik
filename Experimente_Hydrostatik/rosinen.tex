\subsection*{Experiment I: Tanzende Rosinen}
Für diese Low-Cost Experiment benötigt man:
\begin{itemize}
	\item Rosinen (kleine Rosinen funktionieren besser, sollte die Rosinen nicht aufsteigen besteht die Möglichkeit sie in kleine Stücke zu zerteilen)
	\item Glasgefäss zum Beispiel Messbecher
	\item Wasser mit Kohlensäure
\end{itemize}
Die kleinen Rosinen werden in ein Glasgefäss gegeben und dieses mit Wasser mit Kohlensäure aufgefüllt.
Die Rosinen liegen anfangs am Boden. Man beobachtet, dass sich Gasbläschen an die Rosinen setzen.
Einige Rosinen beginnen aufzusteigen. An der Wasseroberfläche entweichen die Gasbläschen aus dem Wasser und die Rosinen sinken wieder zu Boden.
