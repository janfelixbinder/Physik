%\documentclass[11pt,a5paper,titlepage]{article}
\documentclass[12pt,a5paper]{article}
%\documentclass[12pt,a4paper,twoside]{scrartcl}
\usepackage{mystyle}
\usepackage{gplot}
\usepackage{mechanik_v001}
\usepackage{foto_v001}


\usepackage[shell]{gnuplottex}
\usepackage{pgf}
\usepackage{pgfplots}
\usepackage{gnuplot-lua-tikz}
\pgfplotsset{compat=1.9}




%\author{Felix Binder}
\title{Schreibauftrag}
\date{12.05.15}



\def\dir{./Aufgaben/}
\newcommand{\Einbinden}[1]{\input{#1}}


\begin{document}
\maketitle


Stellen Sie sich einen See im Spätsommer vor.
Das Seewasser hat eine Temperatur von konstant \SI{20}{\celsius}.
Die Tage werden nun kälter und der See beginnt abzukühlen.
Zuerst beginnt das Wasser an der Seeoberfläche kälter zu werden.

Beschreiben Sie wie der See abkühlt und dann im Winter gefriert.
Werfen Sie Ihre Texte bitte in mein Fach. Ich werde Sie lesen und Sie Ihnen nächste Woche zurückgeben.


\textbf{Beachten Sie:}\\
Kaltes Wasser hat eine höhere Dichte als warmes Wasser.
Bei \SI{4}{\celsius} ist die Dichte von Wasser am höchsten, kühlt man es weiter ab nimmt die Dichte wieder ab (Anomalie des Wasser).
Bei \SI{0}{\celsius} beginnt Wasser zu gefrieren. Die Dichte von Eis ist geringer als die Dichte von flüssigem Wasser.
Wasser mit höherer Dichte sinkt im See ab.


\end{document}
