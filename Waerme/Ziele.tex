%\documentclass[11pt,a4paper,titlepage,twoside]{article}
\documentclass[12pt,a5paper,twoside]{article}
%\documentclass[12pt,a4paper,twoside]{scrartcl}
\usepackage{mystyle}

%\author{Felix Binder}
\title{Ziele für die Prüfung Wärmelehre am 25.06.15}
\date{}



\def\dir{../Aufgaben_Mechanik/}
\newcommand{\Einbinden}[1]{\input{#1}}


\begin{document}
\maketitle

%\section*{Energie}

\begin{enumerate}
	\item Sie kennen verschieden Temperaturskalen und können Temperaturen umrechnen (von Celsius nach Kelvin und umgekehrt).
	\item Sie können die Längenausdehnung von festen Körpern und Flüssigkeiten bei Temperaturänderung bestimmen.
	\item Sie kennen die ``Anomalie des Wassers'' und können deren Auswirkung auf das Leben auf der Erde erläutern.
	\item Sie können Zustandsänderungen bei Gasen berechnen. Ausserdem kennen Sie das ``Ideale Gas Gesetz'' $p\cdot V = n\cdot R \cdot T$
		und können es in Rechnungen anwenden.
	\item Sie kennen den Begriff der Wärme sowie die spezifische Wärmekapazität und können dies in Aufgaben anwenden.

\end{enumerate}


\end{document}
