%\documentclass[11pt,a4paper,titlepage,twoside]{article}
\documentclass[12pt,a4paper,twoside]{article}
\usepackage{mystyle}
%\usepackage{gplot}
%\usepackage{foto_v001}
%\usepackage{mechanik_v001}

\usepackage{hyperref}
\usepackage{pdfpages}
\usepackage{pdflscape}


%\author{Felix Binder}
\title{Kernphysik}
\date{}

\newcommand{\Kern}[3]{$^{#1}_{\phantom{1}#2}\text{#3}$}
\begin{document}

\section*{Lernziele Radioaktivität}

\begin{itemize}
	\item Sie können die Bindungsenergie an einfachen Beispielen erklären.
	\item Sie kennen das Diagramm der ``Bindungsenergie pro Nukleon'' und können damit
		die Bindungsenergie von verschiedenen Kernen bestimmen.
		Ausserdem können Sie eine Rechnung zur Kernfusion und zur Kernspaltung mit dem Diagramm durchführen und erklären.
	\item Mit Hilfe der Bindungsenergie können Sie erklären warum manche Kerne radioaktive Strahlung abgeben.
		
	\item Sie können die drei Arten radioaktiver Strahlung erklären.
	
	\item Sie kennen die Aktivität und können diese erklären.

	\item Sie kennen das Zerfallsgesetz und können es in Rechnungen anwenden.

	\item Sie können die Halbwertszeit erklären und in Rechnungen anwenden.

	\item Sie können mit Hilfe einer Nuklidkarte die Zerfallsreihe eines Isotops aufstellen.

\end{itemize}

\end{document}
