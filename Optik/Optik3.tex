%\documentclass[11pt,a4paper,titlepage,twoside]{article}
\documentclass[12pt,a4paper,twoside]{article}
\usepackage{mystyle}
\usepackage{optik_v002}
\usepackage{gplot}

%oben und unten
\usepackage{fancyhdr}
\pagestyle{fancy}
\lhead{}
\rhead{}
\rfoot{Felix Binder}
\renewcommand\headrulewidth{0pt}
\renewcommand\footrulewidth{1pt}
%ende oben und unten

%\usepackage{schueler}
%\usepackage{lehrer}
\date{}
%\author{Felix Binder}
\title{Optik}


\begin{document}
%\maketitle

\addtocounter{page}{2}
\addtocounter{section}{8}
\addtocounter{aufgabe}{26}


\section{Abbildungen mit Linsen}
Eine Linse kann einen Gegenstand abbilden. Die Konstruktion verläuft sehr ähnlich
wie bei der Abbildung am gewölbten Spiegel. Wir zeichnen jeweils den Parallelstahl,
den Brennpunktstrahl und den Mittelpunktstahl von der Spitze des Gegenstands ein.

\begin{tikzpicture}
\NObox
	\Biconvex{(IO5)}{Lin}
	\draw [OAchse] (IW)--(IO);
	\draw [OAchse] (IN)--(IS);
	\BP{(IO3)}
	\BP{(IO7)}
	\OG{(IW)}
\end{tikzpicture}


Aus geometrischen Überlegungen kann man für dünne Linsen eine Formel finden,
die die Brennweite $f$, die Bildweite $b$ und die Gegenstandsweite $g$ miteinander
in Beziehung setzt. Es gilt
\begin{eqnarray*}
	\frac{1}{f} = \frac{1}{b} + \frac{1}{g}\text{.}
\end{eqnarray*}

Zwischen der Grösse des Bildes $B$ und der Grösse des Gegenstandes $G$ gilt das selbe Verhältnis,
wie wir es auch schon bei anderen Abbildungen gesehen haben
\begin{eqnarray*}
	A = \frac{B}{G} = \frac{b}{g}\text{.}
\end{eqnarray*}

\begin{aufgabe}
	Die Sammellinse eines Diaprojektors hat eine Brennweite von \SI{10}{cm}. Ein Dia ($G=\SI{36}{mm}$)
	soll auf die Leinwand, die \SI{2.5}{m} von der Linse entfernt ist, abgebildet werden.
	\begin{enumerate} [a)]
		\item Wie weit ist die Linse vom Dia entfernt?
		\item Wie gross ist das Bild auf der Leinwand?
	\end{enumerate}
\end{aufgabe}

\subsection*{$g=2\cdot f$}
\begin{tikzpicture}
\NObox
	\Biconvex{(IO5)}{Lin}
	\draw [OAchse] (IW)--(IO);
	\draw [OAchse] (IN)--(IS);
	\BP{(IO3)}
	\BP{(IO7)}
	\OG{(IO1)}
\end{tikzpicture}

\subsection*{$g>f$ und $g<2\cdot f$}
\begin{tikzpicture}
\NObox
	\Biconvex{(IO5)}{Lin}
	\draw [OAchse] (IW)--(IO);
	\draw [OAchse] (IN)--(IS);
	\BP{(IO3)}
	\BP{(IO7)}
	\OG{(IO2)}
\end{tikzpicture}

\subsection*{$g=f$}
\begin{tikzpicture}
\NObox
	\Biconvex{(IO5)}{Lin}
	\draw [OAchse] (IW)--(IO);
	\draw [OAchse] (IN)--(IS);
	\BP{(IO3)}
	\BP{(IO7)}
	\OG{(IO3)}
\end{tikzpicture}

\subsection*{$g<f$}
\begin{tikzpicture}
\NObox
	\Biconvex{(IO5)}{Lin}
	\draw [OAchse] (IW)--(IO);
	\draw [OAchse] (IN)--(IS);
	\BP{(IO3)}
	\BP{(IO7)}
	\OG{(IO4)}
\end{tikzpicture}


%\newpage
%\includesolutions





\end{document}
