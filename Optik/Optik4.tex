%\documentclass[11pt,a4paper,titlepage,twoside]{article}
\documentclass[12pt,a4paper,twoside]{article}
\usepackage{mystyle}
\usepackage{optik_v002}
\usepackage{foto_v001}
\usepackage{gplot}
\usepackage{import}

%oben und unten
\usepackage{fancyhdr}
\pagestyle{fancy}
\lhead{}
\rhead{}
\rfoot{Felix Binder}
\renewcommand\headrulewidth{0pt}
\renewcommand\footrulewidth{1pt}
%ende oben und unten

%\usepackage{schueler}
%\usepackage{lehrer}
\date{}
%\author{Felix Binder}
\title{Optik}


\begin{document}
%\maketitle

\addtocounter{page}{5}
\addtocounter{section}{9}
\addtocounter{aufgabe}{27}


\section{Optische Geräte}

\subsection{Das menschliche Auge}
%\import{./Eye_scheme.pdf}{./Eye_scheme.pdf_tex}
\begin{center}
\Bildeinbinden{Eye_scheme2.pdf}{0.65\textwidth}
\end{center}

Licht tritt durch die Pupille ins Auge ein. Der Durchmesser der Pupille ist veränderlich. Ist es sehr hell verengt sich die Pupille,
ist es dunkel öffnet sich die Pupille, dadurch kann mehr Licht in das Auge eindringen. Beim Fotoapparat erfüllt die Blende die Funktion der Pupille. 

Die Netzhaut ist eine dünne lichtempfindliche Schicht aus Nervenzellen. Es gibt zwei verschiedene Typen von Nervenzellen auf der Netzhaut,
die Stäbchen und die Zäpfchen. Die Zäpfchen können verschiedene Farben unterscheiden, während die Stäbchen nur zwischen hell und dunkel unterscheiden können.
Ist es dunkel sprechen nur die Stäbchen an, und man kann keine Farben erkennen.

Die Form der Augenlinse, und damit auch die Brennweite der Linse, lässt sich durch die Ziliarkörper etwas verändern.
Befindet sich ein Gegenstand nah am Auge, dann vergrössert der Zilarkörper die Krümmung der Linse, dadurch verringert sich die Brennweite der Linse, und
die Strahlen vom Gegenstand werden wieder auf die Netzhaut fokussiert.
Befindet sich ein Gegenstand zu nah am Auge, kann dieses den Gegenstand nicht mehr scharf auf der Netzhaut abbilden.
Der minimale Abstand, bei dem ein Gegenstand noch scharf dargestellt werden kann heisst Nahpunkt.
Der Nahpunkt kann von Mensch zu Mensch verschieden sein, und ändert sich auch im Laufe des Lebens.
Als Standardwert gilt ein Nahbereich von \SI{25}{Zentimetern}.

\begin{aufgabe}
	Bestimmen Sie (am besten zu zweit) ihren persönlichen Nahpunkt.
	\begin{loesung}
		Der Nahpunkt variiert von Person zu Person, und kann zwischen 10 und \SI{200}{Zentimetern} liegen.
Als Standardwert gilt ein Nahbereich von \SI{25}{Zentimetern}.
	\end{loesung}
\end{aufgabe}
\begin{aufgabe}
	In welchem Bereich liegt die Brennweite des menschlichen Auges. Der Abstand Netzhaut--Linse soll \SI{2.5}{Zentimeter} betragen.
	\begin{loesung}
		Ist ein Gegenstand weit entfernt ($g=\infty$), dann ist die Augenlinse beim Betrachten entspannt.
		Aus der Linsengleichung wird in diesem Fall ($\nicefrac{1}{g}\to 0$)
		\begin{eqnarray*}
			\frac{1}{f}=\frac{1}{b}=\frac{1}{\SI{2.5}{cm}}\text{.}
		\end{eqnarray*}
		Die Brennweite ist dann also \SI{2.5}{cm}.

		Ist der Gegenstand nah am Auge, hier $g=\SI{25}{cm}$, dann gilt
		\begin{eqnarray*}
			\frac{1}{f}=\frac{1}{b} + \frac{1}{g} =\frac{1}{\SI{2.5}{cm}} + \frac{1}{\SI{25}{cm}}=\SI{0.4}{cm^{-1}} + \SI{0.04}{cm^{-1}}=\SI{0.44}{cm^{-1}}\text{.}
		\end{eqnarray*}
		Damit ist die Brennweite für diesen Fall $f=\SI{2.27}{cm}$.
	\end{loesung}
\end{aufgabe}

\subsection{Sehwinkel und Auflösung des Auges}
Lichtstrahlen, die von einem Gegenstand ausgehen fallen unter einem bestimmten Winkel, dem \emph{Sehwinkel} in unser Auge.
Ist der Gegenstand nah, so ist der Sehwinkel gross. Entfernt man den Gegenstand, wird der Sehwinkel kleiner.
Je nach Grösse des Sehwinkels erscheint uns der Gegenstand gross oder klein.

\begin{aufgabe}
	Messen Sie mit einem Geodreieck den Sehwinkel in den zwei Skizzen. Welches Haus erscheint im Auge grösser?
\begin{center}
\begin{tikzpicture}
	\Auge{(0,0)}{Auge}
	\Nikolaushaus{(2,-0.0)}{Haus}
	\draw (Auge O)--(Haus N);
	\draw (Auge O)--(Haus SW);
%	\draw (1,-2) node {grosser Sehwinkel};

	\Auge{(6,0)}{Auge}
	\Nikolaushaus{(14,-0.0)}{Haus}
	\draw (Auge O)--(Haus N);
	\draw (Auge O)--(Haus SW);
%	\draw (8,-2) node {kleinerer Sehwinkel};
\end{tikzpicture}
\end{center}
\end{aufgabe}

\begin{aufgabe}
	Finden Sie eine Formel für den Sehwinkel $\epsilon$. Benutzen sie die Gegenstandsweite und die Gegenstandsgrösse.
	\begin{loesung}
		\begin{eqnarray*}
			\tan\epsilon = \nicefrac{G}{g}
		\end{eqnarray*}
		$G$ ist die Gegenstandsgrösse und $g$ ist die Gegenstandsweite.
	\end{loesung}
\end{aufgabe}

\begin{aufgabe}
	\label{zweiPunkte}
	Sie haben in einem Buch gelesen, dass der minimale Sehwinkel, den das menschliche Auge noch auflösen kann etwa ein sechzigstell eines Grades gross ist.
	Welchen Abstand müssen zwei Punkt mindestens voneinander haben, wenn Sie diese mit ihrem Auge noch als zwei separate Punkte erkennen wollen. 
	Berechnen Sie den Abstand am Nahpunkt des Auges bei \SI{25}{Zentimetern}.
	\begin{loesung}
		Wir benutzen die Formel für den Sehwinkel. $G$ ist gesucht, die anderen Angaben stehen in der Aufgabe.
		\begin{eqnarray*}
			\tan\epsilon = \frac{G}{g}\to G=\tan\epsilon\cdot g=\tan(\frac{1}{60})\cdot\SI{0.25}{m}=\SI{2.91}\cdot\SI{0.25}{m}=\SI{7.3E-5}{m}=\SI{73}{\mu m}
		\end{eqnarray*}
		Sind zwei Punkte also näher als \SI{73}{\mu m} von einander entfernt, 
		kann das Auge sie nicht mehr als zwei Punkte erkennen, wenn sie nicht näher als \SI{25}{cm} vor dem Auge sind.
	\end{loesung}
\end{aufgabe}


\begin{aufgabe}
	Sie machen mit ihrem Handy ein Foto, und wollen es später ausdrucken lassen. Die Kamera in ihrem Handy schafft eine Auflösung von $3264\times2448$
	Pixeln.
	\begin{enumerate} [a)]
		\item Wie viele Pixel hat ihre Kamera?
		\item In welcher grösse können Sie das Foto ausdrucken lassen, ohne Qualitätsverluste festzustellen 
			(sie wollen das Foto wie in Aufgabe \ref{zweiPunkte} im Abstand von \SI{25}{Zentimetern} betrachten können).
		\item Sie wollen das Foto ganz gross zeigen, und vergrössern es soweit, dass jedes Pixel \SI{1}{cm^2} gross ist. Wie gross wird dann das Foto? 
			Aus welchem Abstand müssten Sie es betrachten, damit es für Sie aussieht wie in Aufgabenteil b).
	\end{enumerate}
	\begin{loesung}
		\begin{enumerate} [a)]
			\item Es sind \num{3264} mal \num{2448} Pixel. Das sind total \num{7990272} Pixel. Die Kamera hat also \SI{8}{Megapixel}.
			\item In der Aufgabe \ref{zweiPunkte} haben wir den minimalen Abstand zwischen zwei Punkten berechnet.
				Das können wir hier nutzen.
				\begin{eqnarray*}
					\begin{split}
					3264\cdot\SI{73}{\mu m}&= \SI{0.24}{cm}\\
					2448\cdot\SI{73}{\mu m}&= \SI{0.18}{cm}
					\end{split}
				\end{eqnarray*}
	\item Ist jedes Pixel \SI{1}{cm} mal \SI{1}{cm} gross, dann ist das Foto insgesamt
		\begin{eqnarray*}
					\begin{split}
						3264\cdot\SI{1}{cm}&= \SI{3264}{cm}=\SI{32.64}{m}\\
						2448\cdot\SI{1}{cm}&= \SI{2448}{cm}=\SI{24.48}{m}
					\end{split}
				\end{eqnarray*}
			gross.
			Damit das grosse Foto so aussieht wie das kleine, muss der Sehwinkel gleich gross sein.
			Es muss gelten:
			\begin{eqnarray*}
				\tan\epsilon=\frac{\RI{G}{klein}}{\RI{g}{klein}}=\frac{\RI{G}{gross}}{\RI{g}{gross}}\text{.}
			\end{eqnarray*}
			Auflösen der Formel nach \RI{g}{gross} und einsetzten der Werte ergibt:
			\begin{eqnarray*}
				\RI{g}{gross}=\frac{\RI{G}{gross}}{\RI{G}{klein}}\cdot\RI{g}{glein}=\frac{\SI{32.64}{m}}{\SI{0.24}{m}}\cdot\SI{0.25}{m}=\SI{34}{m}\text{.}	
			\end{eqnarray*}
			Betrachtet man das grosse Foto aus einem Abstand von \SI{34}{m} sieht es aus, wie das kleine Foto in einem Abstand von \SI{25}{cm}.
		\end{enumerate}
	\end{loesung}

\end{aufgabe}

\newpage
\pagestyle{empty}
\includesolutions

\end{document}
