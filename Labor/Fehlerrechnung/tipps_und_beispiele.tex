\documentclass[12pt,a4paper, twosite]{article}
\usepackage{mystyle}

%oben und unten
\usepackage{fancyhdr}
\pagestyle{fancy}
\lhead{St. Michael\\Fribourg}
\rhead{Labor: Messfehler}
\rfoot{Felix Binder}
\lfoot{September 2014}
\renewcommand\headrulewidth{1pt}
\renewcommand\footrulewidth{1pt}
%ende oben und unten

%\newcommand{\Beispiel}[0]{\par\rm{Beispiel:}\par}

\author{}
\date{}
\title{Fragen und Beispiele zu Messungenauigkeiten}

\begin{document}
\maketitle
%\thispagestyle{empty}

\section{Messungenauigkeit}
\Beispiel
Sie wollen die Kraft, die der Luftdruck auf eine runde Fläche ausübt berechnen.
Der Durchmesser der Fläche ist etwa \SI{16}{cm}. Da Sie mit einem normalen Geodreieck mit zerstossenen Enden messen,
schätzen Sie den absoluten Messfehler auf \SI{1}{cm}. Also $d=\SI{0.16}{m}\pm \SI{1}{cm}$.
Der Luftdruck ist etwa $\SI{950}{hPa} \pm \SI{20}{hpa}$.

Aus der Definition des Drucks ($P=\nicefrac{F}{A}$)
erhalten Sie durch umformen
\begin{eqnarray*}
	F=P\cdot A = P\cdot\pi \cdot r^2 = \SI{95000}{Pa}\cdot\pi\cdot(\SI{0.08}{m})^2 = \SI{1910.088}{N}\text{.}
\end{eqnarray*}

Die absolute Messungenauigkeit berechnen Sie so
\begin{eqnarray*}
	\Delta F= P\cdot\pi\cdot r^2\left(\frac{\Delta P}{P} + 0 + 2\cdot\frac{\Delta r}{r}\right)=\SI{1910.088}{N}\cdot\left(\num{0.021}+\num{0.125}\right)=\SI{278.973}{N}
\end{eqnarray*}


%wie gibt man ergebnisse an?
\section{Runden der absoluten Messungenauigkeit}
Runden Sie die absolute Messungenauigkeit auf eine signifikante Stelle.
Sie können beim runden die normalen Rundungsregeln benutzen. 
Falls Sie unsicher sind, dass die Messunsicherheit damit zu klein wird können Sie auch aufrunden.
\Beispiel
\begin{eqnarray*}
	\Delta F= \SI{278.973}{N} \approx \SI{300}{N}
\end{eqnarray*}

\section{Wie gibt man das Endergebnis an?}
\Beispiel
\begin{eqnarray*}
	F =  \SI{1910.088}{N} \pm  \SI{278.973}{N}\approx\SI{1900}{N}\pm\SI{300}{N} 
\end{eqnarray*}

\end{document}
