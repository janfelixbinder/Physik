\documentclass[12pt,a4paper, twosite]{article}
\usepackage{mystyle}

%oben und unten
\usepackage{fancyhdr}
\pagestyle{fancy}
\lhead{St. Michael\\Fribourg}
\rhead{Labor: Messfehler}
\rfoot{Felix Binder}
\lfoot{September 2014}
\renewcommand\headrulewidth{1pt}
\renewcommand\footrulewidth{1pt}
%ende oben und unten


\author{}
\date{}
\title{Bestimmung von Messfehlern}

\begin{document}
\maketitle
%\thispagestyle{empty}

\section{Volumen und Dichte eines Quaders}
\begin{enumerate}[a)]
	\item Bestimmen Sie durch messen mit einem (normalen) Lineal das Volumen eines Quaders.
	Welche Messgenauigkeiten hat das Messgerät? Welche absoluten und relativen Fehler ergeben sich dadurch bei der Messung?
	\item Wiederholen Sie die Messung mit einem Messschieber mit Nonius.
	\item Bestimmen Sie das Volumen durch Verdrängung. Welche Fehler ergeben sich nun?
	\item Welche Dichte hat dieser Quader. Welche absoluten und relativen Fehler haben Sie bei der Dichtebestimmung.
\end{enumerate}

\section{Volumen und Dichte eines Hohlzylinders}
\begin{enumerate}[a)]
	\item Bestimmen Sie durch messen mit einem Messschieber das Volumen eines Hohlzylinders.
	\item Welche Dichte hat der Hohlzylinder?
\end{enumerate}

\end{document}
