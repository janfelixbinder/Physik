\documentclass[12pt,a4paper, twosite]{article}
\usepackage{mystyle}

%oben und unten
\usepackage{fancyhdr}
\pagestyle{fancy}
\lhead{St. Michael\\Fribourg}
\rhead{Labor: Radioaktivität}
\rfoot{Felix Binder}
\lfoot{März 2015}
\renewcommand\headrulewidth{1pt}
\renewcommand\footrulewidth{1pt}
%ende oben und unten

\usepackage{hyperref}

\author{}
\date{}
\title{Radioaktivität}

\begin{document}
\maketitle
%\thispagestyle{empty}

\section*{Sicherheitshinweise}
Während des Labors wird mit schwach radioaktiven Stoffen gearbeitet. Diese haben richtig angewendet keinen Einfluss auf
die Gesundheit. Das Labormaterial im Speziellen das Geiger-Müller-Zählrohr sind sehr empfindlich.
Bei nicht Benutzung ist stets der Plastikschutz aufzustecken. Die Membran ist sehr dünn und darf niemals berührt werden.
Die Kosten bei Neubeschaffung sind sehr hoch (800 Fr.).

\section*{Material}
Für diesen Praktikumsversuch werden folgende Materialien benötigt:
\begin{itemize}
	\item Experimentierkasten Radioaktivität 
	\item Geiger-Müller-Zählrohr
	\item Computer für die Recherche und die Erstellung eines Laborberichtes.
\end{itemize}



\section*{Abschirmung von radioaktiver Strahlung}
Dieses Labor ist mit vielen Einzelmessungen verbunden.
Wählen Sie eine Strahlungsquelle mit möglichst hoher Aktivität.
Benutzen Sie unterschiedliche Materialien, um die radioaktive Strahlung der Quelle abzuschirmen. 

Mögliche Fragestellungen sind:
\begin{itemize}
	\item Warum lassen sich radioaktive Strahlen abschirmen?
	\item Welche Strahlung braucht welches Material zum abschirmen?
	\item Was ist die Halbwertsdicke?
	\item Kann man die Probe durch Abschirmung charakterisieren?
	\item Lässt sich damit auf das Alter der Probe schliessen?
\end{itemize}


\begin{aufgabe}
	Bestimmen Sie den Anteil von $\alpha$- $\beta$- und $\gamma$-Strahlung der Ra-226 Strahlungsquelle.
	Schirmen Sie die Strahlenquelle dazu geeignet ab und messen Sie die durch das Abschirmmaterial gelangte Strahlung.

	Beispiel:\\
	Zuerst messen Sie die Radioaktive Strahlung ohne Abschirmung. 
	Im zweiten Schritt stellen Sie eine Abschirmung aus Plastik zwischen Quelle und Geiger-Müller-Zählrohr.
	Die Plastikschicht schirmt die $\alpha$-Strahlung ab, $\beta$- und $\gamma$-Strahlung können das Plastik aber überwinden.
	Die Differenz zwischen nicht abgeschirmter Impulsrate und der mit der Plastikschicht abgeschirmten Impulsrate ergibt die
	Impulsrate der $\alpha$-Strahlung.
	Für $\beta$- und $\gamma$-Strahlung gehen Sie analog vor.
\end{aufgabe}

\begin{aufgabe}
	Wie viele Schichten Papier schirmen die $\alpha$-Strahlung komplett ab?
\end{aufgabe}

\begin{aufgabe}
	Wie viele Schichten Aluminiumfolie schirmen die $\beta$-Strahlung komplett ab?
\end{aufgabe}

\begin{aufgabe}
	Wie gross muss der Abstand zwischen Quelle und Geiger-Müller-Zählrohr sein, 
	damit die gesamte $\alpha$-Strahlung abgeschirmt wird?

	Für $\alpha$-Teilchen der Emissionsenergie $E_0$ in Luft wurde das folgende Gesetz beobachtet:
	\begin{eqnarray*}
		R = \num{3.1} \cdot E_0^{3/2}\text{.}
	\end{eqnarray*}
	Dabei ist $R$ die maximale Reichweite in Millimetern. Die Emissionsenergie ist in MeV.

	Wie hoch ist die Emissionsenergie bei Ihrer Probe? 

	Achtung:\\
	Rücken Sie Quelle und Zählrohr nicht zu nah zusammen. Die Zählrate ist dann zu hoch für das Geiger-Müller-Zählrohr.

	Bedenken Sie auch, dass der Ausschnitt den das Geiger-Müller-Zählrohr misst mit dem Abstand kleiner Wird.
	Bei einem angenommen Abstand von \SI{0}{cm} nehmen Sie die Hälfte der Zerfälle auf.
	Bei einem grösseren Abstand ist die Fläche auf der sich die Strahlung gleichmässig verteilt 
	ein Halbkreis mit dem Radius gleich dem Abstand zwischen Probe und Zählrohr.
	Wenn Sie nun wissen, wie gross die Detektorfläche des Zählrohrs ist können Sie ihre Messergebnisse korrigieren.

\end{aufgabe}

\section*{Tools: Graphiken und Tabellen}
Unter den zwei folgenden Links finden Sie Programme zum erstellen von Graphiken
\url{http://onlinecharttool.com/} oder \url{http://raw.densitydesign.org/}

Unter dieser Adresse findet sich ein Service zum Umwandeln von Tabellen im Textformat (csv) in das Wiki-Format von Tabellen.
\url{http://mlei.net/shared/tool/csv-wiki.htm}

\end{document}
