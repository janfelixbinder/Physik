\documentclass[12pt,a5paper, twosite]{article}
\usepackage{mystyle}

%oben und unten
\usepackage{fancyhdr}
\pagestyle{fancy}
\lhead{St. Michael\\Fribourg}
\rhead{Labor: Radioaktivität}
\rfoot{Felix Binder}
\lfoot{März 2015}
\renewcommand\headrulewidth{1pt}
\renewcommand\footrulewidth{1pt}
%ende oben und unten

\usepackage{hyperref}

\author{}
\date{}
\title{Wiederholung Radioaktivität}

\begin{document}
\maketitle
%\thispagestyle{empty}

Erstellen Sie einen ein- bis zweiminütigen Audiotrack zu einem der folgenden Themen:


\begin{itemize}
	\item Bindungsenergie
	\item Radioaktive Strahlung
	\item $\alpha$-Strahlung
	\item $\beta$-Strahlung
	\item Zerfallsgesetz
	\item Halbwertszeit
\end{itemize}


Schreiben Sie zuerst eine Geschichte.
Nehmen Sie die Geschichte mit Ihrem Handy oder mit dem Computer auf.
Schneiden Sie die aufgenommene Geschichte und unterlegen Sie Ihren Track mit Musik die Sie mögen.
Laden Sie die fertige Datei in die Dateiablage im educanet2.

\end{document}
