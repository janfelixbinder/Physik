\documentclass[11pt,a5paper, twosite]{article}
\usepackage{mystyle}

%oben und unten
\usepackage{fancyhdr}
\pagestyle{fancy}
\lhead{St. Michael\\Fribourg}
\rhead{Labor: Wärmelehre}
\rfoot{Felix Binder}
\lfoot{April 2015}
\renewcommand\headrulewidth{1pt}
\renewcommand\footrulewidth{1pt}
%ende oben und unten


\author{}
\date{}
\title{Längenausdehnung}

\begin{document}
\maketitle

In diesem Labor sollen Sie den Längenausdehnungskoeffizienten für verschiedene Materialien bestimmen.

\subsection*{Vorüberlegungen}
\begin{enumerate}[a)]
	\item Beschreiben Sie den Versuchsaufbau.
	\item Überlegen Sie sich, warum der Versuchsaufbau so sinnvoll ist.
\end{enumerate}

\subsection*{Vor der Messung}
\begin{enumerate}[c)]
	\item Überlegen Sie sich, welche Messwerte sie in diesem Versuch messen wollen.
	\item Legen Sie eine Tabelle an, in der Sie diese Messwerte protokollieren.
\end{enumerate}

\subsection*{Auswertung}
\begin{enumerate}[e)]
	\item Bestimmen Sie die Ausdehnungskoeffizienten für die verschiedenen Rohre.
	\item Machen Sie eine Fehlerabschätzung.
	\item Vergleichen Sie mit dem Tabellenwert.
\end{enumerate}



\end{document}
