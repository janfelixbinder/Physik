\documentclass[12pt,a5paper, twosite]{article}
\usepackage{mystyle}

%oben und unten
\usepackage{fancyhdr}
\pagestyle{fancy}
\lhead{St. Michael\\Fribourg}
\rhead{Labor: Wärmelehre}
\rfoot{Felix Binder}
\lfoot{April 2015}
\renewcommand\headrulewidth{1pt}
\renewcommand\footrulewidth{1pt}
%ende oben und unten


\author{}
\date{}
\title{Gasthermometer}

\begin{document}
\maketitle

In diesem Labor sollen Sie ein möglichst empfindliches Thermometer bauen.
Schaffen Sie es mit Ihrem Thermometer die Wärmestrahlung Ihrer Hände zu detektieren?

Bauanleitung:\\
Füllen Sie in eine Flasche Wasser. 
Befestigen Sie einen Trinkhalm mit Knete an der Flaschenöffnung, so dass diese vollständig abgedichtet ist.
Um zu testen, ob die Flasche wirklich dicht ist, können Sie vorsichtig in den Trinkhalm blasen.


Bevor Sie beginnen, stellen Sie eine Vermutung auf. 
Ist es möglich, dass der Wasserstand im Trinkhalm nur dadurch steigt, dass Sie Ihre Hände in die Nähe der Flasche bringen (Sie berühren die Flasche nicht).
Oder benötigen Sie Wärmeleitung (Sie fassen die Flasche an).

Protokollieren Sie Ihre Überlegungen im Laborheft.

\end{document}
