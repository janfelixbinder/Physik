\documentclass[11pt,a4paper, twosite]{article}
\usepackage{mystyle}

%oben und unten
\usepackage{fancyhdr}
\pagestyle{fancy}
\lhead{St. Michael\\Fribourg}
\rhead{Labor: Wärmelehre}
\rfoot{Felix Binder}
\lfoot{Juni 2015}
\renewcommand\headrulewidth{1pt}
\renewcommand\footrulewidth{1pt}
%ende oben und unten


\author{}
\date{}
\title{Spezifische Wärme von Wasser}

\begin{document}
\maketitle

In diesem Labor erwärmen Sie Wasser und untersuchen wie viel Energie Sie dazu benötigen.

\subsection*{Aufbau}
Erwärmen Sie zuerst ein Kilogramm Wasser mit einem Tauchsieder.
Messen Sie alle 30 Sekunden, wie warm das Wasser bisher geworden ist und protokollieren Sie alle Temperaturen in einer Tabelle.
Damit Sie die richtige Temperatur messen ist es wichtig, dass Sie das Wasser gut umrühren, nur so kann sich die Wärme gleichmässig im Wasser verteilen.

Wiederholen Sie das Experiment mit zwei Kilogramm Wasser.

\subsection*{Vorüberlegungen}
\begin{enumerate}[a)]
	\item Überlegen Sie, wie die Messkurven (\SI{1}{kg} und \SI{2}{kg}) aussehen könnte.
\end{enumerate}

\subsection*{Auswertung}
\begin{enumerate}[b)]
	\item Zeichen Sie die Messkurven (\SI{1}{kg} und \SI{2}{kg}) für das von Ihnen durchgeführte Experiment.

	\item [c)] Bestimmen Sie aus Ihren Messwerten die spezifische Wärmekapazität von Wasser.
\end{enumerate}



Es gilt die Erhaltung der Energie. Das heisst, die Energie die der Tauchsieder abgibt, wird in Wärme an das Wasser abgegeben.
Dadurch erwärmt sich das Wasser. Es gilt
\begin{eqnarray*}
	Q = c\cdot m\cdot \Delta T
\end{eqnarray*}
$Q$ ist die Wärmemenge, das ist die Menge an Energie, die in Form von Wärme vom Wasser aufgenommen wird.
$m$ ist die Masse, und $\Delta T$ ist die Temperaturdifferenz.
Das kleine $c$ nennt man spezifische Wärmekapazität. Es ist eine Materialkonstante und damit für jedes Material unterschiedlich.




\end{document}
