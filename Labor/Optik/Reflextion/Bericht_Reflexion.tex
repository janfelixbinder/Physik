\documentclass[11pt,a4paper]{article}
\usepackage{mystyle}
%\usepackage{lehrer}


\usepackage{circuitikz}

\author{Felix Binder}
\title{Reflexion am ebenen Spiegel}

\begin{document}
\maketitle

%\thispagestyle{empty}

\section{Einleitung}
Aus dem Alltag kennen wir, dass glatte und polierte Oberflächen Licht reflektieren.
Das wichtigste technische Gerät, dass sich dieses Effekts bedient ist der Spiegel.
Am Spiegel gilt das Reflexionsgesetz, dieses lautet: Einfallswinkel gleich Ausfallswinkel.
Im folgenden Experiment werden wir das Reflexionsgesetz am ebenen Spiegel überprüfen.


\section{Versuchsaufbau}
Um das Reflexionsgesetz am ebenen Spiegel zu überprüfen wollen wir einen Lichtstrahl an
einer Spiegeloberfläche reflektieren lassen.
Als Lichtquelle dient uns eine Halogenlampe. Wir richten die Glühwendel der Lampe so aus,
dass sie senkrecht zur Tischoberfläche steht.
Durch eine Schlitzblende erreichen wir, dass nur ein schmales Lichtbündel auf die Spiegeloberfläche fällt.
Den Spiegel richten wir anfangs so aus, dass das Lichtbündel auf gleichem Wege zurück reflektiert wird.
\begin{center}
Hier sollte eine Zeichnung des Versuchsaufbaus sein.
\end{center}

\section{Versuch}
Durch drehen des Spiegels lässt sich der Einfallswinkel des Lichtbündels ändern. Wir messen als erste Grösse den Einfallswinkel
zum Lot des Spiegels. Das Lot ist eine Gerade, die senkrecht auf der Spiegeloberfläche steht.
Die zweite Grösse die wir messen ist der Ausfallswinkel. Auch dieser wird zum Lot gemessen. 
\begin{center}
	Hier könnte eine Tabelle stehen, in die die Messwerte protokolliert sind.
\end{center}


\section{Auswertung}
Wir stellen fest,
dass der einfallende Lichtstrahl, das Lot und der ausfallende Lichtstrahl stets in einer Ebene liegen.
Ausserdem ist die Grösse des Einfallswinkel gleich der Grösse des Ausfallswinkels.
\begin{center}
	Hier könnte man einen Graph einfügen, um die Messwerte zu visualisieren. 
\end{center}


\section{Zusammenfassung}
In diesem Experiment haben wir das Reflexionsgesetz überprüft und bestätigt. 
Am ebenen Spiegel gilt: Einfallswinkel gleich Ausfallswinkel.
Ausserdem haben wir gefunden, dass einfallender Lichtstrahl, Lot und ausfallender Lichtstrahl bei der Reflexion stets in einer Ebene liegen.

\end{document}
