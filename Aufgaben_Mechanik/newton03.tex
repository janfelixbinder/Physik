
\begin{aufgabe}
	Auf einer Luftkissenbahn steht ein Schlitten mit einer Masse von \SI{5}{kg} ($m_1$). Über eine Schnur ist $m_1$ mit einer anderen Masse
	von \SI{2}{kg} verbunden(siehe Abbildung \ref{fig:luftkissen}).
	Durch die Luftkissenbahn kann Reibung vernachlässigt werden.
	Wie gross ist die Beschleunigung?
	\begin{loesung}
		Die beschleunigende Kraft ist $F=m_2\cdot g$ die zu beschleunigende Masse ist $(m_1 +m_2)$.
		\begin{align*}
			F&=m\cdot a\\
			m_2\cdot g &= (m_1+m_2)\cdot a \to a=g\cdot\frac{m_2}{m1+m_2}=\SI{9.81}\cdot\frac{\SI{2}{kg}}{\SI{5}{kg}+\SI{2}{kg}}=\SI{2.8}{m/s^2}
		\end{align*}
	\end{loesung}
	\kloesung{$a=\SI{2.8}{m/s^2}$}
\end{aufgabe}

