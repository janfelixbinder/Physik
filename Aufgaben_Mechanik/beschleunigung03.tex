

\begin{aufgabe}
Ein Ball wird mit einer Geschwindigkeit von \SI{30}{m/s} nach oben geworfen, und fällt
durch die Fallbeschleunigung wieder zu Boden.
\begin{itemize}
	\item[a)] Zeichnen Sie den Wurf in ein Beschleunigungs-Zeit-, ein Geschwindigkeits-Zeit- und ein Weg-Zeit-Diagramm.
	\item[b)] Wie lange braucht der Ball bis zum höchsten Punkt?
	\item[c)] Wie hoch steigt der Ball insgesamt?
	\item[d)] Der Ball soll \SI{50}{m} hoch kommen. Mit welcher Geschwindigkeit $v_0$ muss er hochgeworfen werden?
\end{itemize}

\kloesung{b) \SI{3}{s}, c) \SI{45}{m}, d) \SI{31.6}{m/s}}

\begin{loesung}
\begin{itemize}
	\item[b)] Am höchsten Punkt ist die Geschwindigkeit $v=0$. Wir können also schreiben $0 = v_0 + a\cdot t \to t=-\frac{v_0}{a}=\frac{\SI{30}{m/s}}{{10}{m/s^2}}=\SI{3}{s}$.

	\item[c)] Dies kann auf drei unterschiedlichen Wegen bestimmt werden. 
\begin{itemize}
	\item Man bestimmt die Fläche im $v$-$t$-Diagramm.
	\item $\bar{v}=\frac{1}{2}\cdot (v_0+v_1)$ $\to s=\bar{v}\cdot t=\SI{15}{m/s}\cdot\SI{3}{s}=\SI{45}{m}$
	\item Oder $s=v_0\cdot t +\frac{1}{2}\cdot a \cdot t^2 = \SI{30}{m/s}\cdot\SI{3}{s}+0.5\cdot(\SI{-10}{m/s^2})\cdot(\SI{3}{s})^2=\SI{90}{m}-\SI{45}{m}=\SI{45}{m}$
\end{itemize}
	\item[d)]
		\begin{align*}
		v^2=v_0^2 + 2\cdot a \cdot \Delta s \to v_0 &= \sqrt{v^2 - 2\cdot a \cdot \Delta s}=\sqrt{\SI{0}{m/s} - 2\cdot (\SI{-10}{m/s^2})\cdot\SI{50}{m}}=\\
			&=\sqrt{\SI{1000}{m^2/s^2}}=\SI{31.6}{m/s}
		\end{align*}
\end{itemize}


\end{loesung}
\end{aufgabe}
