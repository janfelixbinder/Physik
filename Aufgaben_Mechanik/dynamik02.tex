
\begin{aufgabe}
	Ein Velofahrer (\SI{70}{kg}) beschleunigt mit \SI{2}{m/s^2}.
	Das Velo hat eine Masse von \SI{15}{kg}.
	Wie viel Kraft braucht er für die Beschleunigung?
	\kloesung{\SI{170}{N}}
	\begin{loesung}
		Die gesamte Masse ist $\SI{70}{kg} + \SI{15}{kg}=\SI{85}{kg}$.
		Mit dem zweiten Newtonschen Gesetz ergibt sich das folgende:
		\begin{eqnarray*}
			F=m\cdot a=\SI{85}{kg}\cdot\SI{2}{m/s}=\SI{170}{N}\text{.}
		\end{eqnarray*}
	\end{loesung}
\end{aufgabe}

