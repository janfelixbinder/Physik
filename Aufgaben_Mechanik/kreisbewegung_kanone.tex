
\begin{aufgabe}
	Das ersten Newtonsche Gesetz besagt, dass die gleichmässig gleichförmige Geschwindigkeit der natürliche Bewegungszustand
	jedes Körpers ist. Ohne äussere Kräfte ändert sich die Geschwindigkeit eines Körpers nicht.
	Schon Newton erkannte das Prinzip. Nehmen Sie an, Sie stehen auf einem \SI{8000}{m} hohen Berg, und schiessen mit
	einer Kanone. Welche Geschwindigkeit brauch die Kanonenkugel, um einmal um die Erde zu kommen.
	Jede Art von Reibung soll vernachlässigt werden.
\end{aufgabe}
\begin{center}
	\Bildeinbinden{\dir/newtmtn.png}{0.5}
\end{center}
\begin{loesung}

\end{loesung}
