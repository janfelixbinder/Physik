
\begin{aufgabe}
	Ein Schrank (\SI{50}{kg}) soll um zwei Meter verrückt werden. Die Gleitreibungszahl ist \num{0.3}.
	\begin{enumerate} [a)]
		\item Machen Sie eine Skizze und zeichnen Sie die wirkenden Kräfte ein.
		\item	Wie viel Arbeit ist das verrücken des Schrankes?	
		\item   Sie benutzen zum Verrücken einen Rollwagen. Die Rollreibungszahl sei \num{0.05}.
			Müssen Sie mehr oder weniger arbeiten als bei a)? 
		\item Wie viel Arbeit ist nötig für das Verrücken unter b)?
	\end{enumerate}
 \kloesung{b) \SI{294.3}{J}, d) \SI{49}{J}}

\end{aufgabe}

