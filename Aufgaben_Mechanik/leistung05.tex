
\begin{aufgabe}
	Sie haben sicher schon einmal gehört, dass Licht eine Energieform ist. Wenn nicht, kennen Sie ja sicher Solarzellen.
	Diese Bauteile wandeln Lichtenergie in elektrische Energie um. Lichtenergie wird in kleinen Paketen übertragen, den
	sogenannten Photonen. Die Energiemenge, die ein Photon überträgt, ist abhängig von der Farbe des Lichtes.
	Es gilt $E=h\cdot \nu$. Dabei ist $h$ das Planck'sche Wirkungsquantum, sein Wert ist \SI{6.626E-34}{Js} und $\nu$
	die Frequenz des Lichtes.

	Ein roter Laserpointer ($\nu=\SI{4.4E14}{s^{-1}}$) mit einer Lichtleistung von \SI{3}{mW} erreicht einen Wirkungsgrad von \SI{80}{\percent}.
	\begin{enumerate} [a)]
		\item Wie viel Energie wird in einer Sekunde in Laserlicht abgegeben?
			\TAX{Berechnung mit numerischem Resultat K2}
		\item Schätzen Sie ab, wie viele Photonen vom Laserpointer während einer Sekunde abgegeben werden?
			\TAX{Fermiproblem K1}
	\end{enumerate}
	\begin{loesung}
		\begin{enumerate}[a)]
			\item Die Lichtleistung und die Zeit sind gegeben. Damit kommt man auf
				\begin{eqnarray*}
					P=\frac{\Delta E}{\Delta t} \to \Delta E= P\cdot\Delta t=\SI{3}{mW}\cdot\SI{1}{s}=\SI{3}{mWs}=\SI{3}{mJ}\text{.}
				\end{eqnarray*}
			\item Zuerst schätzen wir die Energie eines Photons
				\begin{eqnarray*}
					E=h\cdot\nu\approx=\SI{1E-34}{Js}\cdot\SI{1E14}{s^{-1}} = \SI{1E-20}{J}\text{.}
				\end{eqnarray*}
Nun können wir die Anzahl der Photonen schätzen
				\begin{eqnarray*}
					n=\frac{\Delta E}{\RI{E}{Photon}}\approx\frac{\SI{1E-3}{J}}{\SI{1E-20}{J}}=\num{1E17}\text{.}
				\end{eqnarray*}
				Rechnen wir mit den exakten Werten bekommen wir eine Anzahl von \num{1.03E16} Photonen, also etwa
				\nicefrac{1}{10} unseres Schätzwertes.
		\end{enumerate}
	\end{loesung}
\end{aufgabe}
