
\begin{aufgabe}
Ein Auto beschleunigt gradlinig gleichförmig in \SI{5}{s} von \SI{0}{km/h} auf \SI{100}{km/h}.
Wie gross ist die Beschleunigung?

\kloesung{\SI{5.56}{m/s^2}}

\begin{loesung}
\[\Delta v = v_1-v_0=\SI{100}{km/h}= \SI{27.8}{m/s}\]
\[a=\frac{\Delta v}{\Delta t} =\frac{\SI{27.8}{m/s}}{\SI{5}{s}} = \SI{5.56}{m/s^2}\]
\end{loesung}
\end{aufgabe}

