
\begin{aufgabe}
\StartLueckentext

Kräfte sind \gl. Damit ein Leser gleich sieht, dass es sich um eine \gl Grösse, und nicht um eine \gl Grösse handelt,
schreibt man über dem Formelbuchstaben einen kleinen \gl. Zum Beispiel $\vec{F}$.

Ein Vektor hat einen \gl und eine \gl, so wie ein Pfeil. Daher benutzt man Pfeile, wenn man einen Vektor zeichnen möchte.

Will man zwei Kräfte, die einen gemeinsamen \gl haben, zeichnerisch addieren, so zeichnet man im Kräfteplan ein \gl .
Allgemein kann man Vektoren mit gemeinsamen Angriffspunkt zeichnerisch durch \gl Verschieben addieren.
Erhält man einen geschlossenen Weg aus Pfeilen, bei dem jeweils eine Spitze ein Ende eines Pfeils berühren, so ist die Summe der Vektoren \gl.

\StoppLueckentext
\end{aufgabe}

