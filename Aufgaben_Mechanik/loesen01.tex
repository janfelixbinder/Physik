
\begin{aufgabe}
	Zwei Arbeiter tragen auf ihren Schultern einen \SI{12}{m} langen und \SI{0.6}{kN}
	schweren Balken. Der eine trägt \SI{2}{m}, der andere \SI{1}{m} vom jeweiligen Ende des Balkens entfernt.
	\begin{enumerate} [a)]
		\item Machen Sie sich eine Skizze und zeichnen Sie alle Kräfte ein.
		\item Wählen Sie eine Drehachse und stellen sie die Gleichgewichtsbedingungen auf.
		\item Lösen Sie das Gleichungssystem. Welche Last trägt welcher Arbeiter?
	\end{enumerate}
	\kloesung{c) \SI{0.2667}{kN} und \SI{0.333}{kN}.}

	\begin{loesung}
		\begin{enumerate} [a)]
			\item Die Skizze könnte so aussehen:
				\begin{center}
					\begin{tikzpicture}
						\draw [] (0,0) rectangle (12,0.3);
\draw [fill] (6,0.15) circle(0.1);
\draw [Kraft,->] (6,0.15)--+(-90:3cm) node [right]{\RI{F}{G}};

\draw [Kraft,<-] (2,0)--++(-90:2cm) node [right] {$F_1$};
\draw [Kraft,<-] (11,0)--++(-90:2cm) node [right] {$F_2$};
					\end{tikzpicture}
				\end{center}
			\item Als Drehachse wähle ich den Schwerpunkt des Balken. Die zwei Gleichgewichtsbedingungen lauten:
				\begin{eqnarray*}
					F_1 + F_2 - \RI{F}{G} =0
				\end{eqnarray*}
				und
				\begin{eqnarray*}
					0\cdot\RI{F}{G} - \SI{4}{m}\cdot F_1 + \SI{5}{m}\cdot F_2 =0
				\end{eqnarray*}
			\item Aus der Bedingung für die Drehmomente lässt sich $F_1$ in Abhängigkeit von $F_2$ auflösen.
				\begin{eqnarray*}
					0\cdot\RI{F}{G} - \SI{4}{m}\cdot F_1 + \SI{5}{m}\cdot F_2 =0 \to F_1 =  \frac{5}{4}\cdot F_2
				\end{eqnarray*}
				Dies kann man dann in die Kraftbedingung einsetzen und auflösen.
				\begin{eqnarray*}
					F_1 + F_2 - \RI{F}{G} =0\to  \frac{5}{4}\cdot F_2 + F_2 - \RI{F}{G} =0\to  \frac{9}{4}\cdot F_2 =\RI{F}{G}\to F_2=\frac{4}{9}\cdot\RI{F}{G}=\frac{4}{9}\cdot\SI{0.6}{kN}=\SI{0.2667}{kN}
				\end{eqnarray*}
				$F_1$ ist dann \SI{0.333}{kN} gross.

		\end{enumerate}
	\end{loesung}
\end{aufgabe}
