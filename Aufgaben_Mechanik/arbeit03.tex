%siehe auch sprung.gplot (gnuplot)


\begin{aufgabe}
	Ein Bungeespringer (\SI{65}{kg}) springt von einer hohen Brücke.
	Sein Bungeeseil ist unbelastet 38 Meter lang. Während des Sprungs verlängert sich das Seil auf 55 Meter.

	\begin{enumerate} [a)]
		\item Skizzieren Sie den Sprung von der Brücke. Welche Energien kennen Sie schon?
		\item Welche Geschwindigkeit erreicht der Springer bevor das Seil gedehnt wird?
		\item Berechnen Sie die Federkonstante $D$ des Bungeeseils. Nehmen Sie an, dass das Federgesetz gilt.
		\item Warum erreicht der Springer nicht nach 38 Metern seine grösste Geschwindigkeit?
		%\item Wo erreicht der Springer seine maximale Geschwindigkeit? Nach wie vielen Metern ist das?
		%	Wie hoch ist die maximale Geschwindigkeit?
			%Welche maximale Geschwindigkeit hat der Springer während des Sprungs?
	\end{enumerate}
\end{aufgabe}

