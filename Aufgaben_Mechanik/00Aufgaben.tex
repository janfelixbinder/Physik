%\documentclass[11pt,a4paper,titlepage,twoside]{article}
\documentclass[12pt,a4paper,twoside]{article}

%diese sind neu
\usepackage{style}
\usepackage{aufgaben}

\usepackage{gplot}
\usepackage{foto_v001}
\usepackage{mechanik_v001}

%\author{Felix Binder}
\title{Aufgabensammlung Mechanik}
\date{}


\begin{document}
\maketitle

\section{Einheiten}
\Einbinden{\dir/einheiten01.tex}
\Einbinden{\dir/einheiten02.tex}
\Einbinden{\dir/einheiten03.tex}
\Einbinden{\dir/einheiten04.tex}
\Einbinden{\dir/einheiten05.tex}
\Einbinden{\dir/einheiten06.tex}
\Einbinden{\dir/einheiten07.tex}
\Einbinden{\dir/einheiten08.tex}

\section{Dichte}
\Einbinden{\dir/dichte01.tex}
\Einbinden{\dir/dichte02.tex}
\Einbinden{\dir/dichte03.tex}
\Einbinden{\dir/dichte04.tex}


\section{Geschwindigkeit}
\Einbinden{\dir/geschwindigkeit00a.tex}
\Einbinden{\dir/geschwindigkeit00b.tex}
\Einbinden{\dir/geschwindigkeit01.tex}
\Einbinden{\dir/geschwindigkeit02.tex}
\Einbinden{\dir/geschwindigkeit03.tex}
\Einbinden{\dir/geschwindigkeit04.tex}
\Einbinden{\dir/geschwindigkeit05.tex}
\Einbinden{\dir/geschwindigkeit06.tex}
\Einbinden{\dir/geschwindigkeit07.tex}

\section{Beschleunigung}
\Einbinden{\dir/beschleunigung01.tex}
\Einbinden{\dir/beschleunigung02.tex}
\Einbinden{\dir/kinematik01.tex}

\section{Kreisbewegung}
Die Aufgaben \ref{rasenmaeher}, \ref{flugzeugturbine} und \ref{elektromotor} sind alle vom gleichen Typ.
Die Schwierigkeit besteht darin die Frequenz der Kreisbewegung als gegebene Grösse zu erkennen.
Ist dieses Hindernis überwunden muss die Winkelgeschwindigkeit berechnet werden und im Anschluss die Bahngeschwindigkeit.

\Einbinden{\dir/kreis01.tex}
\Einbinden{\dir/kreis02.tex}
\Einbinden{\dir/kreis03.tex}

\section{Bewegung in drei Raumrichtungen}

\Einbinden{\dir/beschleunigung03.tex}
\Einbinden{\dir/kinematik02.tex}


\Einbinden{\dir/bew3d01.tex}
\Einbinden{\dir/bew3d02.tex}

\section{Gravitation}
\Einbinden{\dir/gravitation01.tex}
\Einbinden{\dir/gravitation02.tex}

\section{Federgesetz}
\Einbinden{\dir/federgesetz01.tex}
\Einbinden{\dir/federgesetz02.tex}
\Einbinden{\dir/federgesetz03.tex}

\section{Vektoren}
\Einbinden{\dir/vektoren01.tex}
\Einbinden{\dir/vektoren02.tex}
\Einbinden{\dir/vektoren03.tex}
\Einbinden{\dir/vektoren04.tex}
\Einbinden{\dir/vektoren05.tex}
\Einbinden{\dir/vektoren06.tex}
\Einbinden{\dir/vektoren07.tex}
\Einbinden{\dir/vektoren08.tex}
\Einbinden{\dir/vektoren09.tex}
\Einbinden{\dir/vektoren10.tex}
\Einbinden{\dir/vektoren11.tex}
\Einbinden{\dir/statik02.tex}

\section{Drehmomente}
\Einbinden{\dir/drehmomente01.tex}
\Einbinden{\dir/drehmomente02.tex}
\Einbinden{\dir/drehmomente03.tex}
\Einbinden{\dir/drehmomente04.tex}
\Einbinden{\dir/drehmomente05.tex}
\Einbinden{\dir/statik01.tex}

\section{Schwerpunkt}
\Einbinden{\dir/schwerpunkt01.tex}
\Einbinden{\dir/schwerpunkt02.tex}
\Einbinden{\dir/schwerpunkt03.tex}
\Einbinden{\dir/schwerpunkt04.tex}
\Einbinden{\dir/statik03.tex}




\section{Lösen von Statikproblemen mit Hilfe des Drehmomentes}

\Einbinden{\dir/loesen01.tex}
\Einbinden{\dir/loesen02.tex}

\Einbinden{\dir/statik_drehmomente_bruecke01.tex}
\Einbinden{\dir/statik_drehmomente_bruecke02.tex}

\section{Reibung}
\Einbinden{\dir/reibung01.tex}
\Einbinden{\dir/reibung02.tex}
\Einbinden{\dir/reibung03.tex}
\Einbinden{\dir/reibung04.tex}
\Einbinden{\dir/reibung05.tex}
\Einbinden{\dir/reibung06.tex}
\Einbinden{\dir/reibung07.tex}

\section{Schiefe Ebene}
\Einbinden{\dir/schiefeEbene01.tex}
\Einbinden{\dir/dynamik05.tex}


\section{Newton'sche Axiome}
\Einbinden{\dir/dynamik01.tex}
\Einbinden{\dir/dynamik02.tex}
\Einbinden{\dir/dynamik04.tex}
\Einbinden{\dir/newton01.tex}
\Einbinden{\dir/newton02.tex}
\Einbinden{\dir/newton03.tex}

\section{Kreisbewegungen}
\Einbinden{\dir/kreisbewegung01.tex}
\Einbinden{\dir/kreisbewegung_eimer.tex}
\Einbinden{\dir/kreisbewegung_reibung.tex}
\Einbinden{\dir/kreisbewegung_kanone.tex}
\Einbinden{\dir/dynamik03.tex}
\Einbinden{\dir/hammerwurf.tex}

\section{Wechselwirkungsgesetz}
\Einbinden{\dir/wechselwirkung01.tex}
\Einbinden{\dir/wechselwirkung02.tex}

\section{Impuls}
\Einbinden{\dir/impuls01.tex}


\section{Arbeit und Energie}
\Einbinden{\dir/arbeit_schrank.tex}
\Einbinden{\dir/arbeit_schlitten.tex}
\Einbinden{\dir/arbeit_hub.tex}
\Einbinden{\dir/Ekin_auto.tex}
\Einbinden{\dir/arbeit_feder01.tex}
\Einbinden{\dir/arbeit_feder02.tex}
\Einbinden{\dir/arbeit_feder03.tex}


\section{Energieerhaltung}
\Einbinden{./arbeit01.tex}
\Einbinden{./arbeit02.tex}
\Einbinden{./arbeit03.tex}
\Einbinden{./arbeit04.tex}
\Einbinden{./energieerhaltung_kugelbahn.tex}

\section{Leistung}
\Einbinden{\dir/leistung01.tex}
\Einbinden{\dir/leistung02.tex}
\Einbinden{\dir/leistung03.tex}
\Einbinden{\dir/leistung04.tex}
\Einbinden{\dir/leistung05.tex}
\Einbinden{./arbeit05.tex}
\Einbinden{\dir/leistung_velo.tex}

\end{document}
