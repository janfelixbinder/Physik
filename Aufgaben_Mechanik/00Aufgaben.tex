%\documentclass[11pt,a4paper,titlepage,twoside]{article}
\documentclass[12pt,a4paper,twoside]{article}

%diese sind neu
\usepackage{style}
\usepackage{aufgaben}

\usepackage{gplot}
\usepackage{foto_v001}
\usepackage{mechanik_v001}



%\author{Felix Binder}
\title{Aufgabensammlung Mechanik}
\date{}


\begin{document}
\maketitle

\section{Reibung}
\Einbinden{./reibung01.tex}
\Einbinden{./reibung02.tex}
\Einbinden{./reibung03.tex}
\Einbinden{./reibung04.tex}


\section{Lösen von Statikproblemen mit Hilfe des Drehmomentes}
\Einbinden{./statik_drehmomente_bruecke01.tex}
\Einbinden{./statik_drehmomente_bruecke02.tex}






\section{Kreisbewegungen}
\Einbinden{./kreisbewegung01.tex}
\Einbinden{\dir/kreisbewegung_eimer.tex}
\Einbinden{\dir/kreisbewegung_reibung.tex}
\Einbinden{\dir/kreisbewegung_kanone.tex}
\Einbinden{./hammerwurf.tex}

\section{Arbeit und Energie}
\Einbinden{\dir/arbeit_schrank.tex}
\Einbinden{\dir/arbeit_schlitten.tex}
\Einbinden{\dir/arbeit_hub.tex}
\Einbinden{\dir/Ekin_auto.tex}
\Einbinden{\dir/arbeit_feder01.tex}
\Einbinden{\dir/arbeit_feder02.tex}
\Einbinden{\dir/arbeit_feder03.tex}


\section{Energieerhaltung}
\Einbinden{./energieerhaltung_kugelbahn.tex}


\end{document}
