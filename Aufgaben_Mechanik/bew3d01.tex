
\begin{aufgabe}
Ein Kaugummi wird horizontal aus dem Fenster ($h=\SI{3}{m}$) eines fahrenden Zuges ($v_{Zug}=\SI{120}{km/h}$) gespuckt.
Die Spuckgeschwindigkeit ist \SI{3}{m/s}.
Wo fällt das Kaugummi zu Boden?


\begin{loesung}
Zuerst berechnen wir das Fallen des Kaugummi im Schwerefeld der Erde.
Es gilt: $s=v_{0z}\cdot t + \frac{1}{2}\cdot a\cdot t^2$. Horizontal meint $v_0z=0$.
$\to t=\sqrt{\frac{2\cdot s}{a}}=\sqrt{\frac{2\cdot \SI{3}{m}}{\SI{10}{m/s^2}}}=\SI{0.77}{s}$
Das Kaugummi hat die gleiche Geschwindigkeit wie der Zug $v_x=\SI{120}{km/h}=\SI{33.3}{m/s}$.
$\to s_x=v_x\cdot t= \SI{33.3}{m/s}\cdot\SI{0.77}{s}=\SI{25.7}{m}$.
    $s_y=v_y\cdot t =\SI{3}{m/s}\cdot \SI{0.77}{s}=\SI{2.3}{m}$

\end{loesung}
\end{aufgabe}
