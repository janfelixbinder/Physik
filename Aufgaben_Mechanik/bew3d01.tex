
\begin{aufgabe}
	Ein Kaugummi wird horizontal aus dem Fenster ($h=\SI{3}{m}$) eines fahrenden Zuges ($\RI{v}{Zug}=\SI{120}{km/h}$) gespuckt.
Die Spuckgeschwindigkeit ist \SI{3}{m/s}.
\begin{enumerate}[a)]
	\item Wo fällt das Kaugummi zu Boden?
	\item Wohin fällt das Kaugummi aus der Sicht des Kaugummispuckes?
		Begründen Sie Ihre Antwort.
\end{enumerate}


\kloesung{$s_x=\SI{26.07}{m}$ und $s_y=\SI{2.3462}{m}$}

\begin{loesung}
	\begin{enumerate}[a)]
		\item 
Zuerst berechnen wir das Fallen des Kaugummi im Schwerefeld der Erde.
Es gilt: ($v_{0z}$ die Anfangsgeschwindigkeit in Richtung Boden ist Null.
\begin{eqnarray*}
s=v_{0z}\cdot t + \frac{1}{2}\cdot a\cdot t^2 \to t=\sqrt{\frac{2\cdot s}{a}}=\sqrt{\frac{2\cdot \SI{3}{m}}{\SI{9.81}{m/s^2}}}=\SI{0.78}{s}
\end{eqnarray*}

Das Kaugummi hat die gleiche Geschwindigkeit wie der Zug $v_x=\SI{120}{km/h}=\SI{33.3}{m/s}$.
\begin{eqnarray*}
	\begin{split}
    s_x=v_x\cdot t= \SI{33.3}{m/s}\cdot\SI{0.78}{s}=\SI{26.069}{m}\\
    s_y=v_y\cdot t =\SI{3}{m/s}\cdot \SI{0.78}{s}=\SI{2.3462}{m}
	\end{split}
\end{eqnarray*}
\item Aus Sicht des Spuckers fällt das Kaugummi senkrecht zu Boden. 
	\end{enumerate}

\end{loesung}
\end{aufgabe}
