
\begin{center}
\begin{tikzpicture}

\draw (-8,0)--(8,0);%Boden

\StativDrehscheibe{(5,0)}{6}{StR}
\StativDrehscheibe{(-5,0)}{5}{StL}

\coordinate (SP) at (1,4);%Schnittpunkt der drei angreifenden Kraefte
\draw [fill] (SP) circle (0.05cm);
\draw (SP)--++(-90:3cm) node [shape=coordinate] (EM){};
\Masse{(EM)}{0.3}{MM}
\draw (MM O) node [right] {$m_2$};

%rechte und linke masse
\draw (StR O)--++(-90:2cm) node [shape=coordinate] (EO){} (StR NW)--(SP);
\draw (StL W)--++(-90:2cm) node [shape=coordinate] (EW){} (StL N) --(SP);
\Masse{(EO)}{0.3}{MO}
\Masse{(EW)}{0.3}{MW}
\draw (MW W) node [left] {$m_1$};
\draw (MO O) node [right] {$m_3$};

\end{tikzpicture}
\end{center}

\begin{aufgabe}
	\begin{enumerate} [a)]
		\item Zeichnen Sie die Gewichtskräfte $F_1'$, $F_2'$ und $F_3'$ der Massen $m_1$, $m_2$ und $m_3$ ein.
		\item Zeichnen Sie die Fadenkräfte $F_1$, $F_2$ und $F_3$ ein, die am Knotenpunkt angreifen.
		\item Konstruieren Sie ein Kräfteparallelogramm der Kräfte $F_1$ und $F_3$, so dass die resultierende Kraft senkrecht nach oben zeigt.
		\item Die Kraft $F_1$ sei \SI{4.5}{N}. Bestimmen Sie die Beträge der Kräfte $F_2$ und $F_3$.
	\end{enumerate}
\end{aufgabe}
