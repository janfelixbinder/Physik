
\begin{aufgabe}
	Ein \SI{50}{kg} schwerer Junge steht im Schwimmbad auf dem Sprungbrett \SI{20}{cm} vom Ende des Brettes entfernt.
	Das Sprungbrett ist drei Meter lang und hat eine Masse von \SI{30}{Kilogramm}.
	Das Sprungbrett liegt vorne und bei einem Meter auf.
	\begin{enumerate} [a)]
		\item Machen Sie sich eine Skizze und zeichnen Sie alle Kräfte ein.
		\item Wählen Sie eine Drehachse und stellen sie die Gleichgewichtsbedingungen auf.
		\item Lösen Sie das Gleichungssystem. Welche Kräfte treten an den Auflagestellen auf?
	\end{enumerate}
	\begin{loesung}
		\begin{enumerate} [a)]
			\item Eine Skizze des Sprungbretts mit allen wirkenden Kräften könnte so aussehen.
		\begin{center}
			\begin{tikzpicture}
				\draw (0,0)--(6,0);
\draw [<-,Kraft] (5.6,0)--++(90:2cm) node [right] {$F=\SI{500}{N}$};%gewichtskraft des jungen
\draw [->,Kraft] (3,0)--++(-90:2cm) node [right] {$F=\SI{300}{N}$};%gewichtskraft brett
\draw [<-,Kraft] (2,0)--++(-90:2cm) node [left] {$F_2$};%stuetze
\draw [<-,Kraft] (0,0)--++(-90:2cm) node [left] {$F_1$};%stuetze
			\end{tikzpicture}
		\end{center}
		\item Um die Kräfte $F_1$ und $F_2$ ermitteln zu können, stellen wir die Summe der Drehmomente auf.
			Ich wähle als erste Drehachse den Punkt an den die Kraft $F_1$ angreift.
			\begin{eqnarray*}
				F_2\cdot\SI{1}{m} -\SI{300}{N}\cdot\SI{1.5}{m} - \SI{500}{N}\cdot\SI{2.8}{m}=0\to F_2=\SI{1850}{N}
			\end{eqnarray*}
			Um die Kraft $F_1$ zu bestimmen wähle ich als Drehachse den Punkt an den die Kraft $F_2$ angreift.
			\begin{eqnarray*}
				-F_1\cdot\SI{1}{m} - \SI{300}{N}\cdot\SI{0.5}{m} -\SI{500}{N}\cdot\SI{1.8}{m}=0\to F_1=-\SI{1050}{N}
			\end{eqnarray*}
			Die Kraft $F_1$ ist negativ. Das heisst, unsere Annahme $F_1$ zeigt nach oben ist falsch. $F_1$ zeigt also nach unten.
			Abschliessend noch einmal die Kräfteverteilung am Sprungbrett:
		\begin{center}
			\begin{tikzpicture}
				\draw (0,0)--(6,0);
\draw [<-,Kraft] (5.6,0)--++(90:2cm) node [right] {$F=\SI{500}{N}$};%gewichtskraft des jungen
\draw [->,Kraft] (3,0)--++(-90:2cm) node [right] {$F=\SI{300}{N}$};%gewichtskraft brett
\draw [<-,Kraft] (2,0)--++(-90:2cm) node [left] {\SI{1850}{N}};%stuetze
\draw [->,Kraft] (0,0)--++(-90:2cm) node [left] {\SI{1050}{N}};%stuetze
			\end{tikzpicture}
		\end{center}
		\end{enumerate}
	\end{loesung}
	\kloesung{c) \SI{1050}{N} und \SI{1850}{N}.}
\end{aufgabe}

