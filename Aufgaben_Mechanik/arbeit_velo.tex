
\begin{aufgabe}
	Ein Velofahrer (\SI{70}{kg}) möchte aus dem Stand auf \SI{10}{km/h} beschleunigen.
	Wie viel muss er dafür arbeiten?
	Nun möchte er weiter beschleunigen, um auf eine Geschwindigkeit von \SI{20}{km/h} zu kommen.
	Wie viel muss er diesmal arbeiten?
	Entspricht das Ihrer Erfahrung?
	\kloesung{\SI{270}{J}, \SI{810}{J}}
\end{aufgabe}

