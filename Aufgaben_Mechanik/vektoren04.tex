
\begin{aufgabe}
	\label{Hund}
	Drei Hunde ziehen einen Hundeschlitten. Der erste Hund zieht mit \SI{37}{N}, 
	der zweite Hund mit \SI{27}{N} unter einem Winkel von \SI{-17}{\degree}. 
	Der dritte Hund zieht mit einer Kraft von \SI{32}{N} und einem Winkel von \SI{12}{\degree} (siehe Zeichnung).
In welche Richtung und mit welcher Kraft wird der Schlitten gezogen. Lösen Sie graphisch.
\begin{center}
	
\begin{tikzpicture}

\draw [->, Kraft] (0,0) -- +(0:3.7cm) node [right] {$F_1$};
\draw [->, Kraft] (0,0) -- +(-17:2.7cm) node [right] {$F_2$};
\draw [->, Kraft] (0,0) -- +(12:3.2cm) node [right] {$F_3$};
\AP{(0,0)}

\end{tikzpicture}
\end{center}


\begin{loesung}
Durch verschieben der Vektoren bekommen wir den resultierenden Vektor.
Seine Länge und sein Winkel können gemessen werden.


\begin{tikzpicture}

\draw [->, Kraft,yshift=-2cm,color=black!30] (0,0) -- +(0:3.7cm) node [midway, above] {$F_1$};
\draw [->, Kraft,yshift=-2cm,color=black!30] (0,0) ++(0:3.7cm) -- ++(-17:2.7cm) node [midway, above] {$F_2$};
\draw [->, Kraft,yshift=-2cm,color=black!30] (0,0) ++(0:3.7cm)++(-17:2.7cm)--++(12:3.2cm) node [midway,above] {$F_3$} node (E) {};

\draw [->,Kraft,yshift=-2cm] (0,0)--(E.center) node [midway, above] {$F_1+F_2+F_3$};

\end{tikzpicture}


\vspace*{1cm}

\end{loesung}
\end{aufgabe}

