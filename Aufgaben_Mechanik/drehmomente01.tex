
\begin{aufgabe}
	Was beobachten Sie an diesen Wippen?
\end{aufgabe}

\begin{center}
\begin{tikzpicture}
	\begin{scope} [xshift=-5cm]
	\Wippe{0}{0}{0}
	\end{scope}
	\begin{scope}[xshift=5cm]
	\Wippe{39}{0}{0}
	\end{scope}
\end{tikzpicture}
\end{center}

\begin{aufgabe}
	Warum verändert sich die Stellung der Wippe?
\end{aufgabe}
\begin{center}
	\begin{tikzpicture}
		\Wippe{-29}{2}{0}
		\draw (Pr) --+(-90:1cm) node [shape=coordinate] (E) {};
		\Masse{(E)}{0.5}{M1}
		\draw (M1 O) node [right] {$m_1$};
	\end{tikzpicture}
\end{center}
\newpage
\begin{aufgabe}
Was sieht man hier?	
\end{aufgabe}
\begin{center}
	\begin{tikzpicture}
		\Wippe{0}{2.5}{-2.5}
		\draw (Pr) --+(-90:1cm) node [shape=coordinate] (Er) {};
		\draw (Pl) --+(-90:1cm) node [shape=coordinate] (El) {};
		\Masse{(Er)}{0.5}{Mr}
		\Masse{(El)}{0.5}{Ml}
		\draw (Mr O) node [right] {$m_1$};
		\draw (Ml W) node [left] {$m_1$};
	\end{tikzpicture}
\end{center}

\begin{aufgabe}
Was hat sich hier verändert?	
\end{aufgabe}
\begin{center}
	\begin{tikzpicture}
		\Wippe{20}{2.5}{-4.0}
		\draw (Pr) --+(-90:1cm) node [shape=coordinate] (Er) {};
		\draw (Pl) --+(-90:1cm) node [shape=coordinate] (El) {};
		\Masse{(Er)}{0.5}{Mr}
		\Masse{(El)}{0.5}{Ml}
		\draw (Mr O) node [right] {$m_1$};
		\draw (Ml W) node [left] {$m_1$};
	\end{tikzpicture}
\end{center}

\begin{aufgabe}
Was hat sich hier verändert?	
\end{aufgabe}
\begin{center}
	\begin{tikzpicture}
		\Wippe{0}{2.5}{-4.0}
		\draw (Pr) --+(-90:1cm) node [shape=coordinate] (Er) {};
		\draw (Pl) --+(-90:1cm) node [shape=coordinate] (El) {};
		\Masse{(Er)}{0.5}{Mr}
		\Masse{(El)}{0.5}{Ml}
		\draw (Mr O) node [right] {$m_1$};
		\draw (Ml W) node [left] {$m_2$};
	\end{tikzpicture}
\end{center}

\begin{aufgabe}
	Betrachen Sie die obige Abbildung. Wenn die Masse $m_1$ \SI{200}{g} hat, wie gross ist dann die Masse $m_2$?
\end{aufgabe}

