\documentclass{article}

\usepackage{pgf}
\usepackage{pgfplots}
%\usepackage{tikz}
\usepackage{gnuplot-lua-tikz}
\pgfplotsset{compat=1.9}

\def\folder{/tmp/auswertung}

\begin{document}
Man benötigt das octave Script ``auswertung.m''. Dieses erstellt eine Datei namens ``hist.dat``. 
Diese Datei wird dann vom gnuplot Skript auswertung.gplot in eine tikz Datei gewandelt.

\section{Histogramm über die Noten der SuS}

\input{\folder/histogramm.tex}

\section{Mittlere Anzahl Punkte pro Aufgabe}
\input{\folder/mittelwertaufgaben.tex}

\section{Wie korrelierten die einzelnen Aufgaben mit der Summe der Punktzahl der anderen Aufgaben?}
\input{\folder/corr1.tex}
\input{\folder/corr2.tex}
\input{\folder/corr3.tex}
\input{\folder/corr4.tex}
\input{\folder/corr5.tex}

\end{document}
