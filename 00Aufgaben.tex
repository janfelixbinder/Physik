%\documentclass[11pt,a4paper,titlepage,twoside]{article}
\documentclass[12pt,a4paper,twoside]{article}
\usepackage{mystyle}
\usepackage{gplot}
\usepackage{foto_v001}
\usepackage{mechanik_v001}

\usepackage{url}

\newcommand{\TAX}[1]{\textit{\small{#1}}}

%\author{Felix Binder}
\title{Aufgabensammlung Mechanik}
\date{}

\def\dir{./}
\newcommand{\Einbinden}[1]{\par Sie finden diese Aufgabe: \url{#1}.\par\input{#1}}

\begin{document}
\maketitle

\section{Reibung}
\Einbinden{./reibung01.tex}
\Einbinden{./reibung02.tex}
\Einbinden{./reibung03.tex}
\Einbinden{./reibung04.tex}


\section{Lösen von Statikproblemen mit Hilfe des Drehmomentes}
\Einbinden{./statik_drehmomente_bruecke01.tex}
\newpage
\Einbinden{./statik_drehmomente_bruecke02.tex}






\newpage
\section{Kreisbewegungen}
\Einbinden{./kreisbewegung01.tex}\newpage
\Einbinden{\dir/kreisbewegung_eimer.tex}\newpage
\Einbinden{\dir/kreisbewegung_reibung.tex}\newpage
\Einbinden{\dir/kreisbewegung_kanone.tex}
\Einbinden{./hammerwurf.tex}\newpage

\section{Arbeit und Energie}
\Einbinden{\dir/arbeit_schrank.tex}\newpage
\Einbinden{\dir/arbeit_schlitten.tex}\newpage
\Einbinden{\dir/arbeit_hub.tex}\newpage
\Einbinden{\dir/Ekin_auto.tex}\newpage
\Einbinden{\dir/arbeit_feder01.tex}\newpage
\Einbinden{\dir/arbeit_feder02.tex}\newpage
\Einbinden{\dir/arbeit_feder03.tex}\newpage


\section{Energieerhaltung}
\Einbinden{./energieerhaltung_kugelbahn.tex}


\newpage
\includesolutions
\end{document}
