
\section*{Auftrieb und archimedisches Prinzip}
Wenn ein schwerer Körper an einer Feder aufgehängt und in Wasser eingetaucht wird, dann
zeigt die Skala an der Federwaage eine geringere Gewichtskraft an, als wenn der Körper
in Luft gewogen würde. Ursache dafür ist eine nach oben gerichtete Kraft, die von dem Wasser
auf den Körper ausgeübt wird und einen Teil der Gewichtskraft kompensiert. Dieses Phänomen
heisst Auftrieb. Der Auftrieb ist noch besser sichtbar, wenn man beispielsweise einen
Korken in Wasser eintaucht. Wenn er vollständig in Wasser eingetaucht ist, dann erfährt 
der Korken aufgrund des Wasserdrucks eine aufwärts gerichtete Kraft, die grösser ist
als seine Gewichtskraft. Diese Kraft, die das Fluid auf einen ganz oder teilweise eingetauchten
Körper ausübt, wird als Auftriebskraft $F_{\textrm A}$ bezeichnet. Sie hängt nicht von der Form
oder Dichte des Körpers ab, sondern nur von der Dichte des Fluids. Ihr Betrag ist gleich Gewichtskraft der
durch den Körper verdrängten Fluidmenge.

\begin{cbox}
Ein Körper, der ganz oder teilweise in ein Fluid eintaucht, erfährt eine Auftriebskraft,
deren Betrag gleich der Gewichtskraft der durch den Körper verdrängten Fluidmenge ist.

\begin{gather*}
	\text{Auftriebskraft} = \text {Fluiddichte} \cdot \text{Fallbeschleunigung}\cdot\text{Volumen}\\
	\RI{F}{A} = \RI{\rho}{w} \cdot g \cdot V
\end{gather*}
\end{cbox}

\begin{aufgabe}
	Was versteht man unter Auftrieb? Wovon hängt der Auftrieb ab?
	\begin{loesung}
		Auftrieb tritt in Flüssigkeiten und Gasen auf und beschreibt eine nach oben gerichtete Kraft.
		Die Auftriebskraft hängt von der Dichte des Fluids und vom Volumen des eingetauchten Körpers ab.
	\end{loesung}
\end{aufgabe}

\begin{aufgabe}
	Wie gross ist der Auftrieb den ein Körper mit einem Volumen von \SI{0.25}{m^3} in Wasser erfährt?
	\kloesung{\SI{2452.5}{N}}
	\begin{loesung}
		\begin{eqnarray*}
			\RI{F}{A} = \RI{\rho}{w} \cdot g \cdot V=\SI{1000}{kg/m^3}\cdot\SI{9.81}{m/s^2}\cdot\SI{0.25}{m^3}=\SI{2452.5}{N}
		\end{eqnarray*}
	\end{loesung}
\end{aufgabe}

\begin{aufgabe}
	Ein Taucher ist im Urlaub am Toten Meer ($\rho=\SI{1240}{kg/m^3}$) und möchte tauchen gehen. 
	Daheim taucht er gewöhnlich in einem Süsswassersee.
	\begin{enumerate} [a)]
		\item Was sollte der Taucher beachten?
		\item Wie viel grösser ist der Auftrieb im Toten Meer im Vergleich zum heimischen See?
	\end{enumerate}
	\kloesung{b) \num{1.24}}
	\begin{loesung}
		\begin{enumerate} [a)]
		\item Die Dichte des Wassers im Toten Meer ist höher als im Süsswassersee.
			Dadurch ist die Auftriebskraft grösser.
		\item
			\begin{eqnarray*}
				\frac{\RI{F}{A}(\textrm{Totes Meer})}{\RI{F}{A}(\textrm{Süsswassersee})}=\frac{\RI{\rho}{w}(\textrm{Totes Meer})\cdot g\cdot V}{\RI{\rho}{w}(\textrm{Süsswasser})\cdot g\cdot V}=\frac{\RI{\rho}{w}(\textrm{Totes Meer})}{\RI{\rho}{w}(\textrm{Süsswasser})}=\frac{\SI{1240}{kg/m^3}}{\SI{1000}{kg/m^3}}=\num{1.24}
			\end{eqnarray*}
		\end{enumerate}
	\end{loesung}

\end{aufgabe}


\newpage

\subsection*{Die Entdeckung des archimedischen Prinzips}
Quelle: \url{http://de.wikipedia.org/wiki/Archimedisches_Prinzip}

Archimedes war von König Hieron II. von Syrakus beauftragt worden, herauszufinden, 
ob dessen Krone wie bestellt aus reinem Gold wäre oder ob das Material durch billigeres Metall gestreckt worden sei. 
Diese Aufgabe stellte Archimedes zunächst vor Probleme, da die Krone natürlich nicht zerstört werden durfte.

Der Überlieferung nach hatte Archimedes schließlich den rettenden Einfall, 
als er zum Baden in eine bis zum Rand gefüllte Wanne stieg und dabei das Wasser überlief. 
Er erkannte, dass die Menge Wasser, die übergelaufen war, genau seinem Körpervolumen entsprach. 
Angeblich lief er dann, nackt wie er war, durch die Straßen und rief Heureka (``Ich habe es gefunden'').

Um die gestellte Aufgabe zu lösen, tauchte er einmal die Krone und dann einen Goldbarren, 
der genauso viel wog wie die Krone, in einen bis zum Rand gefüllten Wasserbehälter und maß die Menge des überlaufenden Wassers. 
Da die Krone mehr Wasser verdrängte als der Goldbarren und somit bei gleichem Gewicht voluminöser war,
musste sie aus einem Material geringerer Dichte, also nicht aus reinem Gold, gefertigt worden sein.

Diese Geschichte wurde vom römischen Architekten Vitruv überliefert.

Obwohl der Legende nach auf dieser Geschichte die Entdeckung des archimedischen Prinzips beruht, 
würde der Versuch von Archimedes auch mit jeder anderen Flüssigkeit funktionieren. 
Das Interessanteste am archimedischen Prinzip, nämlich die Entstehung des Auftriebs und damit die Berechnung der Dichte des Fluids, 
spielt in dieser Entdeckungsgeschichte gar keine Rolle.

\begin{aufgabe}
	Lesen Sie den obigen Text zur Entdeckung des archimedischen Prinzips.
	\begin{enumerate} [a)]
		\item Was hat Archimedes gemacht, um die Echtheit der Goldkrone zu überprüfen.
        \item Was misst Archimedes eigentlich?
		\item Warum funktioniert dieses Verfahren?
		\item Welchen Nachteil hat diese Methode?
		\item Fällt ihnen ein alternatives Verfahren ein, um die Echtheit der Krone zu überprüfen?
	\end{enumerate}
	\begin{loesung}
		\begin{enumerate} [a)]
			\item
            \item Das Volumen der Krone, bzw. das Volumen des Goldbarren. 
			\item Die Dichte von Gold ist sehr hoch. Es gibt keine Materialien, die günstiger sind als Gold mit vergleichbarer Dichte.
			\item Man braucht zusätzlich Gold, mit der gleichen Masse der Krone.
			\item
		\end{enumerate}
	\end{loesung}
\end{aufgabe}

\subsection*{Dichtebestimmung mit der Auftriebskraft}
Am Auftrieb eines Körpers kann man eine erste Einschätzung seiner Dichte machen.
Geht der Körper im Fluid unter, so ist seine Dicht grösser als die Dichte des Fluids.
Beispiele sind der heliumgefüllte Luftballon der aufsteigt oder der Korken der auf der 
Wasseroberfläche schwimmt. In beiden Fällen ist die Dichte des Körpers kleiner als die
Dichte des Fluids. Der Stein, den man in den See wirft geht hingegen unter. Seine Dichte ist
höher als die Dichte des Wassers.


Man kann die Dichte eines Körpers mit Hilfe des Auftriebs aber auch genauer bestimmen.
Der Auftrieb eines Körpers kann relativ leicht mit einer Federwaage bestimmt werden. Zuerst bestimmt man das Gewicht (Gewichtskraft)
des Körpers in Luft ($F_{\textrm G}$). Dann hält man den Körper in die Flüssigkeit und bestimmt den Gewichtsverlust
des Körpers. Der Gewichtsverlust ist die Auftriebskraft.

Mit Gewichtskraft und Auftriebskraft kann man nun die Dichte des Körpers bestimmen:

\begin{eqnarray*}
	\frac{F_{\textrm G}}{F_{\textrm A}}=\frac{\rho\cdot g\cdot V}{\rho_{\textrm w}\cdot g\cdot V} \qquad\to\qquad\frac{F_{\textrm G}}{F_{\textrm A}}=\frac{\rho}{\rho_{\textrm w}}\qquad\to\qquad \rho=\rho_{\textrm w}\cdot\frac{F_{\textrm G}}{F_{\textrm A}} 
\end{eqnarray*}


\begin{aufgabe}
	Das Gewicht eines Körpers in Luft ist \SI{5}{N}. In Wasser getaucht zeigt der Kraftmesser \SI{4.75}{N} an.
	\begin{itemize}	
		\item[a)] Welche Dichte hat der Körper und aus welchem Material besteht er vermutlich?
		\item[b)] Wie gross ist das Volumen des Körpers?
	\end{itemize}	

	\kloesung{a) \SI{20000}{kg/m^3} vermutlich aus Gold, b) \SI{25}{cm^3}}

	\begin{loesung}
	\begin{itemize}	
		\item[a)]
			\begin{eqnarray*}
				F_{\textrm A}=\SI{5}{N}-\SI{4.75}{N}=\SI{0.25}{N}
			\end{eqnarray*}
			\begin{eqnarray*}
				\rho=\rho_{\textrm w}\cdot\frac{F_{\textrm G}}{F_{\textrm A}}=\SI{1000}{kg/m^3}\cdot\frac{\SI{5}{N}}{\SI{0.25}{N}}=\SI{20000}{kg/m^3} 
			\end{eqnarray*}
			Der Körper ist vermutlich aus Gold ($\rho_{\textrm{Gold}}=\SI{19290}{kg/m^3}$).
		\item[b)]
			\begin{eqnarray*}
				m=\rho\cdot V \to V=\frac{m}{\rho}=\frac{F_{\textrm G}}{g\cdot\rho} =\frac{\SI{5}{N}}{\SI{10}{N/kg}\cdot\SI{20000}{kg/m^3}}=\SI{2.5E5}{m^3}=\SI{25}{cm^3}
			\end{eqnarray*}
	\end{itemize}	
	\end{loesung}

\end{aufgabe}



\begin{aufgabe}
	Ein Korken schwimmt auf der Wasseroberfläche. Wie viel vom Korken ist unter der Wasseroberfläche?
	Die Dichte des Korkens ist $\RI{\rho}{Kork}=\SI{200}{kg/m^3}$.

	\kloesung{\SI{20}{\percent}}

	\begin{loesung}
		Der Korken ist im Gleichgewicht: $\RI{F}{A}=\RI{F}{G}$.
		Mit $V^*$ bezeichnen wir das Volumen des Körpers, dass unter der Wasseroberfläche ist.
		Dieses ist für den Auftrieb wichtig.
		\begin{eqnarray*}
			\frac{\RI{F}{A}}{\RI{F}{G}} = \frac{\RI{\rho}{w}\cdot g\cdot V^*}{\RI{\rho}{Kork}\cdot g\cdot V}=1\to\frac{V^*}{V}=\frac{\RI{\rho}{Kork}}{\RI{\rho}{w}}=\frac{\SI{200}{kg/m^3}}{\SI{1000}{kg/m^3}}=\frac{1}{5}
		\end{eqnarray*}
		\SI{20}{\%} des Korkvolumens ist unter der Wasseroberfläche.
	\end{loesung}

\end{aufgabe}



