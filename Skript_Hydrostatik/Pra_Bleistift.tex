\documentclass{beamer}%    Standardklasse für Beamer
%\documentclass[draft]{beamer}%  Entwickler-Modus in dem das File schneller kompiliert wird
%\documentclass[handout]{beamer}%    Zum Ausdrucken besser geeignete Version der Folien
%\documentclass{article}%Artikel-Version; Funktioniert nur, wenn man zusätzlich das Paket beamerarticle mittels \usepackage{beamerarticle} einbindet.
%\usepackage{beamerarticle}
\usepackage[german]{babel}
\usepackage[T1]{fontenc}
\usepackage[utf8]{inputenc}

\usepackage{foto_v001}


\usepackage{lmodern}

%\usetheme{Boadilla}
\usetheme{Berlin}

\setbeamercovered{transparent}
\beamertemplatenavigationsymbolsempty
\setbeamertemplate{footline}[frame number]

\title{Hochschulreife und Studierfähigkeit}

\author[Felix Binder]{Felix Binder}

\begin{document}

%\begin{frame}
%\titlepage	
%\end{frame}

%\section{Der Druck}
\begin{frame}
%	\frametitle{Bleistift}
Nehmen Sie einen Bleistift zwischen zwei Finger, so wie Sie es auf dem Foto sehen.
	\begin{center}%
	\Bildeinbinden{bleistift}{0.6}%
	\end{center}%
\begin{enumerate} [a)]
\item Was spüren Sie?
\item Wie erklären Sie sich das?
\item Machen Sie eine Skizze und zeichnen Sie die wirkenden Kräfte ein.
\end{enumerate}%
\end{frame}

\end{document}



