\section*{Der Luftdruck}

\begin{aufgabe}
	Lesen Sie den Ausschnitt zum Barometer aus der Wikipedia und beantworten Sie folgende Fragen:
	\begin{enumerate} [a)]
		\item Was ist ein Barometer?
		\item Welche Funktion hat das Barometer?
		\item Skizzieren Sie die geschichtliche Entwicklung des Barometers.
		\item Was wurde mit dem Barometer erforscht?
		\item Beschreiben Sie das Funktionsprinzip des Barometers.
		\item Was hat Ihnen besonders gut an dem Artikel gefallen?
	\end{enumerate}
\end{aufgabe}


\begin{aufgabe}
	In der Physiksammlung gibt es ein Quecksilberbarometer. Gehen Sie in kleinen Gruppen (maximal drei Personen)
	und schauen Sie es sich an. Lesen Sie die Höhe der Quecksilbersäule ab.
	Berechnen Sie am Arbeitsplatz den heutigen Luftdruck.
\end{aufgabe}

\begin{aufgabe}
	\label{konstanteDichte}
	Nehmen Sie an, die Dichte der Luft in der Atmosphäre ist nicht abhängig von der Höhe, 
	sondern konstant die Dichte bei Normalbedingungen.
	Bis wohin reicht die Atmosphäre in diesem Fall?
	
	\kloesung{\SI{7943}{m}}

	\begin{loesung}
	In dieser Aufgabe wird angenommen, dass die Luft nicht komprimiert werden kann.
	Die Luft soll sich also wie eine Flüssigkeit verhalten.
	\begin{eqnarray*}
		P=\rho\cdot g\cdot h\to h=\frac{P}{\rho\cdot g}=\frac{\SI{101300}{Pa}}{\SI{1.3}{kg/m^3}\cdot\SI{9.81}{m/s^2}}=\SI{7943.22}{m}
	\end{eqnarray*}
	\end{loesung}

\end{aufgabe}

\begin{aufgabe}
	In Aufgabe \ref{konstanteDichte} haben Sie angenommen, die Dichte der Luft sei über die gesamte Atmosphäre konstant.
	Verbessern Sie dieses Modell, indem Sie die Atmosphäre in \SI{1}{km} Dicke schichten konstanter Dichte zerlegen.
	\begin{enumerate}[a)]
		\item Welcher Luftdruck ergibt sich für \SI{5}{km} bzw.~\SI{10}{km} über dem Meer, wenn Sie von Normalbedingungen
			und konstanter Temperatur auf Meereshöhe ausgehen? 
		\item Vergleichen Sie die Werte aus a) mit der barometrischen Höhenformel:
			\begin{eqnarray*}
				P&=P_0\cdot e^{-\frac{\rho_0 g}{P_0}\cdot h}
			\end{eqnarray*}
	\end{enumerate}
	\kloesung{a) \SI{517}{hPa} und \SI{264}{hPa}, b) \SI{540}{hPa} und \SI{288}{hPa}}
	
\end{aufgabe}


%\begin{aufgabe}
%	An einem geschlossenes Quecksilbermanometer misst man einen Höhenunterschied von \SI{74.2}{cm}.
%	Welchem Druck entspricht das in Pascal?
%	\begin{loesung}
%	\begin{eqnarray*}
%		P=\rho_{\textrm Hg}\cdot g\cdot h = \SI{13546}{kg/m^3}\cdot \SI{9.81}{N/kg}\cdot\SI{0.742}{m}=\SI{98602}{Pa}=\SI{986}{hPa}
%	\end{eqnarray*}
%	\end{loesung}
%\end{aufgabe}

