\subsection*{Die Magdeburger Halbkugeln}
Im Jahre 1654 hat Otto von Guericke in Magdeburg (Deutschland) zum ersten Mal sein später berühmt gewordenes Experiment vorgeführt.
Es ist ein Experiment um die Wirkung des Luftdrucks zu verdeutlichen.
Zwei Halbkugeln werden so aneinander gelegt, dass sie eine Kugel formen.
Dann wird die Luft aus dem inneren der Kugel gepumpt. Der äussere Luftdruck drückt weiterhin auf die zwei Halbkugeln.
Dies verhindert, dass man die Halbkugeln voneinander trennen kann. Der Durchmesser der zwei Halbkugeln betrug damal \SI{42}{cm}
und 16 Pferde waren nicht in der Lage sie zu trennen.

\begin{center}
\Bildeinbinden{./83_Guericke_D.jpg}{0.9}
\end{center}

\begin{aufgabe}
	Machen Sie sich eine Skizze des Experiments und zeichnen Sie die wirkenden Kräfte ein.
Berechnen Sie, wie gross die Kräfte sein müssten um die zwei Halbkugeln zu trennen.

\begin{loesung}
	Die Fläche eines Kreises mit einem Durchmesser von \SI{42}{cm} ist:
	\begin{eqnarray*}
		A=r^2\cdot \pi=(\SI{0.21}{m})^2\cdot\pi=\SI{0.13854}{m^2}.
	\end{eqnarray*}
	\begin{eqnarray*}
		P=\frac{F}{A} \to F=P\cdot A=\SI{1E5}{Pa}\cdot\SI{0.139}{m^2}=\SI{13900}{N}
	\end{eqnarray*}
\end{loesung}
\end{aufgabe}

