
%\subsection*{Experiment: Schwebende Ballone}
\begin{aufgabe}
	Im Unterricht sehen Sie ein Experiment.
	Beantworten Sie dazu die folgenden Fragen.
\begin{enumerate} [a)]
	\item Sie sehen zwei mit Wasser gefüllte Bechergläser in denen jeweils ein Ballon schwebt.
	Warum schweben die Ballone?
\item Die Ballone werden nun vertauscht. Der rechte Ballon wird ins linke Gefäss und der linke Ballon ins rechte Gefäss gegeben.
	Machen Sie eine Skizze. Was beobachten Sie?
\item Wie können Sie sich das Beobachtete erklären?
\end{enumerate}
\end{aufgabe}



\begin{aufgabe}
	Ein Messbecher mit einem Volumen von \SI{200}{ml} 
	ist mit Wasser von \SI{4}{\celsius} bis zum Rand gefüllt ($\rho=\SI{1000}{kg/m^3}$).
	Das Wasser wird auf \SI{80}{\celsius} erwärmt. Dabei laufen \SI{6}{g} Wasser über.
	Nehmen Sie an, dass Volumen des Gefässes ändert sich durch die Temperaturänderung nicht.
	\begin{enumerate} [a)]
		\item Welche Masse hat das Wasser bei \SI{4}{\celsius}?
		\item Welche Dichte hat Wasser von \SI{80}{\celsius}?
	\end{enumerate}
	
	\kloesung{a) \SI{0.2}{kg}, b) \SI{970}{kg/m^3}}

	\begin{loesung}
		\begin{enumerate}[a)]
			\item Die Masse berechnet sich mit der Definition der Dichte:
				\begin{eqnarray*}
					\rho =\frac{m}{V}\to m=\rho\cdot V=\SI{1000}{kg/m^3}\cdot\SI{2E-4}{m^3}=\SI{2E-1}{kg}=\SI{0.2}{kg}
				\end{eqnarray*}
			\item Mit der Definition der Dichte bekommen wir
				\begin{eqnarray*}
					\rho =\frac{m}{V}=\frac{\SI{0.2}{kg}-\SI{0.006}{kg}}{\SI{2E-4}{m^3}}=\SI{970}{kg/m^3}
				\end{eqnarray*}
		\end{enumerate}
	\end{loesung}

\end{aufgabe}


\begin{aufgabe}
	Im Schwimmbad liegt eine Person auf einer Schwimmmatte.
	Die Matte ist zwei Meter lang, ein Meter breit und zehn Zentimeter hoch.
	Die Matte wird durch das Gewicht der Person um etwa \SI{3.5}{Zentimeter} unter Wasser gedrückt.
	Wie schwer ist die Person?
	
	\kloesung{etwa \SI{70}{kg}}
	
	\begin{loesung}
		Die Eintauchtiefe der Matte ist konstant \SI{3.5}{Zentimeter}. 
		Das bedeutet die Gewichtskraft und die Auftriebskraft sind bei diesem Wert im Gleichgewicht.
		\begin{eqnarray*}
			\begin{split}
		\RI{F}{G}&=\RI{F}{A}\\
		m\cdot g &=\RI{\rho}{w}\cdot V\cdot g\\
		m&=\RI{\rho}{w}\cdot V = \SI{1000}{kg/m^3}\cdot\SI{2}{m}\cdot\SI{1}{m}\cdot\SI{0.035}{m}=\SI{70}{kg}
			\end{split}
		\end{eqnarray*}
Das Volumen bei der Auftriebskraft ist das Volumen der Matte, das unter der Wasseroberfläche liegt.
	\end{loesung}
\end{aufgabe}


\begin{aufgabe}
	Während einer Lektion haben wir die Dichte von Luft bestimmt. Erinnern Sie sich noch?
	Ein Vorschlag der gemacht wurde war einen Luftballon mit und ohne Luft zu wiegen.
	Warum kann dieser Vorschlag nicht funktionieren?
\end{aufgabe}

\begin{aufgabe}
	Ein Eiswürfel schwimmt in einem Glas mit Wasser. 
	Wie viel von einem Eiswürfel ist über der Wasseroberfläche zu sehen?
	Die Dichte von Eis ist \SI{917}{kg/m^3}.

	\kloesung{etwa \SI{8}{\percent}}

	\begin{loesung}
		Der Eiswürfel ist im Gleichgewicht. Das heisst $\RI{F}{G}=\RI{F}{A}$.
		\begin{eqnarray*}
			1=\frac{\RI{F}{G}}{\RI{F}{A}}=\frac{\RI{\rho}{Eis}\cdot\RI{V}{Eis}\cdot g}{\RI{\rho}{Wasser}\cdot V^*\cdot g}
		\end{eqnarray*}
		$V^*$ ist das Volumen des Eiswürfels das sich unter Wasser befindet.
		Umstellen der oberen Formel nach $\nicefrac{V^*}{\RI{V}{Eis}}$ gibt uns den Anteil des Einswürfels der unter
		der Wasseroberfläche ist.
		\begin{eqnarray*}
			\frac{V^*}{\RI{V}{Eis}}=\frac{\RI{\rho}{Eis}}{\RI{\rho}{Wasser}} = \frac{\SI{917}{kg/m^3}}{\SI{1000}{kg/m^3}}=\num{0.917}\approx\SI{92}{\percent}
		\end{eqnarray*}
		Der Rest des Eiswürfels (\SI{8}{\percent}) ist also über der Wasseroberfläche.
	\end{loesung}
\end{aufgabe}


\begin{aufgabe}
	Ein Gefäss ist randvoll mit Wasser gefüllt. Auf dem Wasser schwimmt ein Stück Eis.
	Was passiert, wenn das Eis schmilzt? Erklären Sie.
\end{aufgabe}

\begin{aufgabe}
	Ein Liter Wasser ($\rho=\SI{1000}{kg/m^3}$) wird in einem sehr dünnen Gefäss gewogen. 
	Was zeigt die Waage an? Denken Sie bei dieser Aufgabe an den Auftrieb der Luft.

	\kloesung{\SI{0.9987}{kg}}

	\begin{loesung}
		Um den Auftrieb in Luft zu bestimmen braucht man die Dichte der Luft und das Volumen
		des Gegenstandes. Die Dichte ist etwa \SI{1.3}{kg/m^3}. Das Volumen ist $\SI{1}{l}=\SI{1}{dm^3}=\SI{1E-3}{m^3}$.
		Damit ergibt sich die Auftriebskraft zu
		\begin{eqnarray*}
			\RI{F}{A}=\rho\cdot g \cdot V=\SI{1.3}{kg/m^3}\cdot\SI{9.81}{m/s^2}\SI{1E-3}{m^3}=\SI{0.012753}{N}
		\end{eqnarray*}
		Mit der Formel für die Gewichtskraft bekommt man die Masse, die dieser Kraft entspricht
		\begin{eqnarray*}
			\RI{F}{G}=m\cdot g\to m=\frac{\RI{F}{G}}{g}=\frac{\SI{0.012753}{N}}{\SI{9.81}{m/s^2}}=\SI{0.0013}{kg}\text{.}
		\end{eqnarray*}
		Damit ergibt sich für die Anzeige
		\begin{eqnarray*}
			\SI{1}{kg}-\SI{0.0013}{kg}=\SI{0.9987}{kg}\text{.}
		\end{eqnarray*}
	\end{loesung}

\end{aufgabe}

%\begin{aufgabe}
%	Ein Heissluftballon mit einem Volumen von \SI{5000}{m^3} wird auf eine Temperatur von \SI{100}{\celsius} erwärmt.
%	Wie schwer kann der angehängten Korb sein, damit der Ballon anfängt zu schweben?
%\end{aufgabe}

