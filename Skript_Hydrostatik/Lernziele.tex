%\documentclass[11pt,a4paper,titlepage,twoside]{article}
\documentclass[12pt,a4paper,twoside]{article}
\usepackage{mystyle}
\usepackage{foto_v001}
\usepackage{gplot}
\usepackage{tabelle_v001}
\usepackage{mechanik_v001}
%\usepackage{import}

\usepackage{url}

%oben und unten
\usepackage{fancyhdr}
\pagestyle{fancy}
\lhead{}
\rhead{}
\rfoot{Felix Binder}
\lfoot{Hydrostatik Mai 2014}
\renewcommand\headrulewidth{0pt}
\renewcommand\footrulewidth{1pt}
%ende oben und unten

\date{}
%\author{Felix Binder}
\title{Hydrostatik\\{\large Die Physik von Flüssigkeiten und Gasen}}


\begin{document}
\maketitle

%\addtocounter{page}{5}
%\addtocounter{section}{9}
%\addtocounter{aufgabe}{27}

\section*{Lernziele}

\begin{itemize}
	\item Sie können den Hydrostatische Druck in Flüssigkeiten erklären und Berechnungen dazu durchführen.
	\item Sie können erklären auf welchem Prinzip offene und geschlossene Manometern funktionieren und abgelesene Höhenunterschiede in Drücke umrechnen.
	\item Sie verstehen warum der Luftdruck nicht linear mit der Höhe abnimmt 
		und können die Barometrische Höhenformel benutzen um den Luftdruck in 
		einer bestimmten Höhe zu berechnen. Ausserdem können Sie von einem bestimmten Luftdruck auf eine Höhe schliessen.
%	\item Sie können das Prinzip der Hebebühne erklären und die Kraftverstärkung einer hydraulische Hebebühne aus deren Geometrie bestimmen.
	\item Sie wissen wie man den Auftrieb in Flüssigkeiten messen kann und können den Auftrieb in Flüssigkeiten erklären.
	\item Sie können eine Grundvoraussetzung für das Schwimmen von Gegenständen benennen.
	\item Sie können die Auftriebskraft benutzen um die Dichte eines Gegenstand zu berechnen.
\end{itemize}

\end{document}
