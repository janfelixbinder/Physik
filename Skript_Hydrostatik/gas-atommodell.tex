%\documentclass[11pt,a4paper,titlepage,twoside]{article}
\documentclass[12pt,a5paper,landscape]{article}
\usepackage{mystyle}
\usepackage{foto_v001}
\usepackage{gplot}
\usepackage{tabelle_v001}
\usepackage{mechanik_v001}
%\usepackage{import}

\usepackage{url}

%oben und unten
\usepackage{fancyhdr}
\pagestyle{fancy}
\lhead{}
\rhead{}
\rfoot{Felix Binder}
\lfoot{Hydrostatik 2014}
\renewcommand\headrulewidth{0pt}
\renewcommand\footrulewidth{1pt}
%ende oben und unten

\date{}
%\author{Felix Binder}
\title{Hydrostatik\\{\large Die Physik von Flüssigkeiten und Gasen}}


\begin{document}
%\maketitle
\section*{Atomare Gastheorie}

Sie wissen ja sicher, dass Tische, Stühle und alles andere -- auch wir selbst -- aus Atomen besteht.
Auch die Luft (jedes Gas) besteht aus Atomen oder Molekülen.

Stellen Sie sich nun ein Gas vor, indem Sie die Atome (Moleküle) durch Bälle ersetzen.
Da wir von einem Gas sprechen, bewegen sich die Bälle mit grossen Geschwindigkeiten.
Stellen Sie sich vor, dieses Gas befindet sich in einer geschlossenen Box.
Trifft ein Ball auf eine der Wände der Box, wird er wie eine Billardkugel an der Wand reflektiert.
Die Richtung der Kugel ändert sich. Das heisst eine Kraft wirkt auf den Ball.

Druck ist definiert als Kraft pro Fläche. Das heisst das Gas übt einen Druck auf die Seiten der Box aus.

Überlegen sie sich, wie sich der Druck auf die Gefässwand ändert, 
\begin{itemize}
	\item wenn sich nur halb so viele Atome in der Box befinden.
	\item wenn sie doppelt so viele Atome in der Box haben.
\end{itemize}

Wie hat sich die Dichte des Gases geändert?
Wie hat sich das Verhältnis von Dichte und Druck geändert?
\end{document}
