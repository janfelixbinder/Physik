\subsection*{Hydraulische Systeme}
In täglichen Leben werden häufig hydraulische Systeme benutzt, um Kräfte zu verstärken.
Ein Beispiel ist der hydraulische Wagenheber. Die Abbildung zeigt zwei Kolben, die
mit einer Flüssigkeit verbunden sind.
Erhöht man den Druck in einer Flüssigkeit indem man mit einem Kolben eine Kraft auf die
Oberfläche ausübt, so erhöht sich der Druck gleichmässig in der gesamten Flüssigkeit.

\begin{minipage}{0.5\textwidth}
\begin{tikzpicture}[info text/.style={rounded corners, fill=red!20, inner sep=1ex}]
%\begin{tikzpicture}
%\usetikzlibrary{calc,intersections,through,backgrounds}
%\usetikzlibrary{decorations.pathmorphing}
%\draw[step=0.5cm,lightgray] (-0.5,-3.0) grid (6.5,5.0);

\path[draw,fill=blue!50,line width=0.1cm] (0,0)--(0,-6)--(6,-6)--(6,0)--(3,0)--(3,-5)--(1.5,-5)--(1.5,0)--cycle;
\draw [line width=0.1 cm] (0,0)--(0,1);
\draw [line width=0.1 cm] (1.5,0)--(1.5,1);
\draw [line width=0.1 cm] (6,0)--(6,1);
\draw [line width=0.1 cm] (3,0)--(3,1);

%zylinder klein
\path[draw, fill=red!50] (0.05,-0.5)--(0.05,2)--(1,2)--(1.45,2)--(1.45,-0.5)--cycle;

%zylinder gross
\path[draw, fill=red!50] (3.05,-0.25)--(3.05,2.25)--(5.95,2.25)--(5.95,-0.25)--cycle;

\draw[<-,Kraft] (0.75,2)--(0.75,3) node [above] {$\vec{F_1}$};
\draw[->,Kraft] (4.5,2.25)--(4.5,5) node [above] {$\vec{F_2}$};

\draw[<-,line width=0.05cm] (0.75,-0.5)--++(240:2) node [below] {$A_1$};
\draw[<-,line width=0.05cm] (4.5,-0.25)--++(-60:2) node [below] {$A_2$};



\end{tikzpicture}
\end{minipage}
\begin{minipage}{0.5\textwidth}
Wird auf den linken Kolben eine Kraft $F_1$ ausgeübt, dann steigt der Druck in der Flüssigkeit. Der Druckanstieg
ist
\begin{eqnarray*}
	P = \frac{F_1}{A_1}.
\end{eqnarray*}

Der Druck in der Flüssigkeit drückt nun auf den zweiten Kolben mit der Fläche $A_2$. Dies bewirkt eine Kraft $F_2$
\begin{align*}
	F_2 &= P\cdot A_2\\
	&=\frac{F_1}{A_1}\cdot A_2\\
	&=\frac{A_2}{A_1}\cdot F_1
\end{align*}

\end{minipage}
\begin{aufgabe}
	Die zwei Kolben eines Wagenhebers haben einen Durchmesser von \SI{2}{cm} und \SI{8}{cm}.
	Wie gross ist die Kraftverstärkung?

	\begin{loesung}
		\begin{gather*}
			A_1=r^2\cdot\pi=(\SI{0.01}{cm})^2\cdot \pi=\SI{0.0001}{m^2}\cdot \pi\\
			A_2=r^2\cdot\pi=(\SI{0.04}{cm})^2\cdot \pi=\SI{0.0016}{m^2}\cdot \pi\\
		\end{gather*}
		Die Kraftverstärkung ist $A_2/A_1=16$.
	\end{loesung}
\end{aufgabe}


