\usetikzlibrary{positioning}
\begin{center}
\begin{tikzpicture}
%	\node [rectangle, text width=20em](oben){Dichtetabelle\\Die Werte für Gase gelten bei \SI{0}{\celsius}};
	
	%\matrix (links) [below = of oben,Tabelle,text width=10em]
	\matrix (links) [left,Tabelle,text width=10em]
{
Aluminium & 2700 \\
Blei      & 11340\\
Eisen     &  7860\\
Gold      & 19290\\
Kupfer    & 8920 \\
Nickel    & 8900 \\
Platin    & 21450\\
Silber    & 10500\\
Zink      & 7140 \\
          &      \\
Beton     & ca. 2000 \\
Glas      & ca. 2600\\
Plexiglas & 1190\\
Messing   & ca. 8400\\
Stahl     & ca. 7900\\
Holz      & 500 - 700\\
Kork      & ca. 300\\
Eis (\SI{0}{\celsius} & 917 \\
};
%\matrix (rechts) [below= of oben, Tabelle,text width=10em]
\matrix (rechts) [right, Tabelle,text width=10em]
{
Äther     & 714 \\
Alkohol   & 789\\
Olivenöl  & 914\\
Petroleum & 850\\
Quecksilber  & 13546 \\
&  \\
Wasser \SI{0}{\celsius}    & \num{999.84}\\
Wasser \SI{4}{\celsius}    & \num{999.97}\\
Wasser \SI{100}{\celsius}    & \num{958.35}\\
    & \\
	Chlorgas      & \num{3.214} \\
	Helium        & \num{0.1768}\\
	Kohlendioxid  & \num{1.9769} \\
	Luft  & \num{1.2929} \\
	Methan  & \num{0.7168} \\
	Sauerstoff  & \num{1.429} \\
	Stickstoff  & \num{1.2505} \\
	Wasserstoff  & \num{0.0899} \\
	};

\end{tikzpicture}

Die Werte der Gase gelten bei \SI{0}{\celsius}.
\end{center}
