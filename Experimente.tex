%\documentclass[11pt,a4paper,titlepage,twoside]{article}
\documentclass[12pt,a4paper,twoside]{article}
%\documentclass[12pt,a4paper,twoside]{scrartcl}
\usepackage{mystyle}
\usepackage{gplot}
\usepackage{mechanik_v001}
\usepackage{foto_v001}


\usepackage[shell]{gnuplottex}
\usepackage{pgf}
\usepackage{pgfplots}
\usepackage{gnuplot-lua-tikz}
\pgfplotsset{compat=1.9}




%\author{Felix Binder}
\title{Experimente in der Physik}
\date{}



\begin{document}
\maketitle


\section{Warum sollte man Experimente im Physikunterricht durchführen?}

Experimente gehören zur Physik. Sie erlauben vorher getroffene Hypothesen zu bestätigen oder zu widerlegen.


\section{Wie sollte man Experimente im Physikunterricht durchführen?}

Die \SuS müssen wissen warum das Experiment durchgeführt wird.
Vor dem Experiment sollten Hypothesen aufgestellt werden, die dann mit dem Experiment bestätigt oder widerlegt werden.


\section{Verlauf eines Experiments im Physikunterricht}
\begin{itemize}
	\item Aufstellen einer Fragestellung
	\item Diskussion Pro und Contra
	\item Durchführen des Experiments zur Klärung
\end{itemize}


Vermuten

skizzieren, beschreiben

erklären


\end{document}
